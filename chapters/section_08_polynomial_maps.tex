\section{Polynomial Maps}


In this section $k$ is an infinite field.
Until further notice we also fix a finite-dimensional $k$-vector space $V$ with $k$-basis $v_1, \dotsc, v_n$.

\begin{definition}
  By $\mc{P}_k(V)$ we denote set of \emph{polynomial functions} $V \to k$, i.e.\ $f \in \mc{P}_k(V)$ if and only if $f \colon V \to k$ and there is some $p \in k[X_1, \dotsc, X_n]$ such that
  \[
      f\left( \sum_{i=1}^n \lambda_i v_i \right)
    = p(\lambda_1, \dotsc, \lambda_n)
  \]
  for all $\lambda_1, \dotsc, \lambda_n \in k$.
  If the underlying field is clear we also write $\mc{P}(V)$ instead of $\mc{P}_k(V)$.
\end{definition}


This definition does not depend on the chosen basis.
If $(w_1, \dotsc, w_n)$ is another basis of $V$ with $w_i = \sum_{j=1}^n a_{ij} v_j$ for $i = 1, \dotsc, n$ then
\begin{align*}
      f\left( \sum_{i=1}^n \lambda_i w_i \right)
  &=  f\left( \sum_{i,j=1}^n \lambda_i a_{ij} v_j \right)
   =  p\left( \sum_{i=1}^n \lambda_i a_{i1}, \dotsc, \sum_{i=1}^n \lambda_{i} a_{in} \right)  \\
  &=  p'(\lambda_1, \dotsc, \lambda_n)
\end{align*}
for some $p' \in k[X_1, \dotsc, X_n]$.
So if $f \colon V \to k$ is a polynomial in $(v_1, \dotsc, v_n)$ then also in $(w_1, \dotsc, w_n)$.

This has the effect that the restriction of a polynomial function to a vector subspace is again a polynomial function.

\begin{lemma}
  Let $V$ be a finite-dimensional $k$-vector space and $U \subseteq V$ a vector subspace.
  Then for every polynomial function $f \in \mc{P}(V)$ we have $f_{|U} \in \mc{P}(U)$.
\end{lemma}
\begin{proof}
  Let $v_1, \dotsc, v_n$, $v_{n+1}, \dotsc, v_m$ be a $k$-basis of $V$ such that $v_1, \dotsc, v_n$ is a $k$-basis of $U$.
  Because $f \in \mc{P}(V)$ there exist some $p \in k[X_1, \dotsc, X_m]$ with
  \[
      f\left( \sum_{i=1}^m \lambda_i v_i \right)
    = p(\lambda_1, \dotsc, \lambda_m)
  \]
  for all $\lambda_1, \dotsc, \lambda_m \in k$. For
  \[
              \bar{p}
    \coloneqq p(X_1, \dotsc, X_n, 0, \dotsc, 0)
    \in       k[X_1, \dotsc, X_n]
  \]
  we thus have
  \[
      f\left( \sum_{i=1}^n \lambda_i v_i \right)
    = \bar{p}(\lambda_1, \dotsc, \lambda_n)
  \]
  for all $\lambda_1, \dotsc, \lambda_n \in k$.
  Since $v_1, \dotsc, v_n$ is a $k$-basis of $U$ this is equivalent to $f_{|U} \in \mc{P}(U)$.
\end{proof}


\begin{remark}
  If a group $G$ acts linearly on $V$ then it acts linearly on $\mc{P}(V)$ by $(g.f)(v) = f\left(g^{-1}.v\right)$.
\end{remark}


\begin{lemma}
  There is an isomorphism of $k$-algebras
  \[
          \mc{P}(V)
    \cong k[X_1, \dotsc, X_n]
  \]
  where $n = \dim V$.
\end{lemma}
\begin{proof}
  For $1 \leq j \leq n$ define the $j$-th coordinate function (with respect to the chosen basis) as
  \[
            \varphi_j
    \colon  V \to k,
    \quad   \sum_{i=1}^n \lambda_i v_i
    \mapsto \lambda_j \,.
  \]
  By the universal property of the polynomial ring the assignment $X_j \to \varphi_j$ extends to a ring homomorphism
  \[
            \Phi
    \colon  k[X_1, \dotsc, X_n] \to \mc{P}(V),
    \quad   p
    \mapsto \Phi(p)
  \]
  where
  \[
      \Phi(p)\left( \sum_{i=1}^n \lambda_i v_i \right)
    = p(\lambda_1, \dotsc, \lambda_n) \,.
  \]
  Is it clear that $\Phi$ is surjective.
  It is left as an exercise to the reader to check that $\Phi$ is injective.
\end{proof}


\begin{lemma}
  \begin{enumerate}[label=\emph{\alph*)},leftmargin=*]
    \item
      Assume $p \in k[X_1, \dotsc, X_n]$ with $p(\lambda_1, \dotsc, \lambda_n) = 0$ for all $(\lambda_1,\dotsc,\lambda_n) \in k^n$.
      Then $p = 0$.
    \item
      The polynomial functions $\varphi_1, \dotsc, \varphi_n \in \mc{P}(V)$ are algebraically independent over $k$, i.e.\ if $f(\varphi_1, \dotsc, \varphi_n) = 0$ for some polynomial $f$ (over $k$) then $f = 0$.
  \end{enumerate}
\end{lemma}
\begin{proof}
  \begin{enumerate}[label=\emph{\alph*)},leftmargin=*]
    \item
      We show this by induction over $n$.
      
      ($n = 1$)
      Let $p \in k[X_1]$ with $p(\lambda_1) = 0$ for all $\lambda_1 \in k$.
      Since $k$ is infinite $p$ has infinitely many zeroes.
      Therefore $p = 0$.
      
      ($n \geq 2$)
      Assume the claim holds for $n-1$ and $1$.
      Consider $p \in k[X_1, \dotsc, X_n]$ with $p(\lambda_1, \dotsc, \lambda_n) = 0$ for all $(\lambda_1, \dotsc, \lambda_n) \in k^n$.
      We write $p$ as
      \[
          p
        = \sum_{i \in \Natural} f_i(X_1, \dotsc, X_{n-1}) X_n^i
      \]
      with $f_i \in k[X_1, \dotsc, X_{n-1}]$ for all $i \in \Natural$ and $f_i = 0$ for all but finitely many $i \in \Natural$.
      Let $(\lambda_1, \dotsc, \lambda_{n-1}) \in k^{n-1}$ be fixed but arbitrary.
      For all $\lambda_n \in k$ we have
      \[
          0
        = p(\lambda_1, \dotsc, \lambda_n)
        = \sum_{i \in \Natural} f_i(\lambda_1, \dotsc, \lambda_{n-1}) \lambda_n^i
      \]
      By induction hypothesis we find that $f_i(\lambda_1, \dotsc, \lambda_{n-1}) = 0$ for all $i \in \Natural$.
      Because $(\lambda_1, \dotsc, \lambda_{n-1})$ is fixed but arbitrary we can use the induction hypothesis to get that $f_i = 0$ for all $i \in \Natural$.
      So $p = 0$.
    \item
      Assume $f(\varphi_1, \dotsc, \varphi_n) = 0$. Then
      \[
          0
        = f(\varphi_1, \dotsc, \varphi_n)\left(\sum_{i=1}^n \lambda_i v_i\right)
        = f(\lambda_1, \dotsc, \lambda_n)
      \]
      for all $(\lambda_1, \dotsc, \lambda_n) \in k^n$.
      Therefore $f = 0$ by part a).
    \qedhere
  \end{enumerate}
\end{proof}

\begin{warning}
  The assumption that $k$ is infinite is necessary.
  If, for example, $p = X^2 + X \in \Finite_2[X]$, then $p(0) = p(1) = 0$, so $p(\lambda) = 0$ for all $\lambda \in \Finite_2$, but $p \neq 0$.
\end{warning}


\begin{corollary}
  The map $\Phi \colon k[X_1, \dotsc, X_n] \to \mc{P}(V), X_j \mapsto \varphi_j$ is injective.
\end{corollary}


Together with the exercise sheet we find that
\[
          \Phi
  \colon  k[X_1, \dotsc, X_n]
  \to     \mc{P}(V),
  \quad   X_j
  \mapsto \varphi_j
\]
is an isomorphism of $k$-algebras.


\begin{remark}
  For $k$ infinite we have an isomorphism of representations
  \[
            \Phi
    \colon  k[X_1, \dotsc, X_n]
    \to     \mc{P}(k^n),
    \quad   X_i
    \mapsto \varphi_i \,,
  \]
  where $S_n$ acts on $k[X_1, \dotsc, X_n]$ as usual by permuting the $X_i$ and on $\mc{P}(k^n)$ via
  \[
      (\sigma.f)(v)
    = f\left( \sigma^{-1}.v \right)
    \text{ for all }
    \sigma \in S_n,
    f \in \mc{P}(V),
    v \in k^n \,.
  \]
  We know that $\Phi$ is an isomorphism of $k$-vector spaces.
  It is $S_n$-equivariant, since for all $\sigma \in G, p = X_1^{\alpha_1} \dotsm X_n^{\alpha_n} \in k[X_1, \dotsc, X_n]$ and $v = (\lambda_1, \dotsc, \lambda_n) \in k^n$
  \begin{align*}
       \Phi(g.p)(v)
    &= \Phi\left( X_{g(1)}^{\alpha_1} \dotsm X_{g(n)}^{\alpha_n} \right)(v)
     = \lambda_{g(1)}^{\alpha_1} \dotsm \lambda_{g(n)}^{\alpha_n} \\
    &= \Phi(p)( \lambda_{g(1)}, \dotsc, \lambda_{g(n)} )
     = \Phi(p)\left( g^{-1}.(\lambda_1, \dotsc, \lambda_n) \right) \\
    &= (g.\Phi(p))(\lambda_1, \dotsc, \lambda_n) \,.
  \end{align*}
\end{remark}


\begin{definition}
  $f \in \mc{P}(V)$ is \emph{homogeneous of degree $d \in \Integer$} if $f(\lambda y) = \lambda^d f(y)$ for all $\lambda \in k, y \in V$.
  By definition the zero polynomial $f=0$ is homogeneous of degree $d$ for any $d \in \Integer$.
  For $d \in \Integer$ we set
  \[
              \mc{P}(V)_d
    \coloneqq \{
                f \in \mc{P}(V)
              \mid
                f \text{ is homogeneous of degree } d
              \} \,.
  \]
\end{definition}


\begin{lemma}
  \begin{enumerate}[label=\emph{\alph*)},leftmargin=*]
    \item
      $\mc{P}(V)_d$ is a $k$-vector space for all $d \in \Integer$ (via pointwise addition and scalar multiplication).
    \item
      If $f \in \mc{P}(V)_i$ and $g \in \mc{P}(V)_j$ then $fg \in \mc{P}(V)_{i+j}$, where the multiplication is given by pointwise multiplication.
  \end{enumerate}
\end{lemma}
\begin{proof}
  \begin{enumerate}[label=\emph{\alph*)},leftmargin=*]
    \item For $f_1, f_2 \in \mc{P}(V)_d$ we have
      \begin{align*}
            (f_1+f_2)(\lambda v)
        &=  f_1(\lambda v) + f_2(\lambda v)
         =  \lambda^d f_1(v) + \lambda^d f_2(v) \\
        &=  \lambda^d (f_1(v) + f_2(v))
         =  \lambda^d (f_1 + f_2)(v)
      \end{align*}
      for all $\lambda \in k$, $v \in V$, so $f_1 + f_2 \in \mc{P}(V)_d$. If $f \in \mc{P}(V)$ and $\mu \in k$ then
      \[
          (\mu f)(\lambda v)
        = \mu f(\lambda v)
        = \lambda^d \mu f(v)
        = \lambda^d (\mu f)(v) \,,
      \]
      so $\mu f \in \mc{P}(V)_d$.
    \item
      For all $\lambda \in k$ we have for all $v \in V$
      \[
          fg(\lambda v)
        = f(\lambda v) g(\lambda v)
        = \left( \lambda^i f(v) \right)\left( \lambda^j g(v) \right)
        = \lambda^{i+j} f(v) g(v)
        = \lambda^{i+j} (fg)(v) \,,
      \]
      and therefore $fg \in \mc{P}(V)_{i+j}$.
    \qedhere
  \end{enumerate}
\end{proof}

This shows that $\mc{P}(V)$ is a graded algebra via
\[
      \mc{P}(V)
    = \bigoplus_{d \in \Integer} \mc{P}(V)_d \,.
\]
This grading can be seen as inherted from $k[X_1, \dotsc, X_n]$ via the isomorphism $\Phi$. Since
\[
    (\lambda X_1)^{\alpha_1} \dotsm (\lambda X_n)^{\alpha_n}
  = \lambda^{\sum_{i=1}^n \alpha_i} X_1^{\alpha_1} \dotsm X_n^{\alpha_n}
\]
we obtain that a monomial of degree $d$ corresponds to a polynomial function $\Phi(p) \in \mc{P}(V)$ which is homogeneous of degree $d$.


Given a field extension $L/k$ and an $n$-dimensional $k$-vector space $W$ we have an isomorphism of $k$-algebras
\[
        \mc{P}_k(W)
  \cong k[X_1, \dotsc X_n]
\]
and therefore an isomorphism of $L$-algebras
\[
        \mc{P}_k(W)_L
  \cong k[X_1, \dotsc, X_n]_L
  \cong L[X_1, \dotsc, X_n] \,.
\]
Since $\dim_L W_L = \dim_k W = n$ we also have an isomorphism of $L$-algebras
\[
        \mc{P}_L(W_L)
  \cong L[X_1, \dotsc, X_n]
\]
Combining this we have an isomorphism of $L$-algebras $\mc{P}_L(W)_L \cong \mc{P}(W_L)$. Since the isomorphism
\[
        \mc{P}_k(W)
  \cong k[X_1, \dotsc X_n]
\]
depends on choosing a $k$-basis of $W$ and the isomorphism
\[
        \mc{P}_L(W_L)
  \cong L[X_1, \dotsc, X_n]
\]
depends on choosing an $L$-basis of $W_L$ we can not expect this isomorphism $\mc{P}_k(W)_L \cong \mc{P}(W_L)$ to be independent of such choice.
We will come back to this later.


We will know generalize the definition of polynomial functions on maps between arbitrary finite-dimensional $k$-vector spaces.
For this we also stop fixing $V$.


\begin{definition}
  Let $V$ and $W$ be finite dimensional $k$-vector spaces. A map
  \[
            f
    \colon  W
    \to     V
  \]
  is called a \emph{polynomial map} if, given a basis $v_1, \dotsc, v_n$ of $V$, the coordinate functions of $f$ are polynomial, i.e.\ there exist $f_1, \dotsc, f_n \in \mc{P}(W)$ with
  \[
      f(w)
    = \sum_{i=1}^n f_i(w) v_i
  \]
  for all $w \in W$.
  We write
  \[
              \Pol_k(V,W)
    \coloneqq \{
                        f
                \colon  V
                \to     W
              \mid
                f \text{ is a polynomial map}
              \} \,.
  \]
\end{definition}


\begin{remark}
  One can show (as for $\mc{P}_k(W)$) that the definition doesn’t depend on the chosen basis of $V$.
\end{remark}


\begin{remark}
  In the case of $V = k$ we have $\Pol_k(W,k) = \mc{P}_k(W)$.
\end{remark}


\begin{example}
  Let $W$ be a finite dimensional $k$-vector space. Then
  \[
            f
    \colon  W
    \to     W^{\otimes r},
    \quad   w
    \mapsto w \otimes \dotsb \otimes w \,.
  \]
  is a polynomial map.
  To see this choose a basis $w_1, \dotsc, w_n$ of $W$. Then the elements
  \[
      w_{\underline{i}}
    = w_{i_1} \otimes \dotsb \otimes w_{i_r}
    \text{ with }
        \underline{i}
    =   (i_1, \dotsc, i_r)
    \in \{1, \dotsc, n\}^r
  \]
  form a basis of $W^{\otimes r}$.
  For $w \in W$ with $w = \sum_{i=1}^r \lambda_i w_i$ we have
  \[
      f(w)
    = w \otimes \dotsb \otimes w
    =         \left( \sum_{i=1}^r \lambda_i w_i \right)
      \otimes \dotsb
      \otimes \left( \sum_{i=1}^r \lambda_i w_i \right)
    = \sum_{\underline{i}} \lambda_{i_1} \dotsm \lambda_{i_r} w_{\underline{i}} \,.
  \]
  For the polynomials
  \[
      p_{\underline{i}}
    = X_{i_1} \dotsm X_{i_r}
  \]
  and polynomial maps $f_{\underline{i}} \colon W \to k$ with
  \[
      f_{\underline{i}}\left( \sum_{i=1}^r \lambda_i w_i \right)
    = p_{\underline{i}}(\lambda_1, \dotsc, \lambda_r)
  \]
  we thus have
  \[
      f(w)
    = \sum_{\underline{i}} f_{\underline{i}}(w) w_{\underline{i}} \,.
  \]
\end{example}


\begin{lemma}
  For a finite-dimensional $k$-vector space $V$ the map $\id_V \colon V \to V$ is a polynomial map.
  If $U$ and $W$ are finite-dimensonial vector spaces and $f \colon W \to V$ and $g \colon V \to U$ are polynomial maps then $gf \colon W \to U$ is also a polynomial map.
\end{lemma}
\begin{proof}
  The first statement is clear.
  To show the second let $v_1, \dotsc, v_r$ be a basis of $V$, $w_1, \dotsc, w_s$ be a basis of $W$ and $u_1, \dotsc, u_t$ be a basis of $U$.
  Because $f$ is a polynomial map we can find polynomials $P_1, \dotsc, P_s \in k[X_1, \dotsc, X_r]$ such that
  \[
      f\left( \sum_{i=1}^r \lambda_i v_i \right)
    = \sum_{j=1}^s F_j(\lambda_1, \dotsc, \lambda_r) w_j
  \]
  and because $g$ is a polynomial map we can find polynomials $Q_1, \dotsc, Q_t \in k[X_1, \dotsc, X_s]$ such that
  \[
      g\left( \sum_{j=1}^s \mu_j w_j \right)
    = \sum_{k=1}^t G_k(\mu_1, \dotsc, \mu_s) u_k \,.
  \]
  Combining this we find that
  \begin{align*}
        gf\left( \sum_{i=1}^r \lambda_i v_i \right)
    &=  g\left( \sum_{j=1}^s F_j(\lambda_1, \dotsc, \lambda_r) w_j \right) \\
    &=  \sum_{k=1}^t Q_k(F_1(\lambda_1, \dotsc, \lambda_r), \dotsc, F_s(\lambda_1, \dotsc, \lambda_r)) w_k \\
    &=  \sum_{k=1}^t R_k(\lambda_1, \dotsc, \lambda_r) w_k
  \end{align*}
  for the polynomials
  \[
              R_k
    \coloneqq Q_k(F_1(X_1, \dotsc, X_r), \dotsc, F_s(X_1, \dotsc, X_r))
    \in       k[X_1, \dotsc, X_r] \,.
    \qedhere
  \]
\end{proof}


It is now easy to see that the class of finite-dimensional $k$-vector spaces together with the polynomial maps between them form a category.
We will denote this category by $\cpol{k}$.
So the objects in $\cpol{k}$ are finite-dimensional $k$-vector spaces and $\Hom_{\cpol{k}}(W,V) = \Pol_k(W,V)$ for all finite-dimensional $k$-vector spaces $W$ and $V$.
Also notice that every linear map between finite-dimensional $k$-vector spaces is a polynomial map.
Therefore $\cvect{k}$ is a subcategory of $\cpol{k}$.

From the definition of a polynomial map it also directly follows that for any finite-dimensional $k$-vector spaces $W$ and $V$ the set $\Pol_k(W,V)$ forms a $k$-vector space via pointwise addition and scalar multiplication.
This $k$-vector space can be given the structure of a $\mc{P}(W)$-module.

\begin{proposition}
  Let $W$ and $V$ be finite-dimensional $k$-vector spaces.
  The $k$-vector space $\Pol_k(W,V)$ is a $\mc{P}(W)$-module via
  \[
      (g \cdot f)(w)
    = g(w) f(w)
  \]
  for all $g \in \Pol_k(W,V)$,
  $f \in \mc{P}(W)$,
  $w \in W$.
\end{proposition}
\begin{proof}
  It is clear that $\Maps(W,V)$ becomes a $\Maps(W,k)$-module by defining the multiplication as above.
  Since $\mc{P}(W)$ is a $k$-subalgebra of $\Maps(W,k)$ we find that $\Maps(W,V)$ is a $\mc{P}(W)$-module.
  The proposition claims that $\Pol_k(W,V)$ is a $\mc{P}(W)$-submodule of $\Maps(W,V)$.
  So we only need to check that $\Pol_k(W,V)$ is closed under the multiplication from $\mc{P}(W)$.
  
  Let $v_1, \dotsc, v_n$ be a $k$-basis of $V$.
  For $f \in \Pol_k(W,V)$ there exists $f_1, \dotsc, f_n \in \mc{P}(W)$ such that
  \[
      f(w)
    = \sum_{i=1}^n f_i(w) v_i
    \text{ for all }
    w \in W \,.
  \]
  Since for all $g \in \mc{P}(W)$
  \begin{align*}
        (g \cdot f)(w)
    &=  g(w) \cdot f(w)
     =  g(w) \cdot \sum_{i=1}^n f_i(w) v_i \\
    &=  \sum_{i=1}^n g(w) f_i(w) v_i
     =  \sum_{i=1}^n (g \cdot f_i)(w) v_i
  \end{align*}
  we find that $g \cdot f \in \Pol_k(W,V)$ for all $g \in \mc{P}(W)$.
\end{proof}



\begin{lemma}
  Let $W$ and $V$ be finite-dimensional $k$-vector spaces and $f \colon W \to V$ be a polynomial map. Then
  \[
            f^*
    \colon  \mc{P}(V)
    \to     \mc{P}(W),
    \quad   h
    \mapsto h \circ f
  \]
  is an algebra homomorphism.
\end{lemma}
\begin{proof}
  We already know that $f^*$ is well-defined, i.e.\ that $h \circ f \in \mc{P}(W)$ for all $h \in \mc{P}(V)$.
  So we just need to show that $f^*$ is an algebra-homomorphism.
  For this fix a basis $v_1, \dotsc, v_n$ of $V$.
  
  Let $h_1, h_2\in \mc{P}(V)$, i.e.\ we have polynomials $p_1, p_2\in k[X_1, \dotsc, X_n]$ with
  \[
      h_j\left( \sum_{i=1}^n \lambda_i v_i \right)
    = p_j(\lambda_1, \dotsc, \lambda_n)
    \text{ for }
    j = 1, 2 \,.
  \]
  For all $w \in W$ we have
  \begin{align*}
        f^*(h_1+h_2)(w)
    &=  ((h_1 + h_2) \circ f)(w)
     =  (h_1 + h_2)(f(w)) \\
    &=  h_1(f(w)) + h_2(f(w))
     =  (h_1 \circ f)(w) + (h_2 \circ f)(w) \\
    &=  f^*(h_1)(w) + f^*(h_2)(w)
     =  (f^*(h_1)+f^*(h_2))(w)
  \end{align*}
  and therefore
  \[
      f^*(h_1 + h_2)
    = f^*(h_1) + f^*(h_2) \,.
  \]
  For all $\lambda \in k$ and $w \in W$ we have
  \begin{align*}
        f^*(\lambda h_1)(w)
    &=  ((\lambda h_1) \circ f)(w)
     =  (\lambda h_1)(f(w)) \\
    &=  \lambda h_1(f(w))
     =  \lambda (h_1 \circ f)(w) \\
    &=  \lambda f^*(h_1)(w)
     =  (\lambda f^*(h_1))(w)
  \end{align*}
  and therefore
  \[
      f^*(\lambda h_1)
    = \lambda f^*(h_1) \,.
  \]
  This shows that $f^*$ is $k$-linear. That it is also a ring homomorphism follows from the fact that for all $w \in W$
  \begin{align*}
        f^*(h_1 h_2)(w)
    &=  ((h_1 h_2) \circ f)(w)
     =  (h_1 h_2)(f(w)) \\
    &=  h_1(f(w)) h_2(f(w))
     =  (h_1 \circ f)(w) (h_2 \circ f)(w) \\
    &=  f^*(h_1)(w) f^*(h_2)(w)
     =  (f^*(h_1) f^*(h_2))(w)
  \end{align*}
  and therefore
  \[
      f^*(h_1 h_2)
    = f^*(h_1) f^*(h_2) \,.
    \qedhere
  \]
\end{proof}


\begin{definition}
  Given finite-dimensional $k$-vector spaces $W$ and $V$ and a polynomial map $f \colon W \to V$ the algebra homomorphism $f^* \colon \mc{P}(V) \to \mc{P}(W)$ is the \emph{comorphism associated with $f$}.
\end{definition}
  

We have now associated finite-dimensional $k$-vector spaces with $k$-algebras and the polynomial maps between these vector spaces with algebra homomorphisms between the corresponding algebras.
This gives rise to a contravariant functor from $\cpol{k}$ to $\cAlg{k}$, as the next lemma shows.


\begin{proposition}
  Let $U$, $V$ and $W$ be finite-dimensional $k$-vector spaces.
  We have $\id_V^* = \id_{\mc{P}(V)}$ and for polynomial maps $f \colon W \to V$ and $g \colon V \to U$ we have
  \[
      (g \circ f)^*
    = f^* \circ g^* \,.
  \]
\end{proposition}
\begin{proof}
  The first statement is clear.
  The second holds because for all $h \in \mc{P}(U)$
  \begin{align*}
      ( g \circ f)^*(h)
    &=  h \circ (g \circ f)
     =  (h \circ g) \circ f \\
    &=  f^* (h \circ g)
     =  f^*(g^*(h))
     = (f^* \circ g^*)(h) \,.
    \qedhere
  \end{align*}
\end{proof}


It is interesting to notice that this functor is fully faithful.


\begin{proposition}
  Let $W$ and $V$ be finite-dimensional $k$-vector spaces.
  Then the map
  \[
            \Omega
    \colon  \Pol_k(W,V)
    \to     \Hom_{\cAlg{k}}(\mc{P}(V), \mc{P}(W)),
    \quad   f
    \mapsto f^*
  \]
  is a bijection.
\end{proposition}
\begin{proof}
  Fix a basis $v_1, \dotsc, v_n$ of $V$.
  Remember that $\mc{P}(V) = k[\varphi_1, \dotsc, \varphi_n]$ where $\varphi_j \in \mc{P}(V)$ is the $j$-th coordinate function, which is defined as
  \[
      \varphi_j\left( \sum_{i=1}^n \lambda_i v_i \right)
    = \lambda_j \,,
\]
  and $\varphi_1, \dotsc, \varphi_n$ are algebraically independent.
  In particular the map
  \[
            \Psi
    \colon  \mc{P}(W)^n
    \to     \Hom_{\cAlg{k}}(\mc{P}(V),\mc{P}(W)),
    \quad   (f_1, \dotsc, f_n)
    \mapsto (\varphi_j \mapsto f_j)
  \]
  is bijective (this is basically the universal property of the polynomial ring).
  
  Because $v_1, \dotsc, v_n$ is a basis of $V$ we also find that the map
  \[
            \Phi
    \colon  \mc{P}(W)^n
    \to     \Pol_k(W,V),
    \quad   (f_1, \dotsc, f_n)
    \mapsto \left(
                      w
              \mapsto \sum_{i=1}^n f_i(w)v_i
            \right)
  \]
  is bijective.
  
  We notice that for a polynomial map $f \colon W \to V$ and $f_1, \dotsc, f_n \in \mc{P}(W)$ with
  \[
      f(w)
    = \sum_{i=1}^n f_i(w) v_i
    \text{ for all }
    w \in W
  \]
  we have for all $1 \leq j \leq n$, $w \in W$
  \[
      f^*(\varphi_j)(w)
    = (\varphi_j \circ f)(w)
    = \varphi_j\left( \sum_{i=1}^n f_i(w) v_i \right)
    = f_j(w)
  \]
  and therefore for all $1 \leq j \leq n$
  \[
      f^*(\varphi_j)
    = f_j \,.
  \]
  Therefore we have
  \[
      \Omega(\Phi(f_1, \dotsc, f_n))
    = \Omega(f)
    = f^*
    = \Psi(f_1, \dotsc, f_n) \,.
  \]
  
  We have shown that $\Omega \Phi = \Psi$, i.e. that the diagram
  \[
    \begin{tikzcd}
        {}
      & \mc{P}(W)^n
        \arrow[swap]{dl}{\Phi}
        \arrow{dr}{\Psi}
      & {}
      \\
        \Pol_k(W,V)
        \arrow{rr}{\Omega}
      & {}
      & \Hom_{\cAlg{k}}(\mc{P}(V),\mc{P}(W))
    \end{tikzcd}
  \]
  commutes.
  Because $\Phi$ and $\Psi$ are bijections it follows that $\Omega = \Psi \Phi^{-1}$ is a bijection.
\end{proof}
