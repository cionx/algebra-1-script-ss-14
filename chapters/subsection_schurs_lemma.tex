\subsection{Schur’s Lemma}


\begin{proposition}[Schur’s lemma]
  \label{proposition: schurs lemma for modules}
  Let $M, N$ be $R$-modules and let $f \colon M \to N$ be a nonzero homomorphism of $R$-modules.
  \begin{enumerate}
    \item
      \label{enumerate: Schur injective}
      If $M$ is simple then $f$ is injective.
    \item
      \label{enumerate: Schur surjective}
      If $N$ is simple then $f$ is surjective.
  \end{enumerate}
  Let $M, N$ be simple.
  \begin{enumerate}[resume]
    \item
      \label{enumerate: Schur bijective}
      The homomorphism $f$ is bijective.
    \item
      \label{enumerate: Schur endomorphism ring}
      The endomorphism ring $\End_R(M)$ is a skew field.
  \end{enumerate}
  If $R$ has the additional structure of a $k$-algebra then we also have the following:
  \begin{enumerate}[resume]
    \item
      \label{enumerate: Schur divison algebra}
      The endomorphism ring $\End_R(M)$ is a division $k$-algebra.
    \item
      \label{enumerate: Schur scalar for fd}
      If $k$ is algebraically closed and $M$ is finite-dimensional over $k$ then $\End_R(M) = k$.
  \end{enumerate}
\end{proposition}


\begin{proof}
  \leavevmode
  \begin{enumerate}
    \item
      The kernel $\ker(f)$ is a proper submodule of $M$, so $\ker(f) = 0$.
    \item
      The image $\im(f)$ is a nonzero submodule of $N$, so $\im(f) = N$.
    \item
      This follows from parts~\ref*{enumerate: Schur injective}, \ref*{enumerate: Schur surjective}.
    \item
      It follows from $M \neq 0$ that $\id_M \neq 0$.
      The claim is therefore a reformulation of part~\ref*{enumerate: Schur endomorphism ring}.
    \item
      This is a combination of part~\ref*{enumerate: Schur divison algebra} and the $k$-algebra structure of $\End_R(M)$.
    \item
      It follows from part~\ref*{enumerate: Schur scalar for fd} that $\End_R(M)$ is a finite-dimensional skew field extension.
      It follows that $\End_R(M) = k$ because $k$ is algebraically closed.
    \qedhere
  \end{enumerate}
\end{proof}


\begin{remark}
  \hyperref[proposition: Schurs lemma representations]{Schur’s~lemma for representation of groups} can be derived from \hyperref[proposition: schurs lemma for modules]{Schur’s~lemma for modules} by using the correspondence between $k$-representations of a group $G$ and modules over the group algebra $k[G]$.
\end{remark}


\begin{corollary}
  \label{corollary: no nonzero homomorphisms between disjoint semisimple modules}
  Let $M, M'$ be semisimple $R$-modules with $M = \sum_{i \in I} L_i$ and $N = \sum_{j \in J} L'_j$ for simple submodules $L_i \moduleleq M$, $L'_j \moduleleq M'$.
  If there exists a nonzero homomorphism of $R$-modules $f \colon M \to M'$ then $L_i \cong L'_j$ for some $i \in I$, $j \in J$.
\end{corollary}


\begin{proof}
  It follows that $\restrict{f}{L_i} \neq 0$ for some $i \in I$.
  By Remark~\ref{remark: can replace sum by direct sum for ss} we may assume that the sum $\sum_{j \in J} L'_j$ is direct by shrinking $J$ if necessary.
  It then follows from $\restrict{f}{L_i} \neq 0$ that there exists some $j \in J$ for which the decomposition
    \[
                        f'
    \colon              L_i
    \inclusion          M
    \xlongrightarrow{f} M'
    =                   \bigoplus_{k \in J} L'_k
    \projection         L'_j
  \]
  is nonzero.
  It then follows from \hyperref[proposition: schurs lemma for modules]{Schur’s Lemma} that $f'$ is an isomorphism, which shows that $L_i \cong L'_j$.
\end{proof}


\begin{corollary}
  \label{corollary: End is isomorphic to product of matrix rings Schur style}
  \leavevmode
  \begin{enumerate}
    \item
      Suppose that $M = M_1^{\oplus n_1} \oplus \dotsb \oplus M_r^{\oplus n_r}$ where $M_1, \dotsc, M_r$ are pairwise non-isomorphic simple $R$-modules and $n_1, \dotsc, n_r \geq 1$.
      Then
      \begin{align*}
                \End_R(M)
        &\cong  \End_R(M_1^{\oplus n_1}) \times \dotsb \times \End_R(M_r^{\oplus n_r})  \\
        &\cong  \Mat_{n_1}(D_1) \times \dotsb \times \Mat_{n_r}(D_r)
      \end{align*}
      as rings, where $D_i \defined \End_R(M_i)$ for every $i = 1, \dotsc, r$ and the first isomorphism is given by
      \[
                f
        \mapsto \left( \restrict*{f}{M_1^{\oplus n_1}} \,,\, \dotsc \,,\, \restrict*{f}{M_t^{\oplus n_t}} \right) \,.
      \]
    \item
      Suppose more generally that $M = \bigoplus_{i \in I} M_i^{\oplus J_i}$ where $M_i$, $i \in I$ are pairwise non-isomorphic simple $R$-modules and $J_i$, $i \in I$ are index sets.
      Then
      \[
              \End_R(M)
        \cong \prod_{i \in I} \End_R( M_i^{\oplus J_i} )
        \cong \prod_{i \in I} \Mat_{J_i}^{\cf}( D_i )
      \]
      as rings, where $D_i \defined \End_R(M_i)$ for all $i \in I$ and the first isomorphism is given by
      \[
                f
        \mapsto \left( \restrict*{f}{M_i^{\oplus J_i}} \right)_{i \in I} \,.
      \]
      Here $\Mat_{J_i}^{\cf}( D_i )$ denotes the ring of column finite $(J_i \times J_i)$-matrices with entries in $D_i$ (see Definition~\ref{definition: infinite matrices} and Lemma~\ref{lemma: structure on infinite matrices}).
  \end{enumerate}
  If $R$ is a $k$-algebra, then these are isomorphisms of $k$-algebras.
\end{corollary}


\begin{proof}
  \leavevmode
  \begin{enumerate}
    \item
      This follows from Corollary~\ref{corollary: decomposition of endomorphisms for orthogonal modules} because $\Hom_R(M_i, M_j) = 0$ for all $i \neq j$ by \hyperref[proposition: schurs lemma for modules]{Schur’s lemma}.
    \item
      The first isomorphism follows from Corollary~\ref{corollary: decomposition of endomorphism ring into product including sums} because $\Hom_R(M_i, M_j) = 0$ for all $i \neq j$ by \hyperref[proposition: schurs lemma for modules]{Schur’s lemma}.
      The second isomorphism then follows from Corollary~\ref{corollary: endomorphism ring of sum power of fg module} because each $M_i$ is cyclic and thus finitely generated.
    \qedhere
  \end{enumerate}
\end{proof}


% \begin{notation}
%   \label{notation: abuse of notation for schur}
%   In the situation of Corollary~\ref{corollary: End is isomorphic to product of matrix rings Schur style} we will often identify $\End_R(M)$ with $\End_R(M_1^{\oplus n_1}) \times \dotsb \times \End_R(M_r^{\oplus n_r})$ and by abuse of notation just write
%   \[
%       \End_R(M)
%     = \End_R(M_1^{\oplus n_1}) \times \dotsb \times \End_R(M_r^{\oplus n_r}) \,.
%   \]
% \end{notation}


\begin{remark}
  \label{remark: Schur for cardinality big enough}
  For an algebraically closed field $k$ we can ask ourselves how part~\ref*{enumerate: Schur scalar for fd} of \hyperref[proposition: schurs lemma for modules]{Schur’s lemma} generalizes to infinite-dimensional $A$-modules, where $A$ is a $k$-algebra.
  We fix a cardinality $\kappa$ and will examine under what conditions the following holds:
  \begin{enumerate}
    \item
      \label{enumerate: schur holds}
      It holds that $\End_A(M) = k$ for every $k$-algebra $A$ and every simple $A$-module $M$ of dimension $\dim_k M \leq \card \kappa$.
  \end{enumerate}
  
  Note that if $D/k$ is any proper skew field extension then for $A = D$, $M = D$ we have that $\End_A(M) \cong D^\op$ by Lemma~\ref{lemma: End_R(R) = Rop} with $k \subsetneq D^\op$.
  If on the other hand for any $k$-algebra $A$ and simple $A$-module $M$ the inclusion $k \subsetneq \End_A(M) \defines D$ is proper then we can regard $M$ as a left $D$-vector space via
  \[
      \varphi \cdot m
    = \varphi(m)
  \]
  for all $\varphi \in \End_R(M)$, $m \in M$ and it follows that
  \[
          \dim_k M
    =     \dim_k D \cdot \underbrace{\dim_D M}_{\geq 1}
    \geq  \dim_k D \,.
  \]
  Together this shows that condition~\ref*{enumerate: schur holds} is equivalent to the following condition:
  \begin{enumerate}[resume]
    \item 
      \label{enumerate: no proper skew field extension}
      There exist no proper skew field extension $D/k$ of degree $\dim_k D \leq \kappa$.
  \end{enumerate}
  A skew field extension $D/k$ is proper if and only if it is trancendental, i.e.\ if $D$ contains a trancendental element, because $k$ is algebraically closed.
  We can therefore reformulate condition~\ref*{enumerate: no proper skew field extension} as follows:
  \begin{enumerate}[resume]
    \item 
      There exist no trancendental skew field extension $D/k$ of degree $\dim_k D \leq \kappa$, i.e.\ every transcendental skew field extension $D/k$ has degree $\dim_k D > \kappa$.
  \end{enumerate}
  
  If $D/k$ is a trancendental skew field extension with $\varphi \in D$ trancendental over $k$ then it follows that $D$ contains a copy of the function field $k(t)$, namely $k(\varphi)$.
  It then follows that $\dim_k D \geq \dim_k k(t)$.
  We have on the other hand that $k(t)/k$ is a trancendental (skew) field extension.
  Together this shows that for any cardinality $\kappa'$ there exists a trancendental skew field extension $D/k$ of degree $\dim_k D = \kappa'$ if and only if $\kappa' \geq \dim_k k(t)$.
  This leads to the following entry to our list of equivalent conditions:
  \begin{enumerate}[resume]
    \item 
     It holds that $\kappa < \dim_k k(t)$.
  \end{enumerate}
  
  We can determine the dimension $\dim_k k(t)$ in question:
  The elements $1/(t - \lambda) \in k(t)$ with $\lambda \in k$ are linearly independent over $k$, so we find that $\dim_k k(t) \geq \card k$.
  It follows on the other hand from $k$ being infinite that
  \[
      \card k
    = \card k[t]
    = \card k(t) \,,
  \]
  from which it follows that $\dim_k k(t) \leq \card k$.
  This shows that $\dim_k k(t) = \card k$.
  With this we arrive at the last equivalent condition for our list:
  \begin{enumerate}[resume]
    \item
      It holds that $\kappa < \card k$.
  \end{enumerate}
  
  The above disucission is motivated by \cite{Quillen}, which in turn credits \cite{Dixmier}.
%   
%   Part~\ref*{enumerate: Schur scalar for fd} of \hyperref[proposition: schurs lemma for modules]{Schur’s lemma} holds true as long as the cardinality of the algebraically closed field $k$ is strictly larger than the $k$-dimension of $M$, i.e.\ as long as $\card k > \dim_k M$.
%   This generalizes \ref*{enumerate: morphism space is one-dimensional} because every algebraically closed field is infinite.
% 
%   We prove this generalization by showing that there exists some nonzero polynomal $p(t) \in k[t]$ with $p(f) = 0$ (i.e.\ that $f$ is algebraic over $k$).
%   This then shows that $\End_R(M)$ is an algebraic skew field extension of $k$, from which it follows that $\End_R(M) = k$ because $k$ is algebraically closed.
%   The idea of this proof is taken from \cite{Quillen}, where the argument is attributed to \cite{Dixmier}.
% 
%   If $p(f) \neq 0$ for every nonzero polynomial $p(t) \in k[t]$ then $p(f) \colon M \to M$ is an isomorphism for every nonzero $p(t) \in k[t]$ because $\End_R(M)$ is a skew field.
%   It follows that the $k$-vector space structure of $M$ can be extended to a $k(t)$-vector space structure given by
%   \[
%               \frac{p(t)}{q(t)} \cdot m
%     \defined  \left( p(f) q(f)^{-1} \right)(m)
%   \]
%   for all $p(t)/q(t) \in k(t)$, $m \in M$.
%   
%   Note that this is just the universal property of the localization:
%   The ring homomorphism $\varphi \colon k[t] \to \End_R(M)$, $p(t) \mapsto p(f)$ maps every element of $S \defined k[t] \smallsetminus \{0\}$ to a unit, and therefore induces a ring homomorphism
%   \[
%             \Phi
%     \colon  k(t)
%     =       S^{-1} k[t]
%     \to     \End_R(M) \,,
%     \quad   \frac{p(t)}{q(t)}
%     \mapsto p(f) q(f)^{-1} \,.
%   \]
%   By regarding $k(t)$ as a $k$-algebra and the map $\Phi$ as a homomorphism of $k$-algebras $k(t) \to \End_k(M)$, we find that $\Phi$ corresponds to a $k(t)$-module structure on $M$.
%   This is precisely the $k(t)$-vector space structure from above.
%   
%   It follows that
%   \[
%           \dim_k M
%     =     \dim_{k(t)} M \cdot \dim_k k(t)
%     \geq  1 \cdot \card k
%     =     \card k \,,
%   \]
%   where we are using the following facts from linear algebra:
%   \begin{itemize}
%     \item
%       For the first equality we use that if $(b_i)_{i \in I}$ is a $k(t)$-basis of $M$, and $(c_j)_{j \in J}$ is a $k$-basis of $k(t)$, then $(b_i c_j)_{i \in I, j \in J}$ is a $k$-basis of $M$.
%     \item
%       For the inequality we use that the elements $1/(t-\lambda)$ with $\lambda \in k$ are $k$-linearly independent in $k(t)$, so that $\dim_k k(t) \geq \card k$.
%     \item
%       That $\dim_{k(t)} M \geq 1$ follows from $M$ being nonzero.
%   \end{itemize}
%   This contradicts the assumption that $\card k > \dim_k M$, which shows that $p(f) \neq 0$ for some nonzero $p(t) \in k[t]$.
\end{remark}


\begin{example}
  \leavevmode
  \begin{enumerate}
    \item
      If $k$ is any algebraically closed field then $k$ is inifinite and we retrieve part~\ref*{enumerate: Schur scalar for fd} of \hyperref[proposition: schurs lemma for modules]{Schur’s lemma}.
    \item
      If $A$ is a $\Complex$-algebra and $M$ is a countable-dimensional simple $A$-module then $\End_A(M) = \Complex$.
    \item
      For the $\closure{\Rational}$-algebra $A \defined \closure{\Rational}(t)$ the $A$-module $M \defined A$ is simple and countable dimensional over $\overline{\Rational}$ but $\End_A(M) = \closure{\Rational}(t) \supsetneq \overline{\Rational}$.
  \end{enumerate}
\end{example}




