\section{Affine Algebraic Varieties as Spaces}
\label{section: geometry of affine algebraic varietes}


\begin{fluff}
  In this section we will see that affine algebraic varities can be regarded as geometric spaces in their own right.
\end{fluff}


\begin{conventions}
  In the following, $U, V, W$ are finite-dimensional $k$-vector spaces.
\end{conventions}


\begin{definition}
  Let $X \subseteq V$, $Y \subseteq W$ be affine algebraic varietes.
  A map $f \colon X \to Y$ is \emph{polynomial} if it is the restriction of a polynomial map $V \to W$.
  We denote by $\Pol(X,Y)$ the set of polynomial maps $X \to Y$.
\end{definition}


\begin{remark}
  Let $X \subseteq V$, $Y \subseteq W$, $Z \subseteq U$ be affine algebraic varieties..
  \begin{enumerate}
    \item
      A function $f \colon X \to k$ is polynomial if it is the restriction of a polynomial function $V \to k$.
    \item
      The identity $\id_X \colon X \to X$ is polynomial, and for all polynomials maps $f \colon X \to Y$, $g \colon Y \to Z$ their composition $g \circ f \colon X \to Z$ is again polynomial.
      
      It follows that the class of affine algebraic varieties over $k$ together with the polynomial maps between them form a category, which we will denote by $\cAff{k}$:
      The objects of $\cAff{k}$ are affinee algebraic varieties over $k$ and the $\Hom$-sets of $\cAff{k}$ are given by $\Hom_{\cAff{k}}(X,Y) = \Pol(X,Y)$ for all affine algebraic varieties $X, Y$ over $k$.
      
      Note that the category $\cpol{k}$ (see Remark~\ref{remark: category of polynomial vector spaces}) is a full subcategory of $\cAff{k}$.
    \item
      Given a basis $w_1, \dotsc, w_m$ of $W$, a map $f \colon X \to Y$ is polynomial if and only if it is polynomial in each coordinate, i.e.\ if and only if the functions $f_1, \dotsc, f_m \colon V \to k$ with $f(x) = f_1(x) w_1 + \dotsb + f_m(x) w_m$ are polynomial.
    \item
      The polynomial functions $f \colon X \to k$ form a $k$-algebra with pointwise addition and multiplication.
  \end{enumerate}
\end{remark}


\begin{definition}
  For an affine algebraic variety $X \subseteq V$ the \emph{coordinate ring of $X$}, denoted by $\mc{P}(X)$, is the $k$-algebra of polynomial functions $X \to k$, with addition and multiplication being done pointwise.
\end{definition}


\begin{remark}
  Other popular notations for the coordinate ring $\mc{P}(X)$ of an affine algebraic variety $X$ are $A(X)$, $\mc{O}(X)$ and $k[X]$.
\end{remark}


\begin{lemma}
  \label{lemma: coordinate ring as quotient}
  For every affine algebraic variety $X \subseteq V$ the map
  \[
            \mc{P}(V)/\mc{I}(X)
    \to     \mc{P}(X) \,,
    \quad   [f]
    \mapsto \restrict{f}{X}
  \]
  is an isomorphism of $k$-algebras.
\end{lemma}


\begin{proof}
  The map $\mc{P}(V) \to \mc{P}(X)$, $f \mapsto \restrict{f}{X}$ is a surjective homomorphism of $k$-algebras by construction of $\mc{P}(X)$, and that $\mc{I}(X)$ is its kernel is a reformulation of the definition of $\mc{I}(X)$.
\end{proof}


\begin{corollary}
  Let $X \subseteq V$ be an affine algebraic variety.
  \begin{enumerate}
    \item
      The coordinate ring $\mc{P}(X)$ is an integral domain if and only if $X$ is irreducible.
    \item
      The coordinate ring $\mc{P}(X)$ is a field if and only if $X = \{a\}$ is a singleton for some $a \in V$, in which case $\mc{P}(X) = k$.
  \end{enumerate}
\end{corollary}


\begin{proof}
  \leavevmode
  \begin{enumerate}
    \item
      The quotient $\mc{P}(X) \cong \mc{P}(V)/\mc{I}(X)$ is an integral domain if and only if the ideal $\mc{I}(X) \idealeq \mc{P}(V)$ is prime, which, Lemma~\ref{lemma: X is irreducible iff I(X) is prime}, is the case if and only if $X$ is irreducible by .
    \item
      The quotient $\mc{P}(X) \cong \mc{P}(V)/\mc{I}(X)$ is a field if and only if the ideal $\mc{I}(X)$ is a maximal ideal, which, by Lemma~\ref{lemma: correspence between points and vanishing maximal ideals}, holds if and only if $X = \{a\}$ is a singleton for some $a \in V$.
      Then $\mc{P}(X) = \mc{P}(\{x\})$ consists of all maps $\{x\} \to k$, so that $\mc{P}(\{x\}) = k$.
    \qedhere
  \end{enumerate}
\end{proof}


\begin{lemma}
  \label{lemma: functoriality of P for affine algebraic varieties}
  Let $X, Y, Z$ be affine algebraic varieties.
  \begin{enumerate}
    \item
      For every polynomial map $f \colon X \to Y$ the map $f^* \colon \mc{P}(Y) \to \mc{P}(X)$, $\varphi \mapsto \varphi \circ f$ is a well-defined homomorphism of $k$-algebras.
    \item
      We have that $\id_X^* = \id_{\mc{P}(X)}$.
    \item
      For all polynomial maps $f \colon X \to Y$, $g \colon Y \to Z$ we have that $(g \circ f)^* = f^* \circ g^*$.
  \end{enumerate}
\end{lemma}


\begin{remark}
  Lemma~\ref{lemma: functoriality of P for affine algebraic varieties} shows that the coordinate ring $\mc{P}$ defines a contravariant functor $\mc{P} \colon \cAff{k} \to \cAlg{k}$.
  Note that this is an extension of the previously defined functor $\mc{P}$ from Remark~\ref{fluff: functor P on polynomial vector spaces}.
  This functor turns out to be fully faithful, generalizing Propositon~\ref{proposition: P is fully faithful for polynomial vector spaces} to affine algebraic varieties:
\end{remark}


\begin{proposition}
  \label{proposition: P is fully faithful for affine varieties}
  Let $X \subseteq V$, $Y \subseteq W$ be affine algebraic varieties.
  Then the map
  \[
            \Pol(X, Y)
    \to     \Hom_{\cAlg{k}}(\mc{P}(Y), \mc{P}(X)),
    \quad   f
    \mapsto f^*
  \]
  is a bijection.
\end{proposition}


\begin{proof}
  For $Y = W$ the proof given for Propositon~\ref{proposition: P is fully faithful for polynomial vector spaces} still applies without any changes, simply replace $V$ by $X$.
  The general case follows from this special one:
  The inclusion $i \colon Y \to W$ is a polynomial map which results in the following diagram:
  \[
    \begin{tikzcd}
        \Pol(X,Y)
        \arrow{r}[above]{i_*}
        \arrow{d}[left]{\mc{P}_{X,Y}}
      & \Pol(X,W)
        \arrow{d}[right]{\mc{P}_{X,W}}
      \\
        \Hom_{\cAlg{k}}(\mc{P}(Y), \mc{P}(X))
        \arrow{r}[above]{\mc{P}(i)^*}
      & \Hom_{\cAlg{k}}(\mc{P}(W), \mc{P}(X))
    \end{tikzcd}
  \]
  This diagram commutes because for every $f \in \Pol(X,Y)$ we have that
  \[
      \mc{P}(i)^*( \mc{P}(f) )
    = \mc{P}(f) \circ \mc{P}(i)
    = \mc{P}( i \circ f )
    = \mc{P}( i_*(f) ) \,.
  \]
  Since we already know that $\mc{P}_{X,W}$ bijective it now sufficies to show that the image of $i_*$ corresponds to the image of $\mc{P}(i)^*$.
  
  The image of $i_*$ consists precisely of those polynomial maps $f \colon X \to W$ which restrict to a polynomial map $X \to Y$, which is the case if and only if $f(X) \subseteq Y$.
  
  When we identify the coordinate ring $\mc{P}(Y)$ with the quotient $\mc{P}(W)/\mc{I}(Y)$ as explained in Lemma~\ref{lemma: coordinate ring as quotient}, the homomorphism $\mc{P}(i) \colon \mc{P}(W) \to \mc{P}(Y)$, $f \mapsto f \circ i = \restrict{f}{Y}$ corresponds to the canonical projection $\mc{P}(W) \to \mc{P}(W)/\mc{I}(Y)$.
  It follows that the image of $\mc{P}(i)^*$ consists precisely of those $k$-algebra homomorphisms $F \colon \mc{P}(W) \to \mc{P}(X)$ which can be extended to an algebra homomorphisms $\mc{P}(W)/\mc{I}(Y) \to \mc{P}(X)$.
  This is the case if and only if $\mc{I}(Y) \subseteq \ker F$.
  
  We thus need to show a polynomial map $f \colon X \to W$ satisfies $f(X) \subseteq Y$ if and only if $\mc{I}(Y) \subseteq \ker f^*$.
  This holds because
  \[
          f(X) \subseteq Y
    \iff  f(X) \subseteq \mc{V}(\mc{I}(Y))
    \iff  \mc{I}(f(X)) \supseteq \mc{I}(Y)
    \iff  \ker f^* \supseteq \mc{I}(Y) \,,
  \]
  where we use that
  \begin{align*}
        \mc{I}(f(X))
    &=  \{
          g \in \mc{P}(W)
        \suchthat
          \restrict{g}{f(X)} = 0
        \}
      = \{
          g \in \mc{P}(W)
        \suchthat
          \restrict{(g \circ f)}{X} = 0
        \}
    \\
    &=  \{
          g \in \mc{P}(W)
        \suchthat
          g \circ f = 0
        \}
      = \{
          g \in \mc{P}(W)
        \suchthat
          f^*(g) = 0
        \}
      = \ker f^* \,.
  \end{align*}
  This finishes the proof.
\end{proof}


\begin{remark}
  The functor $\mc{P} \colon \cAff{k} \to \cAlg{k}$ is a contravariant embedding by Proposition~\ref{proposition: P is fully faithful for affine varieties}.
  It follows that $\cAff{k}$ is dual to a strictly full subcategory of $\cAlg{k}$.
  It follows from Lemma~\ref{lemma: coordinate ring as quotient} that this strictly full subcategory has as objects precisely those $k$-algebras which are isomorphic to a $k$-algebra of the form $k[X_1, \dotsc, X_n]/I$ where $I \idealeq k[X_1, \dotsc, X_n]$ is a vanishing ideal.
  
  If $k$ is algebraically closed then vanishing ideals are precisely radical ideals, and we get a nice description of the category dual to $\cAff{k}$:
\end{remark}


\begin{theorem}
  If $k$ is algebraically closed, then the functor $\mc{P} \colon \cAff{k} \to \cAlg{k}$ restrict to dualities between strictly full subcategories
  \[
    \left\{
      \begin{tabular}{c}
        affine algebraic \\
        varieties over $k$
      \end{tabular}
    \right\}
    \longto
    \left\{
      \begin{tabular}{c}
        finitely generated \\
        $k$-algebras which are \\
        reduced
      \end{tabular}
    \right\}
  \]
  and
  \[
    \left\{
      \begin{tabular}{c}
        irreducible \\
        affine algebraic \\
        varieties over $k$
      \end{tabular}
    \right\}
    \longto
    \left\{
      \begin{tabular}{c}
        finitely generated \\
        $k$-algebras which are \\
        integral domains
      \end{tabular}
    \right\} \,.
  \]
\end{theorem}


\begin{proof}
  A $k$-algebra $A$ is finitely generated if and only if $A \cong k[X_1, \dotsc, X_n]/I$ for some $n \geq 0$ and some ideal $I \idealeq k[X_1, \dotsc, X_n]$, and the ideal $I$ is radical (resp.\ prime) if and only if $k[X_1, \dotsc, X_n]/I \cong A$ is reduced (resp.\ an integral domain).
\end{proof}


\begin{remark}
  Let $k$ be algebraically closed.
  If $A$ is a finitely generated $k$-algebra then $A \cong k[X_1, \dotsc, X_n]/I$ for some $n \geq 1$ and ideal $I \idealeq k[X_1, \dotsc, X_n]$.
  If $A$ is reduced then the ideal $I$ is radical and it follows that $X \defined \mc{V}(I) \subseteq k^n$ is an affine variety with $\mc{I}(X) = I$.
  It then follows that
  \[
          \mc{P}(X)
    \cong k[X_1, \dotsc, X_n]/\mc{I}(X)
    =     k[X_1, \dotsc, X_n]/I
    \cong A \,.
  \]
  This construction can now be used to constructed an inverse of the duality $\mathcal{P}$.
\end{remark}


\begin{remark}
  \label{remark: five forms of Nullstellen}
  If $k$ is algebraically closed the one could also add another duality in the style of Theorem~\ref{theorem: big correspondence theorems}, namely
  \[
    \left\{
      \begin{tabular}{c}
        one-point \\
        affine algebraic \\
        varieties over $k$
      \end{tabular}
    \right\}
    \longto
    \left\{
      \begin{tabular}{c}
        finitely generated \\
        $k$-algebras which are \\
        fields
      \end{tabular}
    \right\} \,.
  \]
  The image of the left hand side under $\mc{P}$ is (up to isomorphism) just $k$ itself, and we retrieve \hyperref[corollary: finitely generated field extensions are finite]{Zariski’s lemma} from the geometric picture of Corollary~\ref{corollary: algebraically closed correspondence for algebraic subsets and points}.
  With this, we have alltogether shown the following implications:
  \[
    \begin{tikzpicture}[commutative diagrams/every diagram]
      \node (NS) at (-90:2.3cm) {
        \hyperref[theorem: nullstellensatz]{Nullstellensatz}
      };
      \node (SNS) at (-90-72:2cm) {
        \hyperref[theorem: strong nullstellensatz]{
        \begin{tabular}{c}
          strong \\
          Nullstellensatz
        \end{tabular}
        }
      } ;
      \node (COR) at (-90-2*72:2cm) {
        \makebox[5ex][r]{
          Corollary~\ref{corollary: algebraically closed correspondence for algebraic subsets and points}
        }
      };
      \node (ZL) at (-90-3*72:2cm) {
        \makebox[5ex][l]{
        \begin{tabular}{c}
          \hyperref[corollary: finitely generated field extensions are finite]{Zariski’s Lemma} \\
          for alg.\ closed fields
        \end{tabular}
        \phantom{y}
        }
      };
      \node (WNS) at (-90-4*72:2cm) {
        \hyperref[theorem: weak nullstellensatz]{
        \begin{tabular}{c}
          weak \\
          Nullstellensatz
        \end{tabular}
        }
      };
      \draw[implies-implies, double equal sign distance] (NS) -- (SNS);
      \draw[-implies, double equal sign distance] (SNS) -- (COR);
      \draw[-implies, double equal sign distance] (COR) -- (ZL);
      \draw[-implies, double equal sign distance] (ZL) -- (WNS);
      \draw[implies-implies, double equal sign distance] (WNS) -- (NS);
    \end{tikzpicture}
  \]
\end{remark}


\begin{definition}
  Let $X$ be an affine algebraic variety and let $Y \subseteq X$, $I \idealeq \mc{P}(V)$.
  \begin{enumerate}
    \item
      The set
      \[
          \mc{I}_X(Y)
        = \{
            f \in \mc{P}(X)
          \suchthat
            \text{$f(y) = 0$ for every $y \in Y$}
          \}
      \]
      is the \emph{vanishing ideal} of $Y$ in $\mc{P}(X)$.
    \item
      The ideal $I$ is a \emph{vanishing ideal} if $I$ is the vanishing ideal of some subset $Y \subseteq X$.
    \item
      The set
      \[
          \mc{V}_X(I)
        = \{
            x \in X
          \suchthat
            \text{$f(x) = 0$ for every $f \in I$}
          \}
      \]
      is the \emph{vanishing set} of $I$ in $X$.
  \end{enumerate}
\end{definition}


\begin{fluff}
  Let $V \subseteq X$ be an affine variety.
  We identify $\mc{P}(X)$ with $\mc{P}(V)/\mc{I}(X)$ as explained in Lemma~\ref{lemma: coordinate ring as quotient}.
  
  For every subset $Y \subseteq X$ the vanishing ideal $\mc{I}_X(Y)$ is then given by $\mc{I}(Y)/\mc{I}(X)$.
  (This shows in particular that $\mc{I}_X(Y)$ is indeed an ideal in $\mc{P}(X)$.)
  The subset $Y$ is closed in $X$ if and only if it is closed in $V$ because $X$ is closed in $V$.
  Whether $Y$ is irreducible, or just a singleton $Y = \{y\}$ also does not depend on whether we view $Y$ as a subspace of $X$ or of $V$.
  
  Every ideal $I \idealeq \mc{P}(X)$ is of the form $I = I'/\mc{I}(X)$ for a unique ideal $I' \idealeq \mc{P}(V)$ with $I' \supseteq \mc{I}(X)$, and we have that $\mc{V}_X(I) = \mc{V}(I')$.
  The ideal $I$ is reduced (resp.\ prime, resp.\ maximal) if and only if $I'$ is reduced (resp.\ prime, resp.\ maximal) because
  \[
          \mc{P}(X)/I
    =     ( \mc{P}(V)/\mc{I}(X) ) / ( I'/\mc{I}(X) )
    \cong \mc{P}(V)/I' \,.
  \]
  Moreover, $I$ is a vanishing ideal (in $\mc{P}(X)$) if and only if $I'$ is a vanishing ideal (in $\mc{P}(V)$).
  
  With this we find alltogether that the bijections from Theorem~\ref{theorem: big correspondence theorems} generalize to affine algebraic varieties:
\end{fluff}


\begin{theorem}
  \label{theorem: big correspondence theorems for affine variets}
  Let $X$ be an affine variety.
  \begin{enumerate}
    \item
      The maps $\mc{I}_X, \mc{V}_X$ restrict to the following mutually inverse bijections:
      \[
        \begin{matrix}
            \left\{
              \begin{tabular}{c}
                  algebraic subsets \\
                  $Y \subseteq X$
              \end{tabular}
            \right\}
          & \begin{tikzcd}[column sep = large]
                {}
                \arrow[shift left]{r}{\mc{I}_X}
              & {}
                \arrow[shift left]{l}{\mc{V}_X}
            \end{tikzcd}
          & \left\{
              \begin{tabular}{c}
                vanishing ideals \\
                $I \idealeq \mc{P}(X)$
              \end{tabular}
            \right\}
          \\
            {}
          & {}
          & {}
          \\
            \rotatebox[origin=c]{90}{$\subseteq$}
          & {}
          & \rotatebox[origin=c]{90}{$\subseteq$}
          \\
            {}
          & {}
          & {}
          \\
            \left\{
              \begin{tabular}{c}
                  irreducible \\
                  algebraic subsets \\
                  $Y \subseteq X$
              \end{tabular}
            \right\}
          & \begin{tikzcd}[column sep = large]
                {}
                \arrow[shift left]{r}{\mc{I}_X}
              & {}
                \arrow[shift left]{l}{\mc{V}_X}
            \end{tikzcd}
          & \left\{
              \begin{tabular}{c}
                vanishing ideals \\
                $\mf{p} \idealeq \mc{P}(X)$ \\
                which are prime
              \end{tabular}
            \right\}
          \\
            {}
          & {}
          & {}
          \\
            \rotatebox[origin=c]{90}{$\subseteq$}
          & {}
          & \rotatebox[origin=c]{90}{$\subseteq$}
          \\
            {}
          & {}
          & {}
          \\
            \left\{
              \begin{tabular}{c}
                points $x \in X$
              \end{tabular}
            \right\}
          & \begin{tikzcd}[column sep = large]
                {}
                \arrow[shift left]{r}{\mc{I}_X}
              & {}
                \arrow[shift left]{l}{\mc{V}_X}
            \end{tikzcd}
          & \left\{
              \begin{tabular}{c}
                vanishing ideals \\
                $\mf{m} \idealeq \mc{P}(X)$ \\
                which are maximal
              \end{tabular}
            \right\}
        \end{matrix}
      \]
    \item
      If $k$ is algebraically closed then the maps $\mc{I}_X, \mc{V}_X$ restrict to the following mutually inverse bijections:
      \[
        \begin{matrix}
            \left\{
              \begin{tabular}{c}
                  algebraic subsets \\
                  $Y \subseteq X$
              \end{tabular}
            \right\}
          & \begin{tikzcd}[column sep = large]
                {}
                \arrow[shift left]{r}{\mc{I}_X}
              & {}
                \arrow[shift left]{l}{\mc{V}_X}
            \end{tikzcd}
          & \left\{
              \begin{tabular}{c}
                radical ideals \\
                $I \idealeq \mc{P}(X)$
              \end{tabular}
            \right\}
          \\
            {}
          & {}
          & {}
          \\
            \rotatebox[origin=c]{90}{$\subseteq$}
          & {}
          & \rotatebox[origin=c]{90}{$\subseteq$}
          \\
            {}
          & {}
          & {}
          \\
            \left\{
              \begin{tabular}{c}
                  irreducible \\
                  algebraic subsets \\
                  $Y \subseteq X$
              \end{tabular}
            \right\}
          & \begin{tikzcd}[column sep = large]
                {}
                \arrow[shift left]{r}{\mc{I}_X}
              & {}
                \arrow[shift left]{l}{\mc{V}_X}
            \end{tikzcd}
          & \left\{
              \begin{tabular}{c}
                prime ideals \\
                $\mf{p} \idealeq \mc{P}X)$
              \end{tabular}
            \right\}
          \\
            {}
          & {}
          & {}
          \\
            \rotatebox[origin=c]{90}{$\subseteq$}
          & {}
          & \rotatebox[origin=c]{90}{$\subseteq$}
          \\
            {}
          & {}
          & {}
          \\
            \left\{
              \begin{tabular}{c}
                points $x \in X$
              \end{tabular}
            \right\}
          & \begin{tikzcd}[column sep = large]
                {}
                \arrow[shift left]{r}{\mc{I}_X}
              & {}
                \arrow[shift left]{l}{\mc{V}_X}
            \end{tikzcd}
          & \left\{
              \begin{tabular}{c}
                maximal ideals \\
                $\mf{m} \idealeq \mc{P}(X)$
              \end{tabular}
            \right\}
        \end{matrix}
      \]
  \end{enumerate}
\end{theorem}



% TODO: Jacobson Rings





\noindent\hrulefill \, Current progress of reworking these notes. \hrulefill




