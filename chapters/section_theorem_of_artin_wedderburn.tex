\section{Semisimple and Simple Rings}


\begin{conventions}
  In this section $R$ denotes a ring.
\end{conventions}





\subsection{Semisimple Rings}


\begin{definition}
  The ring $R$ is \emph{semisimple} if it is semisimple as an $R$-module.
\end{definition}


% If $R$ is semisimple then
% \[
%     R
%   = \bigoplus_{[E] \in \Irr(R)} R_E
% \]
% is a decomposition into (left) $R$-module by Corollary \ref{theorem: isotypical decomposition}.


\begin{example}
  \label{example: semisimple rings}
  \leavevmode
  \begin{enumerate}
    \item
      Fields are semisimple.
    \item
      If $G$ is a finite group and $k$ a field with $\kchar(k) \ndivides |G|$ then the group algebra $k[G]$ is semisimple as seen in Example~\ref{example: semisimple modules}.
    \item
      For a skew field $D$ the matrix ring $\Mat_n(D)$ is semisimple for all $n > 0$:
      We have seen in Example~\ref{example: simple modules} that $D^n$ is simple as an $\Mat_n(D)$-module.
      We now have that
      \[
          \Mat_n(D)
        = C_1 \oplus \dotsb \oplus C_n
      \]
      for the submodules $C_i \moduleeq \Mat_n(D)$ given by 
      \[
                  C_i
        \defined  \{
                    A \in \Mat_n(D)
                  \suchthat
                    \text{$A$ has nonzero entries only in the $i$-th column}
                  \} \,,
      \]
      and we have that $C_i \cong D^n$ for every $i = 1, \dotsc, n$.
  \end{enumerate}
\end{example}


\begin{proposition}
  If $R$ is semisimple then every $R$-module is semisimple.
\end{proposition}


\begin{proof}
  Every $R$-module is isomorphic to a quotient of a free $R$-moudule, so the claim follows from Lemma~\ref{lemma: inherit semisimple}.
\end{proof}


\begin{lemma}
  \label{lemma: simple module of semisimple ring is direct summand}
  Let $R$ be semisimple with $R = \bigoplus_{i \in I} L_i$ for simple submodules then $L_i \moduleeq R$.
  Then every simple $R$-module is isomorphic to some $L_i$.
\end{lemma}


\begin{proof}
  Let $E$ be a simple $R$-module and let $x \in E$ with $x \neq 0$.
  Then the map $R \to E$, $r \mapsto rx$ is a nonzero homomorphism of $R$-modules and the claim follows from Corollary~\ref{corollary: no nonzero homomorphisms between disjoint semisimple modules}.
\end{proof}


\begin{corollary}
  \label{corollary: D^n is the only simple M_n(D)-module}
  If $D$ is a skew field then $D^n$ is the only simple $\Mat_n(D)$-module up to isomorphism.
\end{corollary}


\begin{proof}
  This follows from Lemma~\ref{lemma: simple module of semisimple ring is direct summand} and the decompositon of $\Mat_n(D)$ into simple submodules from Example~\ref{example: semisimple rings}.
\end{proof}


\begin{lemma}
  \label{lemma: ring is already finite sum of submodules}
  Let $R$ be semisimple with $R = \sum_{i \in I} M_i$ for submodules $M_i \moduleeq R$.
  Then $R = \sum_{j \in J} M_j$ for some finite subset $J \subseteq I$.
\end{lemma}


\begin{proof}
  We can decompose $1 \in R$ as $1 = \sum_{i \in I} e_i$ with $e_i \in M_i$ for every $i \in I$ and $e_i = 0$ for all but finitely many $i \in I$.
  For
  \[
              J
    \defined  \{ i \in I \mid e_i \neq 0 \} \,.
  \]
  the sum $\sum_{j \in J} M_i$ is a submodule of $R$, i.e.\ an ideal in $R$, which therefore contains $1$.
  Thus $\sum_{j \in J} M_i = R$.
\end{proof}


\begin{corollary}
  \label{corollary: semisimple ring is already a finite sum}
  If $R$ is semisimple then $R$ is a finite direct of simple submodules.
\end{corollary}


\begin{proof}
 The claim follows by applying Lemma~\ref{lemma: ring is already finite sum of submodules} to a decomposition into simple submodules.
\end{proof}


\begin{corollary}
  If $R$ is a semisimple then there exist only finitely many simple $R$ modules up to isomorphism.
\end{corollary}


\begin{proof}
  This follows from Corollary~\ref{corollary: semisimple ring is already a finite sum} and Lemma~\ref{lemma: simple module of semisimple ring is direct summand}.
\end{proof}





\subsection{The Theorem of Artin-Wedderburn}


\begin{fluff}
  We now state and proof the theorem of Artin-Weddeburn, which classifies semisimple rings up to isomorphism.
\end{fluff}


\begin{definition}
  The \emph{opposite ring} $R^\op$ has the same underlying additive group as $R$, and the multiplication is given by
  \[
              a * b
    \defined  b \cdot a
    =         ba
  \]
  for all $a, b \in R^\op$, where $\cdot$ denotes the multiplication of $R$.
\end{definition}


\begin{remark}
  \label{remark: basic properties of op}
  \leavevmode
  \begin{enumerate}
    \item
      We have that $( R^\op )^\op = R$.
    \item
      The ring $R$ is commutative if and only if $R = R^\op$.
    \item
      If $D$ is a skew field then $D^\op$ is also a skew field.
    \item
      For every family of rings $(R_i)_{i \in I}$ we have that $( \prod_{i \in I} R_i )^\op = \prod_{i \in I} R_i^\op$.
  \end{enumerate}
\end{remark}


\begin{lemma}
  \label{lemma: op of matrix rings}
  The map
  \[
            \Mat_n(R)^\op
    \to     \Mat_n(R^\op)
    \quad   A
    \mapsto A^T
  \]
  is an isomorphism of rings.
\end{lemma}


\begin{proof}
  We denote the multiplications on $R$ and $\Mat_n(R)$ by $\cdot$, the multiplication of $R^\op$ by $*$ and the multiplication on $\Mat_n(R)^\op$ by $\bullet$.
  The given map is additive and bijective, and for all $A, B \in \Mat_n(R)^\op$ we have that
  \begin{align*}
        \left( (A \bullet B)^T \right)_{ij}
    &=  (A \bullet B)_{ji}
     =  (B \cdot A)_{ji}
     =  \sum_{k=1}^n B_{jk} \cdot A_{ki}  \\
    &=  \sum_{k=1}^n A_{ki} * B_{jk}
     =  \sum_{k=1}^n (A^T)_{ik} * (B^T)_{kj}
     =  \left( (A^T) \bullet (B^T) \right)_{ij}
  \end{align*}
  for all $i, j = 1, \dotsc, n$, and therefore $(A \bullet B)^T = A^T \bullet B^T$.
\end{proof}


% \begin{proposition}
%   \label{proposition: opposite modules}
%   Let $R$ be a ring.
%   We have a $1$:$1$-correspondence between the left $R$-modules and $R^\op$-modules, where for a left $R$-module $M$ the corresponding right $R^\op$-module $M^\op$ is defined as
%   \[
%               m \bullet r
%     \coloneqq r \cdot m
%     \text{ for every }
%     m \in M^\op \,,\,
%     r \in R^\op \,,
%   \]
%   where $\bullet$ denotes the multiplication of $R^\op$ on $M^\op$ and $\cdot$ the multiplication of $R$ on $M$.
% \end{proposition}


% \begin{proof}
%   <Insert obvious calculations here>.
% \end{proof}


\begin{lemma}
  \label{lemma: End_R(R) = Rop}
  The map
  \[
              R^\op
    \to       \End_R(R),
    \quad     r
    \mapsto   (x \mapsto xr)
  \]
  is an isomorphism of rings.
\end{lemma}


\begin{proof}
  We denote the given map by $\Phi$ and the multiplication of $R^\op$ by $*$.
  For every $r \in R^\op$ we have that
  \[
      \Phi(r'x)
    = r' x r
    = r' \Phi(x)
  \]
  for all $r' \in R$, $x \in R$, which shows that $\Phi(r)$ is $R$-linear for every $r \in R$.
  The additivity of $\Phi(r)$ for every $r \in R$ follows from the distributivity of $R$.
  This shows that $\Phi$ is well-defined.
  
  The additivity of $\Phi$ also follows from the distributivity of $R$, and we have that $\Phi(1_{R^\op}) = \Phi(1_R) = \id_R$.
  For all $r_1, r_2 \in R^\op$ we have that
  \[
      \Phi(r_1 * r_2)(x)
    = x (r_1 * r_2)
    = x r_2 r_1
    = \Phi(r_2)(x) r_1
    = \Phi(r_1)(\Phi(r_2)(x))
    = (\Phi(r_1) \circ \Phi(r_2))(x)
  \]
  for every $x \in R$ and thus $\Phi(r_1 * r_2) = \Phi(r_1) \circ \Phi(r_2)$.
  This shows that $\Phi$ is multiplicative.
  Alltogether this shows that $\Phi$ is a ring homomorphism.
  
  For every $r \in R^\op$ we have that $\Phi(r)(1) = r$, which shows that $\Phi$ is injective.
  For every $\varphi \in \End_R(R)$ we have for $r \defined \varphi(1)$ that
  \[
      \varphi(x)
    = \varphi(x \cdot 1)
    = x \cdot \varphi(1)
    = x r
    = \Phi(r)(x)
  \]
  and thus $\Phi(r) = \varphi$.
  This shows that $\Phi$ is surjective.
\end{proof}


% Let $D$ be a skew field.
% Becaus $D^n$ is a simple $\Mat_n(D)$-module we know from Schur’s Lemma that $\End_{\Mat_n(D)}(D^n)$ is a skew field.
% We would like to know how $D$ and $\End_{\Mat_n(D)}(D^n)$ are related.
% We know from linear algebra that for every field $k$ we have
% \[
%         \End_{\Mat_n(k)}(k^n)
%   \cong k.
% \]
% From Lemma \ref{lemma: End_R(R) = Rop} we also know that in the case $n = 1$. 
% \[
%   \End_D(D) \cong D^\op
% \]
% These observations lead to the following Lemma:
% 
% 
% \begin{lemma}
%   Let $D$ be a skew field and $n \geq 1$. Then
%   \[
%               \End_{\Mat_n(D)}\left(D^n\right)
%     \cong     D^\op,
%     \quad     \left(
%                         \vect{x_1 \\ \vdots \\ x_n}
%                 \mapsto \vect{x_1 d \\ \vdots \\ x_n d}
%               \right)
%     \mapsfrom d
%   \]
%   as rings.
% \end{lemma}
% \begin{proof}
%   By $\cdot$ we denote the multiplication in $D$ and by $*$ the multiplication in $D^\op$.
%   For all $d \in D$, $d' \in D^\op$ and $x = (x_1, \dotsc, x_n) \in D^n$ we write
%   \[
%               d x
%     \coloneqq (d x_1, \dotsc, d x_n)
%     \qquad \text{ and } \qquad
%               x d'
%     \coloneqq (x_1 d', \dotsc, x_n d'). 
%   \]
%   We also define
%   \[
%     \pi_i \colon D^n \to D
%   \]
%   as the canonical projection for every $1 \leq i \leq n$.
%   It is clear that $\pi_i$ is $D$-linear for every $1 \leq i \leq n$ where we see $D^n$ and $D$ as left $D$-modules in the usual way.
%   By $e_1, \dotsc, e_n$ we denote the standard basis of $D^n$ (as a left $D$-module).
%   
%   We define
%   \[
%             \varphi
%     \colon  D^\op
%     \to     \End_{\Mat_n(D)}\left( D^n \right),
%     \quad   d
%     \mapsto (x \mapsto x d) \,.
%   \]
%   It is clear that $\varphi$ is well-defined.
%   It is clear that $\varphi$ is additive.
%   That it is also multiplicative (and thus a ring homomorphism) follow from simple calculation:
%   For all $d, d' \in D^\op$ and $x \in D^n$ we have
%   \[
%       \varphi(d * d')(x)
%     = x (d * d')
%     = x (d' \cdot d)
%     = (x d') d
%     = \left( \varphi(d) \circ \varphi(d') \right)(x) \,.
%   \]
%   
%   It is also easy to see that $\varphi$ is injective:
%   For $d, d' \in D^\op$ with $\varphi(d) = \varphi(d')$ we have
%   \[
%       d
%     = \pi_1(e_1 d)
%     = \pi_1(\varphi(d)(e_1))
%     = \pi_1(\varphi(d')(e_1))
%     = \pi_1(e_1 d')
%     = d' \,.
%   \]
%   
%   All that’s left to show is that $\varphi$ is surjective.
%   For this let $f \in \End_{\Mat_n(D)}(D^n)$.
%   $f$ is $D$-linear, because for all $d \in D$ and $x = (x_1, \dotsc, x_n) \in D^n$
%   \[
%       f(dx)
%     = f( \diag(d, \dotsc, d) x)
%     = \diag(d, \dotsc, d) f(x)
%     = d f(x) \,.
%   \]
%   For every $1 \leq i \leq n$ we set $d_i \coloneqq \pi_i(f(e_i))$.
%   We then have for every $1 \leq i \leq n$
%   \[
%       f(e_i)
%     = f(E_{ii} e_i)
%     = E_{ii} f(e_i)
%     = (0, \dotsc, d_i, \dotsc, 0)
%     = e_i d_i 
%   \]
%   and therefore for every $x = (x_1, \dotsc, x_n) \in D^n$
%   \begin{align*}
%         f(x)
%     &=  f(x_1 e_1 + \dotsb + x_n e_n)       \\
%     &=  f(x_1 e_1) + \dotsb + f(x_n e_n)    \\
%     &=  x_1 f(e_1) + \dotsb + x_n f(e_n)    \\
%     &=  x_1 e_1 d_1 + \dotsb + x_n e_n d_n  \\
%     &=  (x_1 d_1, \dotsc, x_n d_n)
%   \end{align*}
%   For every $1 \leq i,j \leq n$ we have
%   \begin{align*}
%         d_i
%     &=  \pi_i(e_i d_i)
%      =  \pi_i(f(e_i))
%      =  \pi_i(f(E_{ij} e_j)) \\
%     &=  \pi_i(E_{ij} f(e_j))
%      =  \pi_i(E_{ij} e_j d_j)
%      =  \pi_i(e_i d_j)
%      =  d_j \,.
%   \end{align*}   
%   This shows that $f = \varphi(d)$ for $d \coloneqq d_1 = \dotsb = d_n$.
% \end{proof}


\begin{theorem}[Artin--Wedderburn]
  \label{theorem: artin wedderburn theorem}
  Let $R$ be semisimple.
  \begin{enumerate}
    \item
      If
      \[
              R
        \cong M_1^{\oplus n_1} \oplus \dotsb \oplus M_r^{\oplus n_r}
      \]
      for $r \geq 0$, pairwise non-isomorphic simple $R$-modules $M_1, \dotsc, M_r$ and $n_1, \dotsc, n_r \geq 1$, then
      \[
              R
        \cong \Mat_{n_1}(D_1) \times \dotsb \times  \Mat_{n_r}(D_r)
      \]
      as rings with $D_i = \End(M_i)^\op$ for every $i = 1, \dotsc, r$.
      \item
      This decomposition is unique in the following sense:
      If
      \[
              R
        \cong \Mat_{n_1}(D_1) \times \dotsb \times  \Mat_{n_r}(D_r)
      \]
      for some $r \geq 0$, $n_1, \dotsc, n_r \geq 1$ and skew fields $D_1, \dotsc, D_r$, then the number $r$ is as above and the pairs $(n_1,D_1), \dotsc, (n_r,D_r)$ are as above up to permutation and isomorphism of the $D_i$.
  \end{enumerate}
\end{theorem}


\begin{proof}
  \leavevmode
  \begin{enumerate}
    \item
      It follows from Lemma~\ref{lemma: End_R(R) = Rop} and Corollary~\ref{corollary: End is isomorphic to product of matrix rings Schur style} that
      \[
                R^\op
        \cong   \End_R(R)
        \cong   \End_R(M_1^{\oplus n_1} \oplus \dotsb \oplus M_r^{\oplus n_r})
        \cong   \Mat_{n_1}(D_1) \times \dotsb \times \Mat_{n_r}(D_r) \,.
      \]
      It further follows from Remark~\ref{remark: basic properties of op} and Lemma~\ref{lemma: op of matrix rings} that
      \begin{align*}
                R
        =      (R^\op)^\op
        &\cong  \left( \Mat_{n_1}(D_1) \times \dotsb \times \Mat_{n_r}(D_r) \right)^\op \\
        &=      \Mat_{n_1}(D_1)^\op \times \dotsb \times \Mat_{n_r}(D_r)^\op  \\
        &\cong  \Mat_{n_1}(D_1^\op) \times \dotsb \times \Mat_{n_r}(D_r^\op) \,.
      \end{align*}
    \item
      Let $\varphi \colon R \to \Mat_{n'_1}(D'_1) \times \dotsb \times \Mat_{n'_{r'}}(D'_{r'})$ be such an isomorphism.
      
  \end{enumerate}
\end{proof}



Notice that by Corollary \ref{corollary: simple modules over product of matrix algebras} it follows that a ring of the above form has exactly $r$ simple modules up to isomorphism, namely $D_1^{n_1}, \dotsc, D_r^{n_r}$.


\begin{corollary}
  Let $R$ be a semisimple ring (with $1$) and $M$ a faithful $R$-module, i.e.\ $rm = 0$ for every $r \in R$ implies $m = 0$.
  Then the isotypical components of $M$ are all nonzero.
  In particular $M$ contains every simple $R$-module up to isomorphism.
\end{corollary}
\begin{proof}
  By Artin--Wedderburn we have
  \[
    R \cong M_{n_1}(D_1) \times \dotsb \times M_{n_s}(D_s)
  \]
  for $s \geq 1$, $n_1, \dotsc, n_s \geq 1$ and skew field $D_1, \dotsc, D_s$.
  Then $D_1^{n_1}, \dotsc, D_r^{n_r}$ are a complete set of representatives of $\Irr(R)$, where $(A_1, \dotsc, A_s) \in R$ acts on $x \in D_i^{n_i}$ by $(A_1, \dotsc, A_n) \cdot x = A_i x$.
  Since $R$ is semisimple so is $M$ and thus we have a decomposition $M \cong \bigoplus_{i=1}^s M_{D_i^{n_i}}$ into isotypical components.
  Suppose that $M_{D_i^{n_i}} = 0$ for some $1 \leq i \leq s$.
  Then every element $A \in M_{n_i}(D_i) \subseteq R$ acts by multiplication with zero on $M$, which contradicts the faithfulness of $M$.
  Thus the isotypical components $M_{D_i^{n_i}}$ are all nonzero.
\end{proof}




\begin{corollary}
  Let $R$ be a semisimple ring (with $1$).
  Then $R^\op$ is also semisimple.
\end{corollary}
\begin{proof}
  By Artin--Wedderburn we have
  \[
    R \cong \Mat_{n_1}(D_1) \times \dotsm \times \Mat_{n_r}(D_r)
  \]
  for $r \geq 1$, $n_1, \dotsc, n_r \geq 1$ and skew fields $D_1, \dotsc, D_r$.
  Therefore
  \begin{align*}
            R^\op
    &\cong  \left( \Mat_{n_1}(D_1) \times \dotsm \times \Mat_{n_r}(D_r) \right)^\op \\
    &=      \Mat_{n_1}(D_1)^\op \times \dotsm \times \Mat_{n_r}(D_r)^\op \\
    &=      \Mat_{n_1}\left( D_1^\op \right) \times \dotsm \times \Mat_{n_r}\left( D_r^\op \right).
  \end{align*}
  Since $D_1^\op, \dotsc, D_r^\op$ are skew fields we find that $R^\op$ is semisimple by Artin--Wedderburn.
\end{proof}


\begin{corollary}
  Let $A$ be a finite-dimensional semisimple $k$-algebra.
  Then $A$ has finitely many nonzero minimal left ideals (i.e.\ simple $A$-submodules) $I_1, \dotsc, I_r$ (up to isomorphism of left ideal) and
  \[
    A \cong \Mat_{n_1}(D_1) \times \dotsm \times \Mat_{n_r}(D_r)
  \]
  where $D_i = \End_A(I_i)^\op$.
\end{corollary}
\begin{proof}
  We will prove this later.
\end{proof}


\begin{corollary}\label{corollary: semisimple algebra product of matrix algebras over field}
  Let $k$ be an algebraically closed field and $A$ a finite-dimensional semisimple $k$-algebra.
  Then
  \[
    A \cong \Mat_{n_1}(k) \times \dotsm \times \Mat_{n_r}(k)
  \]
  as $k$-algebras for some $r \geq 1$ and $n_1, \dotsc, n_r \geq 1$.
\end{corollary}
\begin{proof}
  Using Artin--Wedderburn we find that we have an isomorphism of rings
  \[
    A \cong \Mat_{n_1}(D_1) \times \dotsm \times \Mat_{n_r}(D_r)
  \]
  for some $r \geq 1$, $n_1, \dotsc, n_r \geq 1$ and skew fields $D_1, \dotsc, D_r$ where
  \[
    D_i = \End_A(S_i)^\op
  \]
  for a simple $A$-module $S_i$ for every $1 \leq i \leq r$.
  By Proposition \ref{proposition: simple modules over finite-dimensional algebras} $\dim_k S_i < \infty$ and thus by Schur’s Lemma $D_i = k$ for every $1 \leq i \leq r$.
\end{proof}


% \begin{proof}[Proof of Artin--Wedderburn]
%   We start by showing the existance:
%   By Lemma \ref{lemma: semisimple ring with 1 only finitely many summands} we have $R = \bigoplus_{i = 1}^n L_i$ as $R$-modules for some $n \geq 1$ and simple submodules $L_i \subseteq R$.
%   By sorting these submoduls by isomorphism classes we get
%   \[
%     R \cong n_1 V_1 \oplus \dotsb \oplus n_r V_r
%   \]
%   for $n_1, \dotsc, n_r \geq 1$ and pairwise non-isomorphic simple $R$-modules $V_1, \dotsc, V_r$.
%   Since it is enough to prove the theorem for $n_1 V_1 \oplus \dotsb \oplus n_r V_r$ we will assume that $R = n_1 V_1 \oplus \dotsb \oplus n_r V_r$.
%   
%   By Schur’s Lemma we find for every $1 \leq i \leq r$
%   \[
%           \End_R(n_i V_i)
%     \cong \Mat_{n_i}(\End_R(V_i))
%   \]
%   where $\End_R(V_i)$ is a skew field, and
%   \[
%           \End_R(n_1 V_1 \oplus \dotsb \oplus n_r V_r)
%     \cong \End_R(n_1 V_1) \oplus \dotsb \oplus \End_R(n_r V_r)
%   \]
%   because the $V_i$ are pairwise non-isomorpic.
%   Using Lemma \ref{lemma: End_R(R) = Rop} we find that
%   \begin{align*}
%             R^\op
%     &\cong  \End_R(R) \\
%     &\cong  \End_R(n_1 V_1 \oplus \dotsb \oplus n_r V_r) \\
%     &\cong  \End_R(n_1 V_1) \times \dotsb \times \End_R(n_r V_r) \\
%     &\cong  \Mat_{n_1}(D_1) \times \dotsb \times \Mat_{n_r}(D_r)
%   \end{align*}
%   as rings for the skew field $D_i \coloneqq \End_R(V_i)$.
%   Therefore
%   \begin{align*}
%             R
%     &\cong  \left( R^\op \right)^\op \\
%     &\cong  \left( \Mat_{n_1}(D_1) \times \dotsb \times \Mat_{n_r}(D_r) \right)^\op \\
%     &=      \Mat_{n_1}(D_1)^\op \times \dotsb \times \Mat_{n_r}(D_r)^\op \\
%     &\cong  \Mat_{n_1}\left( D_1^\op \right) \times \dotsb \times \Mat_{n_r}\left( D_r^\op \right)
%   \end{align*}
%   as rings where $D_i^\op$ is a skew field for every $1 \leq i \leq r$.
%   
%   To see the uniquness let
%   \begin{align}
%             R
%     &\cong  \Mat_{n_1}(D_1) \times \dotsm \times \Mat_{n_r}(D_r) \,,
%     \label{eqn: artin wedderburn isomorphisms 1}
%   \shortintertext{and}
%             R
%     &\cong \Mat_{n'_1}(D'_1) \times \dotsm \times \Mat_{n'_s}(D'_s)
%     \label{eqn: artin wedderburn isomorphisms 2} 
%   \end{align}
%   for $r, s \geq 1$, $n_1, \dotsc, n_r, n'_1, \dotsc, n'_s \geq 1$ and skew fields $D_1, \dotsc, D_r$, $D'_1, \dotsc, D'_s$.
%   We start by noticing that
%   \[
%       r
%     = |\Irr(R)|
%     = s \,,
%   \]
%   so $r$ is unique.
%   Using the isomorphisms \eqref{eqn: artin wedderburn isomorphisms 1} and \eqref{eqn: artin wedderburn isomorphisms 2} of rings we can make $\Mat_{n_1}(D_1) \times \dotsm \times \Mat_{n_r}(D_r)$ and $\Mat_{n'_1}(D'_1) \times \dotsm \times \Mat_{n'_r}(D'_r)$ into $R$-modules, such that \eqref{eqn: artin wedderburn isomorphisms 1} and \eqref{eqn: artin wedderburn isomorphisms 2} are also isomorphisms of $R$-modules.
%   By decomposing $\Mat_{n_i}(D_i)$ into simple $R$-submodules (which are the same as simple $\Mat_{n_i}(D_i)$ submodules) $\Mat_{n_i}(D_i) = C^i_1 \oplus \dotsb \oplus C^i_{n_i}$ in the usual way (so $C^i_j$ are the matrices in $\Mat_{n_i}(D_i)$ for which all but the $j$-the column are zero and $C^i_j \cong D_i^{n_i}$) we get a decomposition
%   \[
%       \Mat_{n_1}(D_1) \times \dotsm \times \Mat_{n_r}(D_r)
%     = \bigoplus_{i=1}^r \bigoplus_{j=1}^{n_i} C^i_j
%   \]
%   into simple $R$-submodules.
%   In the same way we get a decomposition
%   \[
%       \Mat_{n'_1}(D'_1) \times \dotsm \times \Mat_{n'_r}(D'_r)
%     = \bigoplus_{i=1}^r \bigoplus_{j=1}^{n'_i} C'^i_j
%   \]
%   into simple $R$-submodules.
%   We know that $C^{i_1}_{j_1} \cong C^{i_2}_{j_2}$ as $R$-modules if and only if $i_1 = i_2$, the same goes for the $C'^i_j$.
%   In particular both $C^1_1, \dotsc, C^r_1$ and $C'^1_1, \dotsc, C'^r_1$ are a complete collection of representatives of $\Irr(R)$.
%   Since the $R$-endomorphism rings of the $C^i_j$ and $C'^i_j$ are skew fields by Schur’s Lemma we find by the theorem of Krull-Remak-Schmidt (which we will not prove in this lecture) that there exists a bijection
%   \[
%             \pi
%     \colon  \left\{
%               C^i_j
%             \,\middle|\,
%               1 \leq i \leq r, 
%               1 \leq j \leq n_i
%             \right\}
%     \to     \left\{
%               C'^i_j
%             \,\middle|\,
%               1 \leq i \leq r,
%               1 \leq j \leq n'_i
%             \right\}
%   \]
%   such that $\pi(C^i_j) \cong C^i_j$ for every $C^i_j$.
%   Since $\pi$ restricts to bijections between the isomorphism classes of the $C^i_j$ and $C'^i_j$ we find a bijection
%   \[
%             \tau
%     \colon  \{1, \dotsc, r\}
%     \to     \{1, \dotsc, r\}
%   \]
%   such that $\pi$ restricts to a bijection
%   \[
%             \pi_i
%     \colon  \{ C^i_1, \dotsc, C^i_{n_i} \}
%     \to     \{ C^{\tau(i)}_1, \dotsc, C^{\tau(i)}_{n'_i} \}
%   \]
%   for every $1 \leq i \leq r$.
%   Thus we find that $n_i = n'_i$.
%   Because we have $C^i_1 \cong C^{\tau(i)}_1$ for every $1 \leq i \leq r$ and
%   \[
%           \End_R(C^i_1)
%     \cong \End_{\Mat_{n_i}(D_i)}(C^i_1)
%     \cong \End_{\Mat_{n_i}(D_i)}(D^n)
%     \cong D_i^\op
%   \]
%   as well as $\End_R(C^{\tau(i)}_1) \cong D_{\tau(i)}'^\op$ we also find that
%   \[
%           D_i^\op
%     \cong D_{\tau(i)}'^\op
%   \]
%   and thus $D_i \cong D'_{\tau(i)}$.
% \end{proof}





\subsection{Simple Rings}


\begin{definition}
  A ring $R$ is called \emph{simple} if $R \neq 0$ and $R = R_E$ for some simple $R$-module $E$.
  In particular $R$ is semisimple.
\end{definition}


\begin{corollary}
\label{corollary: simple rings one simple module}
  Let $R$ be a simple ring.
  Then there is exactly one simple $R$-module up to isomorphism.
\end{corollary}
\begin{proof}
  Because $R$ is simple we have $R = M_F$ for some simple submodle $F \subseteq R$.
  For every simple $R$-module $E$ we have $E \cong F'$ for some simple $R$-module $F' \subseteq R$.
  Since $F' \subseteq M_F$ we have $F' \cong F$ and thus $E \cong F$.
\end{proof}


\begin{definition}
  A ring $R$ is called \emph{simple} if it’s only two-sided ideals are $R$ and $0$.
\end{definition}


\begin{warning}
  This definition of a simple ring is no equivalent to the last one:
  Earlier we defined a ring $R$ to be simple if it is semisimple and has precisely one simple module up to isomorphism.
  We will refer to these rings as \emph{simple according to definition 1}.
  Rings which are simple according to the new definition above will be referred to as just \emph{simple}.
\end{warning}


\begin{example}
  \begin{enumerate}[label=\emph{\alph*)},leftmargin=*]
    \item
      Let $D$ be a division ring and $n \geq 1$.
      We have already seen that $\Mat_n(D)$ is a simple according to definition 1.
      It is also simple:
      Let $I \subseteq \Mat_n(D)$ be a two-sided ideal with $I \neq 0$.
      Let $A = (a_{ij})_{1 \leq i,j \leq n} \in I$ with $A \neq 0$.
      Then $a_{ij} \neq 0$ for some $1 \leq i,j \leq n$.
      Therefore
      \[
          \diag\left( a_{ij}^{-1}, \dotsc, a_{ij}^{-1} \right) E_{ii} A E_{jj}
        = E_{ij} \in I
      \]
      and thus for every $1 \leq k,l \leq n$
      \[
            E_{kl}
        =   E_{ki} E_{ij} E_{jl}
        \in I \,.
      \]
      Since $I$ is a $D$-submodule of $\Mat_n(D)$ we find that $I = \Mat_n(D)$.
    \item
      The Weyl-algebra
      \[
          \mc{A}_2
        = k \gen{X,\partial} / (X \partial - \partial X - 1)
      \]
      is simple, but not simple according to definition 1.
  \end{enumerate}
\end{example}


\begin{warning}
  A simple ring $R$ is not necessarily simple as an $R$-module.
  A counterexample is $\Mat_n(D)$ for a skew field $D$ and $n \geq 2$.
\end{warning}


\begin{lemma}
  Let $R$ be a ring (with $1$).
  If $R$ is simple according to definition 1 it is also simple.
\end{lemma}
\begin{proof}
  Since $R$ is semisimple we have
  \[
    R \cong \Mat_{n_1}(D_1) \times \dotsb \times \Mat_{n_r}(D_r)
  \]
  for $r \geq 1$, $n_1, \dotsc, n_r \geq 1$ and skew fields $D_1, \dotsc, D_r$ by Artin--Wedderburn.
  Since $r = |\Irr(R)| = 1$ we have
  \[
    R \cong \Mat_n(D)
  \]
  for $n \geq 1$ and a skew field $D$.
\end{proof}


We can also ask ourselves under what conditions a simple ring $R$ is semisimple (and thus semisimple as an $R$-module). The following theorem by Wedderburn answers that question:


\begin{theorem}[Wedderburn]
  Let $R$ be a simple ring (with $1$). Then the following are equivalent:
  \begin{enumerate}[label=\emph{\roman*)},leftmargin=*]
    \item \label{enum: semisimple}
      $R$ is semisimple.
    \item \label{enum: left artian}
      $R$ is (left) artian.
    \item \label{enum: minimal left ideal}
      $R$ has a minimal left ideal $I \neq 0$.
    \item \label{enum: matrix ring over skew field}
      $R \cong \Mat_n(D)$ for some $n \in \Natural$ and skew field $D$.
  \end{enumerate}
\end{theorem}
\begin{proof}
  The equivalence of \ref{enum: semisimple} and \ref{enum: matrix ring over skew field} follows directly from Artin--Wedderburn.
  
  To show that \ref{enum: semisimple} implies \ref{enum: left artian} suppose that \ref{enum: semisimple} holds.
  Then $R = \bigoplus_{i=1}^s V_i$ where $V_i \subseteq R$ is a simple $R$-module for every $1 \leq i \leq s$.
  Then
  \[
              0
    \subseteq V_1
    \subseteq V_1 \oplus V_2
    \subseteq \dotsb
    \subseteq V_1 \oplus \dotsb \oplus V_s
    =         R
  \]
  is a composition series of $R$, so by the Jordan-Hölder theorem (which we will not prove in this lecture) every strictly decreasing chain of left ideals in $R$ stabilizes (after at most $s$ ideal).
  
  To see that \ref{enum: left artian} implies \ref{enum: minimal left ideal} notice that if \ref{enum: minimal left ideal} does not hold we have an infinite chain
  \[
                A
    \supsetneq  I_1
    \supsetneq  I_2
    \supsetneq  I_3
    \supsetneq  \dotso
  \]
  of strictly decreasing nonzero left ideals, which contradicts \ref{enum: left artian}.
  
  Last we show that \ref{enum: minimal left ideal} implies \ref{enum: semisimple}.
  Suppose that $I \neq 0$ is a minimal left ideal.
  Then for every $r \in R$ the left ideal $Ir$ is either zero or minimal (i.e.\ simple as an $R$-submodule), since the map
  \[
            \varphi
    \colon  I
    \to     Ir,
    \quad   x
    \mapsto xr
  \]
  is an epimorphism of $R$-modules and thus either zero or an isomorphism (since $I$ is a simple $R$ module).
  Now
  \[
              J
    \coloneqq \sum_{r \in R} Ir
    =         IR
  \]
  is a two-sided ideal in $R$ which is nonzero (because $0 \subsetneq I = I1 \subseteq J$), so $J = R$.
  This show that $R$ is the sum of simple submodules.
\end{proof}


\begin{corollary}
  Let $A$ be finite-dimensional simple $k$-algebra.
  Then $A$ is semisimple and $A \cong \Mat_n(D)$ for some skew field $D$ and $n \in \Natural$.
\end{corollary}
\begin{proof}
  Because $A$ is finite-dimenisonal it contains a minimal ideal $I \neq 0$.
  The rest follows from Wedderburn’s theorem.
\end{proof}


\begin{lemma}
  Let $D$ be a skew field and $n \geq 1$.
  Then $Z(D)$ is a field and
  \[
    Z(\Mat_n(D)) \cong Z(D)
  \]
  as rings.
\end{lemma}
\begin{proof}
  We start by showing that $Z(D)$ is a field.
  We know that $Z(D) \subseteq D$ is a commutative subring (with $1$).
  Since $0 \neq 1$ in $D$ we also have $0 \neq 1$ in $Z(D)$.
  $Z(D)$ is also an integral domain, since $D$ is.
  All that we need to show is that for every $x \in Z(D)$ we also have $x^{-1} \in Z(D)$.
  This is clear, because for every $y \in D$
  \[
      x^{-1} y
    = x^{-1} y x x^{-1}
    = x^{-1} x y x^{-1}
    = y x^{-1} \,.
  \]
  
  Next we show that $Z(\Mat_n(D)) \cong Z(D)$.
  For this let $A \in Z(\Mat_n(D))$.
  We first show that $A$ is a diagonal matrix.
  To see this let $\pi_{ij} \colon \Mat_n(D) \to D$ be the canonical projection of the $(i,j)$-th coordinate for all $1 \leq i,j \leq n$.
  For all $1 \leq i,j \leq n$ we have
  \[
      a_{ij}
    = \pi_{ij}(E_{ii} A_{ij} E_{jj})
    = \pi_{ij}(E_{ii} E_{jj} A)
    = \delta_{ij} a_{ij} \,,
  \]
  so $a_{ij} = 0$ for $i \neq j$.
  Let $d_1, \dotsc, d_n \in D$ with $A = \diag(d_1, \dotsc, d_n)$.
  For every $1 \leq i,j \leq n$ we have
  \begin{align*}
        d_i
    &=  \pi_{ii}(A E_{ii})
     =  \pi_{ii}(A E_{ij} E_{jj} E_{ji})
     =  \pi_{ii}(E_{ij} A E_{jj} E_{ji}) \\
    &=  \pi_{ii}(E_{ij} d_j E_{jj} E_{ji})
     =  \pi_{ii}(d_j E_{ij} E_{jj} E_{ji})
     =  \pi_{ii}(d_j E_{ii})
     =  d_j \,,
  \end{align*}
  so $A = \diag(d, \dotsc, d)$ for $d \coloneqq d_1 = \dotsb = d_n$.
  Since $A$ commutes with all diagonal matrices we have $d \in Z(D)$.
\end{proof}





\subsection{Central Simple \texorpdfstring{$k$}{k}-algebras}


\begin{definition}
  Let $k$ be a field.
  A $k$-algebra $A$ is called a \emph{central simple algebra (over $k$)} if $A$ is finite-dimensional, simple and $Z(A) = k$.
\end{definition}


\begin{lemma}\label{lemma: Z(A o B) = Z(A) o Z(B)}
  Let $A$ and $B$ be $k$-algebras.
  Then
  \[
      Z(A \otimes_k B)
    = Z(A) \otimes_k Z(B) \,.
  \]
\end{lemma}


\begin{recall}
  Let $k$ be a field $V$ and $W$ be $k$-vector spaces.
  We know from linear algebra that every element $x \in V \otimes_k W$ can be written as a finite sum of simple tensors $x = \sum_{i=1}^n v_i \otimes w_i$.
  Furthermore $v_1, \dotsc, v_n$ are unique if $w_1, \dotsc, w_n \in W$ are linearly independent.
  \begin{proof}
    We can assume w.l.o.g.\ that $W = \vspan_k \{w_1, \dotsc, w_n\}$.
    We have for every $1 \leq i \leq n$ a $k$-bilinear map
    \[
              s_i
      \colon  V \times W \to V,
      \quad   \left(v, \sum_{i=1}^n \lambda_i w_i\right)
      \mapsto \lambda_i v \,.
    \]
    and thus a $k$-linear map
    \[
              f_i
      \colon  V \otimes_k W
      \to     V,
      \quad   v \otimes w_j
      \mapsto \delta_{ij} v \,.
    \]
    For $x \in V \otimes_k W$ with $x = \sum_{j=1}^n v_j \otimes w_j = \sum_{j=1}^n v'_j \otimes w_j$ we have
    \[
        0
      =   \left( \sum_{j=1}^n v_j \otimes w_j \right)
        - \left( \sum_{j=1}^n v'_j \otimes w_j \right)
      = \sum_{j=1}^n (v_j - v'_j) \otimes w_j
    \]
    and therefore for every $1 \leq i \leq n$
    \[
        v_i - v'_i
      = f_i\left( \sum_{j=1}^n (v_j - v'_j) \otimes w_j\right)
      = f_i(0)
      = 0 \,.
      \qedhere
    \]
  \end{proof}
\end{recall}


\begin{proof}[Proof of the Lemma]
  It is clear that $Z(A) \otimes_k Z(B) \subseteq Z(A \otimes_k B)$.
  To show the other inclusion let $x \in Z(A \otimes_k B)$.
  We can write $x = \sum_{i=1}^n a_i \otimes b_i$.
  We can assume w.l.o.g.\ that both $a_1, \dotsc, a_n$ and $b_1, \dotsc, b_n$ are linearly independent.
  For every $a \in A$ we have
  \[
      \sum_{i=1}^n (a a_i) \otimes b_i
    = (a \otimes 1) x
    = x (a \otimes 1)
    = \sum_{i=1}^n (a_i a) \otimes b_i
  \]
  and thus $a_i a = a a_i$ (because $b_1, \dotsc, b_n$ are linearly independent).
  So $a_i \in Z(A)$ for every $1 \leq i \leq n$.
  In the same way we find that $b_1, \dotsc, b_n \in Z(B)$.
  This shows that $x \in Z(A) \otimes_k Z(B)$.
\end{proof}


\begin{proposition}
  Let $A$ and $B$ be central simple algebras over the same field $k$.
  Then $A \otimes_k B$ is a central simple algebra.
\end{proposition}
\begin{proof}
  Since both $A$ and $B$ are finite-dimensional the same holds for $A \otimes_k B$.
  By Lemma \ref{lemma: Z(A o B) = Z(A) o Z(B)} we have
  \[
      Z(A \otimes_k B)
    = Z(A) \otimes_k Z(B
    = k \otimes_k k
    = k \,.
  \]
  So we only need to show that $A \otimes_k B$ only contains $0$ and $A \otimes_k B$ as two-sided ideals.
  To show this let $I \subseteq A \otimes_k B$ be a two-sided ideal with $I \neq 0$.
  We can write every $u \in I$ as $u = \sum_{i=1}^n a_i \otimes b_i$ where $b_1, \dotsc, b_n$ are linearly independent.
  Let $u \in I$ with $u \neq 0$ such that $u$ can be written as above so that the number of summands is minimal with respect to all nonzero elements in $I$.
  Let
  \begin{equation}\label{eqn: u as a sum}
    u = \sum_{i=1}^n a_i \otimes b_i
  \end{equation}
  be such a sum.
  Since $n$ is minimal we have $a_1 \neq 0$.
  Therefore the two-sided ideal $A a_1 A \subseteq A$ is non-zero, so $A a_1 A = A$ because $A$ is simple.
  In particular there exists $c, c' \in A$ with $1 = c a_1 c'$.
  By multiplying \eqref{eqn: u as a sum} from the left with $(c \otimes 1)$ and from the right with $(c' \otimes 1)$ we see that the element
  \[
              x
    \coloneqq (c \otimes 1) u (c' \otimes 1)
    \in       I
  \]
  can be written as
  \begin{equation}\label{eqn: x as a sum}
        x
    =   1 \otimes b_1
      + a'_2 \otimes b_2
      + \dotsb
      + a'_n \otimes b_n
  \end{equation}
  where $b_1, \dotsc, b_n$ are linearly independent.
  In particular $x \neq 0$.
  For every $a \in A$ we have
  \[
        (a \otimes 1) x - x (a \otimes 1)
    =   (a a'_2 - a'_2 a) \otimes b_2
      + \dotsb
      + (a a'_n - a'_n a) \otimes b_2 \in I \,.
  \]
  By the minimality of $u$ we find that
  \[
      (a \otimes 1) x - x (a \otimes 1)
    = 0
  \]
  for every $a \in A$.
  Because $b_2, \dotsc, b_n$ are linearly independent it follows that $a a'_i - a'_i a = 0$ for all $a \in A$ and $2 \leq i \leq n$.
  So $a'_2, \dotsc, a'_n \in Z(A) = k$.
  Using \eqref{eqn: x as a sum} we find that $x = 1 \otimes b$ for some $b \in B$.
  Since $x \neq 0$ we also have $b \neq 0$.
  Because $B$ is simple we find that $BbB = B$ and therefore
  \[
              I
    \supseteq (1 \otimes B) x (1 \otimes B)
    =         1 \otimes (BbB)
    =         1 \otimes B \,.
  \]
  Using this we find that
  \[
              I
    \supseteq (A \otimes 1) (1 \otimes B)
    =         A \otimes_k B \,.
  \]
  So $0$ and $A \otimes_k B$ are the only two-sided ideals in $A \otimes_k B$.
\end{proof}
