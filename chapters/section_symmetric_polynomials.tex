\section{Symmetric Polynomials}


\begin{fluff}
  The symmetric group $S_n$ acts by $k$-algebra automorphisms on the polynomial ring $k[X_1, \dotsc, X_n]$ by
  \[
      \sigma.f(X_1, \dotsc, X_n)
    = f(X_{\sigma(1)}, \dotsc, X_{\sigma(n)}) \,.
  \]
  In this section we will be concerned by the $k$-algebra of invariants $k[X_1, \dotsc, X_n]^{S_n}$.
\end{fluff}


\begin{definition}
  Let $k$ be a field.
  The polynomial $f \in k[X_1, \dotsc, X_n]^{S_n}$ are \emph{symmetric}, and $k[X_1, \dotsc, X_n]^{S_n}$ is the \emph{ring of symmetric polynomials \textup(in $n$ variables\textup) \textup(over $k$\textup)}.
\end{definition}


\begin{example}
  \label{example: symmetric polynomials}
  In $k[X_1, X_2, X_3]$ we have the symmetric polynomials
  \begin{align*}
                p_2
    &\coloneqq  X_1^2 + X_2^2 + X_3^2 \,,
    \\
                h_2
    &\coloneqq  X_1^2 + X_1 X_2 + X_1 X_3 + X_2^2 + X_2 X_3 + X_3^2 \,,
    \\
                e_2
    &\coloneqq  X_1 X_2 + X_1 X_3 + X_2 X_3 \,,
    \\
                m_{(4,4,2)}
    &\coloneqq  X_1^4 X_2^2 X_3^2 + X_1^2 X_2^4 X_3^2 + X_1^2 X_2^2 X_3^4 \,.
  \end{align*}
  In the next subsections we will generalize these examples.
\end{example}


\begin{lemma}
  \label{lemma: symmetric iff all homogeneous parts are symmetric}
  With respect to the usual grading $k[X_1, \dotsc, X_n] = \bigoplus_{d \in \Natural} k[X_1, \dotsc, X_n]_d$ a polynomial $f \in k[X_1, \dotsc, X_n]$ is symmetric if and only if all of its homogeneous parts are symmetric.
\end{lemma}


\begin{proof}
  The decomposition $k[X_1, \dotsc, X_n] = \bigoplus_{d \geq 0} k[X_1, \dotsc, X_n]_d$ is a decomposition into subrepresentations of $S_n$, thus the claim follows from Lemma~\ref{lemma: direct sum and invariants commute}.
\end{proof}

\begin{fluff}
  In the following subsections we will consider families of symmetric polynomials which generalize the polynomials given in Example~\ref{example: symmetric polynomials}.
  
  We will start off with the so called \emph{elemantary symmetric polynomials}.
  We prove the famous \emph{fundamental theorem of symmetric functions}, which roughly states that every every symmetric polynomial can be uniquely expressed in terms of the elementary symmetric polynomials.
  
  We will then use the elementary symmetric polynomials to study other kinds of symmetric polynomials:
  Namely the \emph{complete homogeneous symmetric polynomials}, \emph{power sums} \emph{monomial symmetric polynomials}.
  Along the way we will also introduce \emph{partitions} as a natural way for labeling these different kinds of symmetric polynomials.
\end{fluff}


\begin{conventions}
  For this section we fix a number of variables $n \in \Natural$.
\end{conventions}




\input{chapters/subsection_the_fundamental_theorem_of_symmetric_functions}
\input{chapters/subsection_complete_homogeneous_symmetric_polynomials}
\input{chapters/subsection_power_symmetric_polynomials}
\input{chapters/subsection_partitions}
\input{chapters/subsection_monomial_symmetric_polynomials}
\input{chapters/subsection_other_symmetric_polynomials_associated_to_partitions}




