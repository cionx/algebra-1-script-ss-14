\section{Other Symmetric Polynomials Associated to Partitions}


\begin{definition}
  For a partition $\lambda = (\lambda_1, \dotsc, \lambda_r)$ the corresponding \emph{elementary symmetric polynomial} is given by
  \[
              e_\lambda
    \defined  e_{\lambda_1} \dotsm e_{\lambda_r} \,,
  \]
  the corresponding \emph{complete symmetric polynomial} is given by
  \[
              h_\lambda
    \defined  h_{\lambda_1} \dotsm h_{\lambda_r} \,,
  \]
  the corresponding \emph{power symmetric polynomial} is given by
  \[
              p_\lambda
    \defined  p_{\lambda_1} \dotsm p_{\lambda_r} \,.
  \]
\end{definition}


\begin{fluff}
  We know from the \hyperref[theorem: fundamental theorem of symmetric functions]{fundamental theorem of symmetric functions} that the elementary symmetric polynomials $e_1, \dotsc, e_n$ generate $k[X_1, \dotsc, X_n]^{S_n}$ as a $k$-algebra and are algebraically independent. This is equivalent to saying that the monomials in $e_1, \dotsc, e_n$, i.e.\
  \[
      e^{\,\mindex{\alpha}}
    = e_1^{\alpha_1} \dotsm e_n^{\alpha_n}
    \quad\text{with}\quad
      \mindex{\alpha}
    = (\alpha_1, \dotsc, \alpha_n)
  \]
  form a $k$-basis of $k[X_1, \dotsc, X_n]^{S_n}$.
  Note that $e^{\,\mindex{\alpha}}$ coincides with $e_\lambda$ for the partition
  \[
    \lambda
  = (
      \underbrace{n, \dotsc, n}_{\alpha_n},
      \underbrace{n-1, \dotsc, n-1}_{\alpha_{n-1}},
      \dotsc,
      \underbrace{1, \dotsc, 1}_{\alpha_1}
    ) \,.
  \]
  Also note that the above formula gives a bijection
  \[
    \{ \text{multi-indices $\mindex{\alpha} \in \Natural^n$} \}
    \longleftrightarrow
    \{ \text{partitions $\lambda \in \Par$ with $\lambda_1 \leq n$} \} \,.
  \]
  With this we arrive at the following reformulation of the \hyperref[theorem: fundamental theorem of symmetric functions]{fundamental theorem of symmetric functions}:
\end{fluff}


\begin{corollary}
  The symmetric polynomials
  \[
      e_\lambda
    \quad\text{with}\quad
      \lambda \in \Par \,,
      \lambda_1 \leq n
  \]
  form a $k$-basis of $k[X_1, \dotsc, X_n]^{S_n}$.
\end{corollary}


\begin{remark}
  We can show the same statements for the polynomials $h_\lambda$ since $h_1, \dotsc, h_n$ generate $k[X_1, \dotsc, X_n]^{S_n}$ as a $k$-algebra and are algebraically independent.
  If $k$ is a field with $\ringchar(k) = 0$ or $\ringchar(k) > n$ we can also show the same for the polynomials $p_\lambda$.
\end{remark}


% TODO (optional): Adding the basis of schur polynomials.




