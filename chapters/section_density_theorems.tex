\section{Centralizers and Density Theorems}


\begin{conventions}
  In this section $R$ denotes a ring.
\end{conventions}


\begin{definition}
  The \emph{centralizer} or \emph{commutant} of a subset $S \subseteq R$ is
  \[
      \centralizer_R(S)
    = \{
        r \in R
      \suchthat
        \text{$rs = sr$ for every $s \in S$}
      \} \,.
  \]
  The set $S''$ is the \emph{double centralizer} or \emph{double commutant} \emph{bicommutant} of $S$.
\end{definition}


\begin{definition}
  The \emph{center} of $R$ is
  \[
              \ringcenter(R)
    \defined  \centralizer_R(R)
    =         \{
                r \in R
              \suchthat
                \text{$rs = sr$ for every $s \in R$}
              \} \,.
  \]
\end{definition}


\begin{lemma}
  For every subset $S \subseteq R$ the centralizer $\centralizer_R(S)$ is a subring of $R$.
  The center $\ringcenter(R)$ in particular is a subring of $R$.
\end{lemma}


\begin{proof}
  We have that $1 \in \centralizer_R(S)$.
  For all $r_1, r_2 \in \centralizer_R(S)$ we have for every $s \in S$ that
  \[
      (r_1 + r_2)s
    = r_1 s + r_2 s
    = s r_1 + s r_2
    = s (r_1 + r_2)
  \]
  and therefore $r_1 + r_2 \in \centralizer_R(S)$, as well as
  \[
      r_1 r_2 s
    = r_1 s r_2
    = s r_1 r_2
  \]
  and therefore $r_1 r_2 \in \centralizer_R(S)$.
  For every $r \in \centralizer_R(S)$ we have for every $s \in S$ that
  \[
      (-r) s
    = - (rs)
    = - (sr)
    = s (-r)
  \]
  and therefore $-r \in \centralizer_R(S)$.
\end{proof}


\begin{notation}
  We will often denote the centralizer of $S \subseteq R$ by $S'$ instead of $\centralizer_R(S)$.
\end{notation}


\begin{lemma}
  \label{lemma: centralizes is Galois connection}
  Let $S, T \subseteq R$ be subsets.
  \begin{enumerate}
    \item
      If $S \subseteq T$ then $S' \supseteq T'$.
    \item
      We have that $S \subseteq T'$ if and only if $S' \supseteq T$.
  \end{enumerate}
\end{lemma}


\begin{proof}
  \begin{enumerate}[start = 2]
    \item
      Both conditions express that every $s \in S$ commutes with every $t \in T$.
    \qedhere
  \end{enumerate}
\end{proof}


\begin{corollary}
  Let $S \subseteq R$ be a subsets.
  Then $S \subseteq S''$ and $S''' = S'$.
\end{corollary}


\begin{proof}
  That $S \subseteq S''$ follows from $S' \supseteq S'$ by Lemma~\ref{lemma: centralizes is Galois connection}.
  It follows from $S \subseteq S''$ that $S' \supseteq S'''$ because $(-)'$ is order-reversing and it follows from $S'' \supseteq S''$ that $S' \subseteq S'''$ by Lemma~\ref{lemma: centralizes is Galois connection}.
\end{proof}


\begin{fluff}
  We will now try to give some motivation for the upcoming theorems as well as for some of the later sections:
  
  Let $M$ be an abelian group.
  Then $M$ is a left $\End_\Integer(M)$-module via
  \[
      f \cdot m
    = f(m) \,.
  \]
  
  Suppose that $R \subseteq \End_\Integer(M)$ is a subring.
  Then the abelian groups $M$ inherts an $R$-module structure by restriction.
  The centralizer $R'$ consists of all additive maps $f \colon M \to M$ such that
  \[
    f \circ r = r \circ f
  \]
  for all $r \in R$, which holds if and only if
  \[
    f(r \cdot m) = r \cdot f(m)
  \]
  for all $r \in R$, $m \in M$.
  We therefore have that $R' = \End_R(M)$.
  By applying this result to the subring $\End_R(M) \subseteq \End_\Integer(M)$ we find that
  \[
      R'' 
    = \End_{R'}(M)
    = \End_{\End_R(M)}(M) \,.
  \]
  The inclusion $R \subseteq R''$ tells us that every $r \in R$ acts on $M$ by $\End_R(M)$-endomorphisms.
  
  Suppose more generally that $M$ is an $R$-module.
  Then the $R$-module of $M$ corresponds to a ring homomorphism
  \[
            \Phi
    \colon  R
    \to     \End_\Integer(M) \,,
    \quad   r
    \mapsto (m \mapsto rm) \,. 
  \]
  By the above discussion we have that
  \[
      \im(\Phi)'
    = \End_{\im(\Phi)}(M)
    = \End_R(M) 
  \]
  as well as $\im(\Phi) \subseteq \End_{\End_R(M)}(M)$.
  By abuse of notation we will therefore also write
  \[
              R'
    \defined  \End_R(M) \,,
  \]
  even if $R$ itself is not a subring of $\End_\Integer(M)$.
  We then do not have that $R \subseteq R''$, but $\Phi$ restrict to a homomorphism $R \to R''$, which we will refer to as the \emph{canonical homomorphism}.
  
  We will be concerned with the following problems regarding centralizers:
  \begin{itemize}
    \item
      Suppose that $R \subseteq \End_\Integer(M)$ is a subring.
      Then under what conditions do we have that $R = R''$?
      That is, under what conditions does $R$ have the \emph{double centralizer property}?
    \item
      Suppose more generally that $M$ is an $R$-module.
      Then under what conditions does $R$ have the double centralizer property in the sense that the canonical homomorphism $R \xrightarrow{\Phi} R''$ is surjective?
    \item
      Suppose that $R, S \subseteq \End_\Integer(M)$ are subrings.
      Then under what conditions do we have that $R = S'$ and $S' = R$?
      That is, under what conditions do $R$ and $S$ \emph{centralize each other}?
    \item
      Suppose more generally that $M$ is both an $R$-module and and $S$-module and let $\Phi \colon R \to \End_\Integer(M)$ and $\Psi \colon S \to \End_\Integer(M)$ be the corresponding ring homomorphisms.
      Suppose further that the actions of $R, S$ on $M$ commute, i.e.\ such that
      \[
          r \cdot (s \cdot m)
        = s \cdot (r \cdot m)
      \]
      for all $r \in R$, $s \in S$, $m \in M$.
      Then $\im(\Phi) \subseteq S'$ and $\im(\Psi) \subseteq R'$.
      Under what conditions are these equalities, that is, under what conditions do $R$ and $S$ centralize each other?
  \end{itemize}
  
  We will see some partial results on this questions in the upcoming sections, mostly for $k$-algebras in combination with some semisimplicity conditions.
  We will finish this section by giving two classical results by Jacobson on these problems which holds for (semi)simple modules over arbitrary rings.
\end{fluff}


\begin{theorem}[1.\ Jacobson density theorem]
  Let $M$ be a semisimple $R$-module and let $\Phi \colon R \to R''$ be the canonical ring homomorphism.
  Then $R$ (or more precisely $\im(\Phi)$ is \enquote{dense} in $R''$ in the sense that for every $f \in R''$ and every finitel collection $m_1, \dotsc, m_t \in M$ there exists some $r \in R$ with
  \[
      f(m_i)
    = r \cdot m_i
  \]
  for all $i = 1, \dotsc, t$.
\end{theorem}


\begin{proof}
  We first consider the case $t = 1$:
  
  Let $m = m_1 \in M$, let $f \in R''$ and let $C$ be a direct complement of the cylic submodule $Rm \moduleleq M$.
  Let $\pi \colon M \to M$ be the projection onto $Rm$ along the decomposition $M = Rm \oplus C$.
  Then $\pi$ is $R$-linear, i.e.\ an element of $R'$, and it follows that $f$ and $\pi$ commute.
  It follows that
  \[
        f(m)
    =   f(\pi(m))
    =   \pi(f(m))
    \in Rm
  \]
  which shows that $f(m) = rm$ for some $r \in R$.
  
  Suppose now that $t \geq 2$ and let $f \in R''$.
  We extend $f$ to an additive map
  \[
              \hat{f}
    \defined  f^{\oplus t}
    \colon    M^{\oplus t}
    \to       M^{\oplus t} \,,
  \]
  which is in matrix form (see Appendix~\ref{appendix: homomorphisms between direct sums}) given by the diagonal matrix
  \[
      \hat{f}
    = \begin{bmatrix}
        f &         &   \\  
          & \ddots  &   \\
          &         & f
      \end{bmatrix}.
  \]
  It then follows from $f \in R'' = \End_{\End_R(M)}(M)$ that $\hat{f} \in \End_{\End_R(M^{\oplus t})}(M^{\oplus t})$:
  We can represent every $R$-linear map $g \colon M^{\oplus t} \to M^{\oplus t}$ as a matrix
  \[
      g
    = \begin{bmatrix}
        g_{11}  & \cdots  & g_{1n}  \\
        \vdots  & \ddots  & \vdots  \\
        g_{n1}  & \cdots  & g_{nn}
      \end{bmatrix}
  \]
  with entries $g_{ij} \in \End_R(M) = R'$.
  It then folllows that $f$ commutes with every $g_{ij}$.
  It further follows from this by the usual rules of matrix multiplication that $\hat{f}$ and $g$ commute.
  
  It follows from the previously considered case $t = 1$ that for all $m_1, \dotsc, m_t \in M$ there exists some $r \in R$ with
  \[
      (f(m_1), \dotsc, f(m_t))
    = \hat{f}(m_1, \dotsc, m_t)
    = r \cdot (m_1, \dotsc, m_t) \,.
  \]
  We find that $f(m_i) = r \cdot m_i$ for all $i = 1, \dotsc, t$.
\end{proof}


\noindent\hrulefill


\begin{remark}
  In the special case that $M = R$ this results into an isomorphism
  \begin{align*}
                R
    &\cong      \End_{\End_R(R)}(R) \,, \\
                r
    &\mapsto    (m \mapsto rm) \,,  \\
                \varphi(1)
    &\mapsfrom  \varphi \,.
  \end{align*}
\end{remark}


\begin{theorem}[2.\ Jacobson density theorem]
  Let $R$ be a ring (with $1$) and $N$ a simple $R$-module.
  Let $u_1, \dotsc, u_s \in N$ be linearly independent over $\End_R(N)$ and $v_1, \dotsc, v_n \in N$ arbitrary.
  Then there exists $r \in R$ with
  \[
      r u_i
    = v_i
    \text{ for all }
    1 \leq i \leq s \,.
  \]
  This is equivalent to saying that $N^s$ is generated by $(u_1, \dotsc, u_s)$ as an $R$-module.
\end{theorem}


\begin{proof}
  Let $x \coloneqq (u_1, \dotsc, u_s)$.
  Because $N^s$ is semisimple we have $N^s = Rx \oplus Q$ as $R$-modules for some $R$-submodule $Q \subseteq N^s$.
  Consider the projection (along this decomposition)
  \[
                        \pi
    \colon              N^s
    \twoheadrightarrow  Q
    \hookrightarrow     N^s \,.
  \]
  Then $\pi \in \End_R(N^s)$.
  $\pi$ is given as a matrix $(d_{ij})_{1 \leq i,j \leq s}$ with entries in $\End_R(N)$.
  Because $\pi(x) = 0$ and we have
  \[
      d_{i1} u_1 + \dotsc + d_{is} u_s
    = 0
    \text{ for all }
    1 \leq i \leq s \,.
  \]
  Since $u_1, \dotsc, u_s$ are linearly independent over $\End_R(N)$ we find that $d_{ij} = 0$ for all $1 \leq i,j \leq s$ and therefore $\pi = 0$.
  From this we find that $Q = 0$ and thus $Rx = N^s$.
\end{proof}


% \begin{lemma}
%   \label{lemma: k alg. closed and D/k f.d. division algebra then D=k}
%   Let $k$ be an algebraically closed field and $D$ a finite-dimensional division algebra over $k$.
%   Then $D = k$.
% \end{lemma}
% \begin{proof}
%   Let $a \in D$ with $a \neq 0$.
%   Because $\dim_k D < \infty$ we know that the elements $1$, $a$, $a^2$, $a^3$, \dots\ are linearly dependent.
%   So there exists $p \in k[X]$ with $p(a) = 0$.
%   Since $k$ is algebraically closed we have $p = \prod_{i=1}^n (X-a_i)$ for some $n \in \Natural$ and $a_1, \dotsc, a_n \in k$.
%   Since
%   \[
%       0
%     = p(a)
%     = \prod_{i=1}^n (a-a_i)
%   \]
%   we find that $a = a_i$ for some $1 \leq i \leq n$ and thus $a \in k$.
% \end{proof}
% 
% 
% \begin{remark}
%   That $k$ is algebraically closed is not only sufficient but also necessary.
%   To see this let $k$ be a field which is not algebraically closed and $f \in k[X]$ such that $\deg f > 1$ and $f$ has no zeroes (in $k$).
%   Then $L \coloneqq k[X]/(f)$ is a finite field extension $L/k$ with $L \supsetneq k$.
% \end{remark}
