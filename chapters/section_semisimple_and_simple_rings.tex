\section{Semisimple and Simple Rings}


\begin{fluff}
  In this section we introduce and classify semisimple rings.
\end{fluff}


\begin{conventions}
  In this section $R$ denotes a ring.
\end{conventions}


\begin{lemma}
  \label{lemma: center of matrix ring}
  For every $n \geq 0$ the map
  \[
            \ringcenter(R)
    \to     \ringcenter(\Mat_n(R))  \,,
    \quad   z
    \mapsto z I_n
  \]
  is an isomorphism of rings.
\end{lemma}


\begin{proof}
  The given map is a well-defined injective ring homomorphism and we need to show that it is surjective, i.e.\ that every $A \in \ringcenter(\Mat_n(R))$ is of the form $A = z I_n$ for some $z \in \ringcenter(R)$.
  
  We have for all $i = 1, \dotsc, n$ that
  \[
      \text{$i$-th column of $A$}
    = A e_i
    = A E_{ii} e_i
    = E_{ii} A e_i
    = \vect{ 0 & \cdots & 0 & a_{ii} & 0 & \cdots & 0}^T \,,
  \]
  which shows that $A$ is a diagonal matrix.
  We have for all $i,j = 1, \dotsc, n$ that
  \[
      A_{ii} E_{ij}
    = A E_{ij}
    = E_{ij} A
    = A_{jj} E_{ij}
  \]
  and therefore $A_{ii} = A_{jj}$.
  This shows that $A$ is a scalar matrix, i.e.\ that $A = zI_n$ for some $z \in R$.
  We have for every $r \in R$ that
  \[
      (rz) I_n
    = (r I_n) (z I_n)
    = (r I_n) A
    = A (r I_n)
    = (z I_n) (r I_n)
    = (z I_n) I_n \,,
  \]
  which shows that $rz = zr$ for every $r \in R$ and therefore that $z \in \ringcenter(R)$.
\end{proof}


\subsection{Semisimple Rings \& Artin--Wedderburn}


\begin{definition}
  The ring $R$ is \emph{semisimple} if it is semisimple as an $R$-module.
\end{definition}


\begin{example}
  \label{example: semisimple rings}
  \leavevmode
  \begin{enumerate}
    \item
      Fields and skew fields are semisimple.
    \item
      If $G$ is a finite group and $k$ a field with $\kchar(k) \ndivides |G|$ then the group algebra $k[G]$ is semisimple by \hyperref[theorem: maschkes theorem]{Maschke’s theorem} as seen in Example~\ref{example: semisimple modules}.
    \item
      For a skew field $D$ the matrix ring $\Mat_n(D)$ is semisimple for all $n > 0$:
      We have seen in Example~\ref{example: simple modules} that $D^n$ is simple as an $\Mat_n(D)$-module.
      We now have that
      \[
          \Mat_n(D)
        = C_1 \oplus \dotsb \oplus C_n
      \]
      for the submodules $C_i \moduleleq \Mat_n(D)$ given by 
      \[
                  C_i
        \defined  \{
                    A \in \Mat_n(D)
                  \suchthat
                    \text{$A$ has nonzero entries only in the $i$-th column}
                  \} \,,
      \]
      and we have that $C_i \cong D^n$ for every $i = 1, \dotsc, n$.
      
      Note that with respect to Corollary~\ref{corollary: correspondence idempotents and direct ideal decompositions} this decomposition corresponds to the complete set of parwise orthogonal idempotents $E_{11}, \dotsc, E_{nn} \in \Mat_n(D)$.
      Indeed, we have for every $i = 1, \dotsc, n$ that $C_i = \Mat_n(D) E_{ii}$.
  \end{enumerate}
\end{example}





\subsubsection{General Properties of Semisimple Rings}


\begin{proposition}
  The $R$ is semisimple if and only if every $R$-module is semisimple.
\end{proposition}


\begin{proof}
  If every $R$-module $M$ is semisimple then this holds in particular for $M = R$.
  Every $R$-module is isomorphic to a quotient of a free $R$-moudule, so if $R$ is semisimple then every $R$-module is semisimple by Lemma~\ref{lemma: inherit semisimple}.
\end{proof}


\begin{lemma}
  \label{lemma: simple module of semisimple ring is direct summand}
  Let $R$ be semisimple with $R = \bigoplus_{i \in I} L_i$ for simple submodules then $L_i \moduleleq R$.
  Then every simple $R$-module is isomorphic to some $L_i$.
\end{lemma}


\begin{proof}
  Let $E$ be a simple $R$-module and let $x \in E$ with $x \neq 0$.
  Then the map $R \to E$, $r \mapsto rx$ is a nonzero homomorphism of $R$-modules and the claim follows from Corollary~\ref{corollary: no nonzero homomorphisms between disjoint semisimple modules}.
\end{proof}


\begin{example}
  \label{example: D^n is the only simple M_n(D)-module}
  It follows from Lemma~\ref{lemma: simple module of semisimple ring is direct summand} and the decompositon of $\Mat_n(D)$ into simple submodules from Example~\ref{example: semisimple rings} that $D^n$ is the only simple $\Mat_n(D)$-module up to isomorphism
\end{example}


\begin{lemma}
  \label{lemma: ring is already finite sum of submodules}
  Let $R$ be semisimple with $R = \sum_{i \in I} M_i$ for submodules $M_i \moduleleq R$.
  Then $R = \sum_{j \in J} M_j$ for some finite subset $J \subseteq I$.
\end{lemma}


\begin{proof}
  We can decompose $1 \in R$ as $1 = \sum_{i \in I} e_i$ with $e_i \in M_i$ for every $i \in I$ and $e_i = 0$ for all but finitely many $i \in I$.
  For
  \[
              J
    \defined  \{ i \in I \mid e_i \neq 0 \} \,.
  \]
  the sum $\sum_{j \in J} M_i$ is a submodule of $R$, i.e.\ an ideal in $R$, which therefore contains $1$.
  Thus $\sum_{j \in J} M_i = R$.
\end{proof}


\begin{corollary}
  \label{corollary: semisimple ring is already a finite sum}
  If $R$ is semisimple then $R$ is a finite direct of simple submodules.
\end{corollary}


\begin{proof}
 The claim follows by applying Lemma~\ref{lemma: ring is already finite sum of submodules} to a decomposition into simple submodules.
\end{proof}


\begin{corollary}
  \label{corollary: ss rings have only finitely many simple modules}
  If $R$ is a semisimple then there exist only finitely many simple $R$ modules up to isomorphism.
\end{corollary}


\begin{proof}
  This follows from Corollary~\ref{corollary: semisimple ring is already a finite sum} and Lemma~\ref{lemma: simple module of semisimple ring is direct summand}.
\end{proof}


\begin{corollary}
  \label{corollary: semisimple rings are notherian artinian}
  Every semisimple ring is both noetherian and artinian.
\end{corollary}


\begin{proof}
  By using Corollary~\ref{corollary: ss rings have only finitely many simple modules} we may write
  \[
    R = L_1 \oplus \dotsb \oplus L_n
  \]
  for some simple submodules $L_i \moduleleq R$.
  It then follows that
  \[
                0
    \modulelneq L_1
    \modulelneq L_1 \oplus L_2
    \modulelneq \dotsb
    \modulelneq L_1 \oplus \dotsb \oplus L_n
    =           R
  \]
  is a composition series of $R$ of length $n$.
  It follows from the \hyperref[theorem: jordan hoelder theorem]{Jordan-Hölder theorem} that every strictly increasing (resp.\ strictly decreasing) sequence of submodules of $R$ stabilizes after at most $n$ steps (see Corollary~\ref{corollary: consequences of jordan hoelder}).
\end{proof}


\begin{fluff}
  The main goal of this subsection is to state and prove the theorem of Artin--Wedderburn, which classifies semisimple rings up to isomorphism.
  For this we will need some knowledge about the opposite ring $R^\op$, a brief introduction to which can be found in Appendix~\ref{appendix: the opposite ring}.
\end{fluff}





\subsubsection{Products of Matrix Rings over Skew Fields}


\begin{fluff}
  We start by taking a closer look at matrix rings over skew fields, and how products of those kind of rings behave.
  For this we will need some understanding of how modules over a products of rings $R_1 \times \dotsb \times R_n$ look like.
  An explanation of this can be found in appendix~\ref{appendix: modules over products of rings}.
  We will also use some of the notation introduced in this appendix.
\end{fluff}


\begin{proposition}
  \label{proposition: product of semisimple}
  Let $R_1, R_2$ be rings and let $M_i$ be an $R_i$-module for $i = 1, 2$.
  \begin{enumerate}
    \item
      \label{enumerate: when boxplus is simple}
      The $(R_1 \times R_2)$-module $M_1 \boxplus M_2$ is simple if and only if either ($M_1$ is a simple $R_1$-module and $M_2 = 0$) or ($M_1 = 0$ and $M_2$ is a simple $R_2$-module).
    \item
      The map
      \begin{align*}
                  \Irr(R_1) \amalg \Irr(R_2)
        &\longto  \Irr(R_1 \times R_2) \,,
        \\
                  [E]
        &\mapsto  \begin{cases}
                    E \boxplus 0  & \text{if $[E] \in \Irr(R_1)$} \,, \\
                    0 \boxplus E  & \text{if $[E] \in \Irr(R_2)$}
                  \end{cases}
      \end{align*}
      is a well-defined bijection.
    \item
      \label{enumerate: when boxplus is semisimple}
      The $(R_1 \times R_2)$-module $M_1 \boxplus M_2$ is semisimple if and only if $M_i$ is semisimple as an $R_i$-module for $i = 1, 2$.
    \item
      The ring $R_1 \times R_2$ is semisimple if and only if both $R_1$ and $R_2$ are semisimple.
  \end{enumerate}
\end{proposition}


\begin{proof}
  \leavevmode
  \begin{enumerate}
    \item
      Let $\mc{S}_i$ be the set of $R_i$-submodules of $M_i$ for $i = 1, 2$ and let $\mc{S}$ be the set of $(R_1 \times R_2)$-submodules of $M_1 \boxplus M_2$.
      The map
      \[
                  \mc{S}_1 \times \mc{S}_2
        \to      \mc{S},
        \quad    (N_1, N_2)
        \mapsto  N_1 \boxplus N_2
      \]
      is a bijection by Proposition~\ref{proposition: submodules of products over rings}, from which it follows that
      \[
        |\mc{S}| = |\mc{S}_1| \cdot |\mc{S}_2| \,.
      \]
      The $(R_1 \times R_2)$-module $M_1 \boxplus M_2$ is simple if and only if $|\mc{S}| = 2$.
      This is the case if and only if either ($|\mc{S}_1| = 2$ and $|\mc{S}_2| = 1$) or ($|\mc{S}_1| = 1$ and $|\mc{S}_2| = 2$), which is equivalent to ($M_1$ simple, $M_2 = 0$), resp.\ ($M_1 = 0$ and $M_2$ simple).
    \item
      This follows by restricting the bijection from Corollary~\ref{corollary: isomorphism classes of modules over products} according to part~\ref*{enumerate: when boxplus is simple}.
        \item
      This can be seen in two ways:
      
      \begin{itemize}
        \item
          Every submodule $N \moduleleq M_1 \boxplus M_2$ is of the form $N = N_1 \boxplus N_2$ for unique $R_i$-submodules $N_i \moduleleq M_i$ by Proposition~\ref{proposition: submodules of products over rings}.
          It thus follows from Corollary~\ref{corollary: direct summands for modules over products} that every submodule of $M_1 \boxplus M_2$ has a direct complement if and only if every submodules of $M_i$ has a direct complement for both $i = 1, 2$.
        \item
          Suppose that $M_1, M_2$ are semisimple.
          Then $M_i = \bigoplus_{j \in J_i} L^j_1$ for simple submodules $L^j_i \moduleleq M_i$.
          It then follows that
          \[
              M_1 \boxplus M_2
            = \left( \bigoplus_{j \in J_1} L^j_1 \right)
              \boxplus
              \left( \bigoplus_{j \in J_2} L^j_2 \right)
            = \bigoplus_{j \in J_1} (L^j_1 \boxplus 0)
              \oplus
              \bigoplus_{j \in J_2} (0 \boxplus L^j_2)
          \]
          is a decomposition into submodules which are simple by part~\ref*{enumerate: when boxplus is simple}.
          
          Suppose now that $M_1 \boxplus M_2$ is semisimple.
          Then there exists a decomposition $M_1 \boxplus M_2 = \bigoplus_{j \in J} L^j$ into simple submodules $L^j \moduleleq M_1 \boxplus M_2$.
          Every $L^j$ is of the form $L^j = L^j_1 \boxplus L^j_2$ for unique $R_i$-submodules $L^j_i \moduleleq M_i$ by Proposition~\ref{proposition: submodules of products over rings}.
          It follows from part~\ref*{enumerate: when boxplus is simple} that $J$ is the disjoint union of
          \[
              J_1
            = \{ j \in J \suchthat L^j_2 = 0 \}
            \quad\text{and}\quad
              J_2
            = \{ j \in J \suchthat L^j_1 = 0 \}
          \]
          and that $L^j_i$ is simple for every $j \in L_i$.
          It follows that
          \begingroup
          \allowdisplaybreaks
          \begin{align*}
                M_1 \boxplus M_2
            &=  \bigoplus_{j \in J} L^j
            =   \bigoplus_{j \in J} ( L^j_1 \boxplus  L^j_2 )
            \\
            &=  \left(
                  \bigoplus_{j \in J_1} ( L^j_1 \boxplus 0 )
                \right)
                \oplus
                \left(
                  \bigoplus_{j \in J_2} ( 0 \boxplus  L^j_2 )
                \right)
            \\
            &=  \left(
                  \left( \bigoplus_{j \in J_1} L^j_1 \right) \boxplus 0
                \right)
                \oplus
                \left(
                  0 \boxplus \left( \bigoplus_{j \in J_2} ( 0 \boxplus  L^j_2 ) \right)
                \right)
            \\
            &=  \left( \bigoplus_{j \in J_1} L^j_1 \right)
                \boxplus
                \left( \bigoplus_{j \in J_2} L^j_2 \right)
          \end{align*}
          \endgroup
          and therefore that $M_i = \bigoplus_{j \in J_i} L^j_i$ is a direct sum of simple modules.
      \end{itemize}
    \item
      We have that $R_1 \times R_2 = R_1 \boxplus R_2$ as $(R_1 \times R_2)$-modules.
      The claim therefore follows from part~\ref*{enumerate: when boxplus is semisimple}.
    \qedhere
  \end{enumerate}
\end{proof}


\begin{corollary}
  \label{corollary: artin wedderburn rings are semisimple}
  Let $D_1, \dotsc, D_r$ be skew fields and let $n_1, \dotsc, n_r \geq 1$.
  \begin{enumerate}
    \item
      The ring $R \defined  \Mat_{n_1}(D_1) \times \dotsb \times  \Mat_{n_r}(D_r)$ is semisimple.
    \item
      The $R$-modules $S_1, \dotsc, S_r$ with
      \[
                  S_i
        \defined  0 \boxplus \dotsb \boxplus 0 \boxplus D_i^{n_i} \boxplus 0 \boxplus \dotsb \boxplus 0
      \]
      where $D_i^{n_i}$ is in the $i$-th position form a set of representatives of the isomorphism classes of simple $R$-modules.
    \item
      We have that $R \cong \bigoplus_{i=1}^r S_i^{\oplus n_i}$ as $R$-modules.
  \end{enumerate}
\end{corollary}


\begin{fluff}
  We will also need the endomorphisms rings of the simple modules $S_1, \dotsc, S_r$ from Corollary~\ref{corollary: artin wedderburn rings are semisimple}.
\end{fluff}


\begin{lemma}
  \label{lemma: matrix vector space correspondence for skew fields}
  Let $D$ be a skew-field and $n \geq 1$.
  Then the map
  \[
            \Phi 
    \colon  D^\op
    \to     \End_{\Mat_n(D)}(D^n) \,,
    \quad   d
    \mapsto \left(
                      \vect{x_1 \\ \vdots \\ x_n}
              \mapsto \vect{x_1 \\ \vdots \\ x_n} d
              =       \vect{x_1 d \\ \vdots \\ x_n d}
            \right)
  \]
  is an isomorphism of rings.
\end{lemma}


\begin{proof}
  We denote the multiplication of $D^\op$ by $*$.
  
  The column space $D^n$ carries the structure of a right $D$-module via scalar multiplication from the right.
  This right $D$-module structure corresponds to a left $D^\op$-modules structure (see Proposition~\ref{proposition: left right modules under op}), which in turn corresponds to a ring homomorphism $\Phi' \colon D^\op \to \End_\Integer(D^n)$ as described above.
  For every matrix $A \in \Mat_n(D)$, vector $x \in D^n$ and scalar $d \in D^\op$ we have that
  \[
      A(xd)
    = Axd
    = (Ax)d \,,
  \]
  which shows that $\Phi'$ restrict to $\Phi \colon D^\op \to \End_{\Mat_n(D)}(D^n)$ as desired.
  
  It remains to show that $\Phi$ is bijective.
  For $d, d' \neq 1$ we have that
  \[
          \Phi(d)(e_1)
    =     e_1 d
    \neq  e_1 d'
    =     \Phi(d')(e_1) \,,
  \]
  which shows that $\Phi$ is injective.
  To see that $\Phi$ is surjective let $f \in \End_{\Mat_n(D)}(D^n)$.
  Let $A \in \Mat_n(D)$ be the matrix whose first column is $e_1$ and whose other columns are $0$, so that
  \[
      A
    = \begin{bmatrix}
        1       & 0       & \cdots  & 0       \\
        0       & 0       & \cdots  & 0       \\
        \vdots  & \vdots  & \ddots  & \vdots  \\
        0       & 0       & \cdots  & 0
      \end{bmatrix}.
  \]
  Then $A e_1 = e_1$ and therefore
  \[
      A f(e_1)
    = f(A e_1)
    = f(e_1) \,,
  \]
  which shows that $f(e_1)$ is of the form
  \[
      f(e_1)
    = \vect{d \\ 0 \\ \vdots \\ 0}
    = e_1 d
  \]
  for some $d \in D$.
  For every other $x \in D^n$ there exists some $A \in \Mat_n(D)$ with $Ae_1 = x$ (take $x$ as the first column of $A$) and it follows that
  \[
      f(x)
    = f(A e_1)
    = A f(e_1)
    = A e_1 d
    = x d \,.
  \]
  This shows that $f(x) = xd$ for every $x \in D^n$, which shows that $\Phi$ is surjective.
\end{proof}


\begin{corollary}
  \label{corollary: endomorphism ring of Si}
  In the situation and notation of Corollary~\ref{corollary: artin wedderburn rings are semisimple} we have that $\End_R(S_i) \cong D_i^\op$ for every $i = 1, \dotsc, r$.
\end{corollary}


\begin{proof}
  We have that
  \begin{align*}
            \End_{\Mat_{n_1}(D_1) \times \dotsb \times \Mat_{n_r}(D_r)}(S_i)
    &\cong  0 \times \dotsb \times 0 \times \End_{\Mat_{n_i}(D_i)}(D^{n_i}) \times 0 \times \dotsb \times 0 \\
    &\cong  \End_{\Mat_{n_i}(D_i)}(D^{n_i})
     \cong  D_i^\op
  \end{align*}
  by Corollary~\ref{label: endomorphism ring of boxsum}.
\end{proof}



\begin{notation}
  \label{notation: simple modules over products of matrix rings}
  By abuse of notation we will often denote the simple modules $S_1, \dotsc, S_r$ from Corollary~\ref{corollary: artin wedderburn rings are semisimple} instead by $D_1^{n_1}, \dotsc, D_r^{n_r}$.
  Note that we then have that
  \[
          \End_{\Mat_{n_1}(D_1) \times \dotsb \times \Mat_{n_r}(D_r)}(D_i^{n_i})
    \cong D_i^\op
  \]
  by Corollary~\ref{corollary: endomorphism ring of Si}.
\end{notation}







\subsubsection{The Theorem of Artin--Wedderburn}


\begin{lemma}
  \label{lemma: isotypical components are two sided ideals}
  If $E$ is a simple $R$-module then the isotypical component $R_E$ is a two-sided ideal of $R$.
\end{lemma}


\begin{proof}
  The isotypical component $R_E$ is a submodule of $R$, and therefore a left ideal.
  For every $r \in R$ the map $R \to R$, $x \mapsto xr$ is a homomorphism of $R$-modules and therefore maps $R_E$ into $R_E$ by Lemma~\ref{lemma: functioriality of isotypical components}.
  Therefore $R_E$ is also a right ideal.
\end{proof}


\begin{fluff}
  \label{fluff: intro to artin wedderburn}
  If $R$ is semisimple then by Corollary~\ref{corollary: ss rings have only finitely many simple modules} there exist only finitely many simple $R$-modules $E_1, \dotsc, E_r$ up to isomorphism.
  The \hyperref[theorem: isotypical decomposition]{isotypical decomposition} then reads
  \[
      R
    = R_{E_1} \times \dotsb \times R_{E_r}
  \]
  and each $R_{E_i}$ is a non-trivial two-sided ideal by Lemma~\ref{lemma: simple module of semisimple ring is direct summand} and Lemma~\ref{lemma: isotypical components are two sided ideals}.
  Each $R_{E_i}$ is then itself a ring with respect to the addition and multiplication inherited from $R$ as explained in Proposition~\ref{proposition: factor ideals are again rings}.
  Each $R_{E_i}$ is itself semisimple by Proposition~\ref{proposition: product of semisimple} with $E_i$ being its only simple module up to isomorphism.
  We will now see that the $R_{E_i}$ are already isomorphic to a matrix ring over a skew field:
\end{fluff}


\begin{theorem}[Artin--Wedderburn]
  \label{theorem: artin wedderburn theorem}
  Let $R$ be semisimple.
  \begin{enumerate}
    \item
      If
      \[
              R
        \cong E_1^{\oplus n_1} \oplus \dotsb \oplus E_r^{\oplus n_r}
      \]
      for some $r \geq 0$, pairwise non-isomorphic simple $R$-modules $E_1, \dotsc, E_r$ and suitable $n_1, \dotsc, n_r \geq 1$, then
      \begin{align*}
                R
        &\cong  \End_R(E_1^{\oplus n_1}) \times \dotsb \times \End_R(E_r^{\oplus n_r})  \\
        &\cong  \Mat_{n_1}(D_1) \times \dotsb \times  \Mat_{n_r}(D_r)
      \end{align*}
      as rings with $D_i = \End(E_i)^\op$ for every $i = 1, \dotsc, r$.
      If $R$ is a $k$-algebra then this is an isomorphism of $k$-algebras.
    \item
      This decomposition is unique in the following sense:
      If
      \[
              R
        \cong \Mat_{m_1}(D'_1) \times \dotsb \times \Mat_{m_s}(D'_s)
      \]
      for any $s \geq 0$, $m_1, \dotsc, m_s \geq 1$ and skew fields $D'_1, \dotsc, D'_s$ then $r = s$ and the pairs $(D_1, n_1), \dotsc, (D_r, n_r)$ coincide with the pairs $(D'_1, m_1), \dotsc, (D'_s, m_s)$ up to permutation and isomorphism, i.e.\ there exists a bijection $\pi \colon \{1, \dotsc, r\} \to \{1, \dotsc, s\}$ such that $m_{\pi(i)} = n_i$ and $D'_{\pi(i)} \cong D_i$ for every $i = 1, \dotsc, r$.
  \end{enumerate}
\end{theorem}


\begin{proof}
  \leavevmode
  \begin{enumerate}
    \item
      It follows from Lemma~\ref{lemma: End_R(R) = Rop} and Corollary~\ref{corollary: End is isomorphic to product of matrix rings Schur style} that
      \begin{align*}
                R^\op
         \cong  \End_R(R)
        &\cong  \End_R(E_1^{\oplus n_1} \oplus \dotsb \oplus E_r^{\oplus n_r})  \\
        &\cong  \End_R(E_1^{\oplus n_1}) \times \dotsb \times \End_R(E_r^{\oplus n_r})  \\
        &\cong  \Mat_{n_1}(D_1) \times \dotsb \times \Mat_{n_r}(D_r) \,.
      \end{align*}
      It further follows from Remark~\ref{remark: basic properties of op} and Lemma~\ref{lemma: op of matrix rings} that
      \begin{align*}
                R
        =      (R^\op)^\op
        &\cong  \left( \Mat_{n_1}(D_1) \times \dotsb \times \Mat_{n_r}(D_r) \right)^\op \\
        &=      \Mat_{n_1}(D_1)^\op \times \dotsb \times \Mat_{n_r}(D_r)^\op  \\
        &\cong  \Mat_{n_1}(D_1^\op) \times \dotsb \times \Mat_{n_r}(D_r^\op) \,.
      \end{align*}
    \item
      Let $\varphi \colon R \to \Mat_{m_1}(D'_1) \times \dotsb \times \Mat_{m_s}(D'_s) \defined R'$ be an isomorphism of rings.
      By using Corollary~\ref{corollary: artin wedderburn rings are semisimple} (and the Notation of \ref{notation: simple modules over products of matrix rings}) we have that
      \[
              R'
        \cong {D'_1}^{\oplus m_1} \oplus \dotsb \oplus {D'_s}^{\oplus m_s}
      \]
      as $R'$-modules.
      For every $i = 1, \dotsc, r$ we can pull back the $R'$-module structure of ${D'_i}^{\oplus m_i}$ to an $R$-module structure.
      The ${D'_i}^{\oplus m_i}$ thus become simple pairwise non-isomorpic $R$-modules with
      \[
              R
        \cong {D'_i}^{\oplus m_i} \oplus \dotsb \oplus {D'_i}^{\oplus m_i}
      \]
      as $R$-modules.
      
      By using the uniqueness of multiplicities of simple summands (see Theorem~\ref{theorem: multiplicity well-defined} and Remark~\ref{remark: uniqueness of multiplicities alternative formulation}) it follows that the two decompositions
      \[
              R
        =     E_1^{\oplus n_1} \oplus \dotsb \oplus E_r^{\oplus n_r}
        \cong {D'_1}^{\oplus m_1} \oplus \dotsb \oplus {D'_1}^{\oplus m_1}
      \]
      into simple submodules coincide up to permutation and isomorphism:
      We have that $r = s$ and there exists a bijection $\pi \colon \{1, \dotsc, r\} \to \{1, \dotsc, s\}$ such that $m_{\pi(i)} = n_i$ for every $i = 1, \dotsc, r$ and $D'_{\pi(i)} \cong E_i$ for every $i = 1, \dotsc, r$.
      By again using Corollary~\ref{corollary: artin wedderburn rings are semisimple} we find that
      \[
              D_i
        =     \End_R(E_i)^\op
        \cong \End_R({D'_i}^{\oplus m_i})^\op
        =     \End_{R'}({D'_i}^{\oplus m_i})^\op
        \cong ((D'_i)^\op)^\op
        =     D'_i
      \]
      as rings.
      This finishes the proof.
    \qedhere
  \end{enumerate}
\end{proof}


\begin{remark}
  \leavevmode
  \begin{enumerate}
    \item
      Note that under an isomorphism of rings $R \cong \Mat_{n_1}(D_1) \times \dotsb \times \Mat_{n_r}(D_r)$ the isotypical components $R_{E_1}, \dotsc, R_{E_r}$ correspond (not necessarily in the same order) to the isotypical components $\Mat_{n_1}(D_1), \dotsc, \Mat_{n_r}(D_r)$.
      We have therefore proven our claim from \ref{fluff: intro to artin wedderburn} that the factors $R_{E_i}$ are isomorphic to matrix rings over skew fields.
      Note however that the decomposition
      \[
          R
        = R_{E_1} \times \dotsb \times R_{E_r}
      \]
      is canonical, while the decomposition
      \[
              R
        \cong \Mat_{n_1}(D_1) \times \dotsb \times \Mat_{n_r}(D_r)
      \]
      depends on the choice of decompositions of $R_{E_i}$ into a direct sums of simple submodules.
    \item
      Under the isomorphism of $R$-modules $R \cong E_1^{n_1} \oplus \dotsb \oplus E_r^{n_r}$ the isotypical component $R_{E_i}$ corresponds to the direct summand $E_i^{n_i}$.
      In the above proof of the \hyperref[theorem: artin wedderburn theorem]{theorem of Artin--Wedderburn} we have therefore actually constructed an isomorphism
      \[
                                R^\op
        \xlongrightarrow{\sim}  \End_R(R_{E_1}) \times \dotsb \times \End_R(R_{E_r})
      \]
      which maps $x \in R^\op$ to $(f_1, \dotsc, f_r)$ with $f_i(y) = yx$ for all $i = 1, \dotsc, r$.
      
      This decomposition of $R^\op$ is canonical and does not depend on the further decomposition of $R_{E_i}$ into a direct sum of simple submodules $R_{E_1} \cong E_i^{\oplus n_i}$, contrary to the identification of $\End_R(R_{E_i})$ with $\Mat_{n_i}(\End_R(E_i))$.
  \end{enumerate}
\end{remark}



\begin{remark}
  Corollary~\ref{corollary: artin wedderburn rings are semisimple} and the \hyperref[theorem: artin wedderburn theorem]{theorem of Artin--Wedderburn} together give a classification of semisimple rings up to isomorphism:
  Semisimple rings are precisely the products of matrix rings over skew fields.
\end{remark}







\subsubsection{Applications of Artin--Wedderburn}


\begin{corollary}
  If $R$ is semisimple then $R^\op$ is also semisimple.
\end{corollary}


\begin{proof}
  By the \hyperref[theorem: artin wedderburn theorem]{theorem of Artin--Wedderburn} we have an isomorphism of rings
  \[
          R
    \cong \Mat_{n_1}(D_1) \times \dotsm \times \Mat_{n_r}(D_r)
  \]
  for some $r \geq 0$, $n_1, \dotsc, n_r \geq 1$ and skew fields $D_1, \dotsc, D_r$.
  It then follows that
  \begin{align*}
            R^\op
    &\cong  \left( \Mat_{n_1}(D_1) \times \dotsm \times \Mat_{n_r}(D_r) \right)^\op \\
    &=      \Mat_{n_1}(D_1)^\op \times \dotsm \times \Mat_{n_r}(D_r)^\op \\
    &=      \Mat_{n_1}\left( D_1^\op \right) \times \dotsm \times \Mat_{n_r}\left( D_r^\op \right).
  \end{align*}
  The rings $D_i^\op$ are skew fields because the $D_i$ are skew fields.
  It follows from Corollary~\ref{corollary: artin wedderburn rings are semisimple} that $R^\op$ is semisimple.
\end{proof}


\begin{corollary}
  \label{corollary: semisimple algebra product of matrix algebras}
  Let $A$ be a finite-dimensional semisimple $k$-algebra.
  \begin{enumerate}
    \item
      The $k$-algebra $A$ contains up to isomorphism of $A$-modules only finitely many nonzero minimal left ideals $I_1, \dotsc, I_r \idealleq A$ (which are pairwise non-isomorphic), and we have that
      \[
              A
        \cong \Mat_{n_1}(D_1) \times \dotsm \times \Mat_{n_r}(D_r)
      \]
      where $D_i = \End_A(I_i)^\op$ for every $i = 1, \dotsc, r$.
      
    \item
      If $k$ is algebraically closed then
      \[
              A
        \cong \Mat_{n_1}(k) \times \dotsm \times \Mat_{n_r}(k)
      \]
      as $k$-algebras.
  \end{enumerate}
\end{corollary}


\begin{proof}
  \leavevmode
  \begin{enumerate}
    \item
      A nonzero minimal left ideal $I \idealleq A$ is the same as a simple $A$-submodule of $M$.
      The claim is therefore just a repretition of the \hyperref[theorem: artin wedderburn theorem]{theorem of Artin--Wedderburn}.
    \item
      It follows from Corollary~\ref{corollary: simple modules over fd algebras are fd} that each $I_j$ is finite-dimensional, and it thus further follows from \hyperref[proposition: schurs lemma for modules]{Schur’s Lemma} that $D_i = k$.
    \qedhere
  \end{enumerate}
\end{proof}


\begin{definition}
  An $R$-module $M$ is \emph{faithful} if for every $r_1, r_2 \in R$ with $r_1 \neq r_2$ there exists some $m \in M$ with $r m_1 \neq r m_2$.
\end{definition}


\begin{remark}
  An $R$-module $M$ is faithful if any of the following equivalent conditions is fullfilled:
  \begin{enumerate}
    \item
      The module $M$ is faithful.
    \item
      The corresponding ring homomorphism $R \to \End_\Integer(M)$ is injective.
    \item
      For every $r \in R$ with $r \neq 0$ there exists some $m \in M$ with $rm \neq 0$.
    \item
      The annihilator $\Ann_R(M) = \{r \in R \suchthat rm = 0\}$ is $0$.
  \end{enumerate}
\end{remark}


\begin{example}
  The $R$-module $R$ is injective because we can choose $m = 1$.
\end{example}


\begin{corollary}
  If $R$ is semisimple and $M$ a faithful $R$-module then the isotypical components of $M$ are all nonzero, i.e.\ $M$ contains every simple $R$-module up to isomorphism.
\end{corollary}


\begin{proof}
  By the \hyperref[theorem: artin wedderburn theorem]{theorem of Artin--Wedderburn} we may assume w.l.o.g.\ that
  \[
    R = M_{n_1}(D_1) \times \dotsb \times M_{n_r}(D_r)
  \]
  for $r \geq 0$, $n_1, \dotsc, n_r \geq 1$ and skew field $D_1, \dotsc, D_r$.
  Then $D_1^{n_1}, \dotsc, D_r^{n_r}$ form a complete set of representatives of $\Irr(R)$.
  
  The module $M$ is semisimple because $R$ is semisimpe, so there exists a decomposition into isotypical components $M \cong \bigoplus_{i=1}^s M_{D_i^{n_i}}$.
  If $M_{D_i^{n_i}} = 0$ for some $1 \leq i \leq s$ then every element $A \in M_{n_i}(D_i) \subseteq R$ would act by multiplication with zero on $M$, which would contradicts the faithfulness of $M$.
  The isotypical components $M_{D_i^{n_i}}$ are therefore all nonzero.
\end{proof}





\subsection{Simple Rings \& Weddeburn}


\begin{definition}
  The ring $R$ is simple if it is nonzero and $0, R$ are the only two-sided ideals of $R$, i.e.\ if $R$ contains precisely two two-sided ideals.
\end{definition}


\begin{example}
  \label{example: matrix algebra over skew field is simple}
  If $D$ is a division ring and $n \geq 1$ then $M_n(D)$ is simple.
  This follows from the following lemma:
\end{example}


\begin{lemma}
  For every $n \geq 1$ the map
  \begin{align*}
              \{ \text{two-sided ideals $I \idealleq R$} \}
    &\longto  \{ \text{two-sided ideals $J \idealleq \Mat_n(R)$} \} \,,
    \\
                  I
    &\longmapsto  \Mat_n(I)
  \end{align*}
  is a well-defined bijection.
\end{lemma}


\begin{proof}
  If $I \idealleq R$ is a two-sided ideal then the canonical projection $\pi \colon R \to R/I$ is a ring homomorphism.
  The induced ring homomorphism $\Mat_n(R) \to \Mat_n(R/I)$ has $\Mat_n(I)$ as its kernel, which is therefore a two-sided ideal in $\Mat_n(R)$.
  
  Let on the other hand $J \idealleq \Mat_n(R)$ be a two-sided ideal.
  For all $i, j = 1, \dotsc, n$ let
  \[
      I_{ij}
    = \{
        r \in R
      \suchthat
        \text{there exists a matrix $A \in J$ whose $ij$-th coefficient is $r$}
      \} \,.
  \]
  Then $I_{ij}$ is a two-sided ideal in $R$:
  The projection $\pi_{ij} \colon \Mat_n(R) \to R$ onto the $ij$-th coefficient is a homomorphism of both left and right $R$-modules and the two-sided ideal $J$ is both a left and right $R$-submodule of $\Mat_n(R)$.
  It follows that $\pi_{ij}(J)$ is both a left and right $R$-submodule of $R$, i.e.\ a two-sided ideal.
  
  By multiplying a matrix $A \in J$ with permutation matrices from the left and from the right we can move every coefficient of $A$ to every other position without leaving $J$.
  It follows that the ideal $I \defined I_{ij}$ does not depend on the position $i,j$, and we have have that $J = \Mat_n(I)$ by construction of $I$.
\end{proof}


\begin{warning}
  A simple ring $R$ is not necessarily simple as an $R$-module:
  The ring $\Mat_n(D)$ for a skew field $D$ and $n \geq 2$ is a counterexample.
\end{warning}


% \begin{proposition}
%   \label{proposition: when semisimple is simple}
%   If $R$ is semisimple then the following are equivalent:
%   \begin{enumerate}
%     \item
%       \label{enumerate: is simple}
%       The ring $R$ is simple.
%     \item
%       \label{enumerate: only one simple}
%       The ring $R$ has only one simple $R$-module up to isomorphism
%     \item
%       \label{enumerate: is a matrix ring}
%       We have that $R \cong \Mat_n(D)$ for some $n \in \Natural$ and skew field $D$.
%   \end{enumerate}
% \end{proposition}
% 
% 
% \begin{proof}
%   It follows from the \hyperref[theorem: artin wedderburn theorem]{theorem of Artin--Wedderburn} that
%   \[
%     R \cong \Mat_{n_1}(D_1) \times \dotsb \times \Mat_{n_r}(D_r)
%   \]
%   for $r = |\Irr(R)|$, $n_1, \dotsc, n_r \geq 1$ and skew fields $D_1, \dotsc, D_r$.
%   
%   \begin{description}
%     \item[\ref*{enumerate: is a matrix ring} $\implies$ \ref*{enumerate: is simple}]
%       This follows from Example~\ref{example: simple ring}.
%     \item[\ref*{enumerate: is simple} $\implies$ \ref*{enumerate: only one simple}]
%       The $\Mat_{n_i}(D_i)$ correspond to two-sided ideal in $R$.
%       For $r \geq 2$ it would follow that $R$ contains nonzero proper two-sided ideals.
%     \item[\ref*{enumerate: only one simple} $\implies$ \ref*{enumerate: is a matrix ring}]
%       We have that $r = 1$ and therefore that $R \cong \Mat_{n_1}(D_1)$.
%     \qedhere
%   \end{description}
% \end{proof}
% 
% 
% \begin{theorem}[Wedderburn]
%   \label{theorem: wedderburns theorem}
%   If $R$ is simple then the following are equivalent:
%   \begin{enumerate}
%     \item
%       \label{enumerate: is semisimple}
%       The ring $R$ is semisimple.
%     \item 
%       \label{enumerate: is left artian}
%       The ring $R$ is (left) artian.
%     \item
%       \label{enumerate: has minimal left ideal}
%       The ring $R$ has a minimal nonzero left ideal $I$.
%     \item
%       \label{enumerate: is matrix ring over skew field}
%       We have that $R \cong \Mat_n(D)$ for some $n \in \Natural$ and skew field $D$.
%   \end{enumerate}
%   The skew field $D$ is then unique up to isomorphism.
% \end{theorem}
% 
% 
% \begin{proof}
%   The uniqueness of $D$ follows from the \hyperref[theorem: artin wedderburn theorem]{theorem of Artin--Wedderburn}
%   \begin{description}
%     \item[\ref*{enumerate: is semisimple} $\iff$ \ref*{enumerate: is matrix ring over skew field}]
%       This is part of Proposition~\ref{proposition: when semisimple is simple}.
%     \item[\ref*{enumerate: is semisimple} $\implies$ \ref*{enumerate: is left artian}]
%       This is part of Corollary~\ref{corollary: semisimple rings are notherian artinian}.
%     \item[\ref*{enumerate: is left artian} $\implies$ \ref*{enumerate: has minimal left ideal}]
%       Starting with any nonzero left ideal $I_1 \idealleq R$ there would otherwise exist for every $n \geq 1$ an ideal $I_{n+1} \idealleq R$ with $I_{n+1} \subsetneq I_n$, resulting in a descreasing sequence of ideals which does not stablize.
%     \item[\ref*{enumerate: has minimal left ideal} $\implies$ \ref*{enumerate: is semisimple}]
%       The isotypical component $R_I$ is a two-sided ideal by Lemma~\ref{lemma: isotypical components are two sided ideals} and nonzero by assumption.
%       It follows that $R = R_I$ is semisimple.
%     \qedhere
%   \end{description}
% \end{proof}


\begin{warning}
  Not every simple ring is semisimple, despite its name:
\end{warning}


\begin{example}
  \label{example: simple but not semisimple}
  Let $\ringchar(k) = 0$.
  The (first) Weyl algebra $\weyl = \weyl_1$ from subsection~\ref{subsection: first weyl algebra} is simple but not semisimple:
  
  To show that $\weyl$ is not semisimple it sufficies by Corollary~\ref{corollary: semisimple rings are notherian artinian} to observe that $\weyl$ is not artinian:
  We have seen that $\weyl$ has a basis given by $X^n \del^m$ with $n, m \geq 0$.
  It then follows for every $i \geq 0$ that
  \[
      \weyl \del^i
    = \gen{ X^n \del^{m+i} \suchthat n, m \geq 0 }_k
    = \gen{ X^n \del^m \suchthat n \geq 0, m \geq i }_k  \,,
  \]
  which shows that
  \[
                \weyl
    =           \weyl \del^0
    \idealgneq  \weyl \del^1
    \idealgneq  \weyl \del^2
    \idealgneq  \dotsb
  \]
  is a strictly decreasing sequence of left ideals of $\weyl$.
  
  To show that $\weyl$ is simple let $I \idealleq \weyl$ be a two-sided ideal and let $f \in I$ be non-zero.
  We can then write $f$ uniquely as
  \[
      f
    = \sum_{m \geq 0} p_m \del^m
  \]
  with $p_m \in k[X] \subseteq \weyl$ for all $m \geq 0$.
  It follows from $\del X = X \del + 1$ by induction that 
  \[
      \del^n X
    = X \del^n + n \del^{n-1}
  \]
  for all $n \geq 0$.
  It follows that
  \begin{align*}
        f X - X f
    &=  \sum_{m \geq 0} \left( p_m \del^m X - X p_m \del^m \right)  \\
    &=  \sum_{m \geq 0} \left( p_m X \del^m + m p_m \del^{m-1} - p_m X \del^m \right)
     =  \sum_{m \geq 0} m p_m \del^{m-1} \,.
  \end{align*}
  For the maximal index $m \geq 0$ with $p_m \neq 0$ it follows that $m! p_m \in I$ and therefore that
  \[
    p_m \in I \,.
  \]
  This shows that $I$ already contains a non-zero polynomial from $k[X] \subseteq \weyl$.
  We have seen in subsection~\ref{subsection: first weyl algebra} (namely Equation~\eqref{equation: motivation skew polynomial ring}) that the formula $\del X = X \del + 1$ generalizes to
  \[
    \del p = p \del + p'
  \]
  for all $p \in k[X] \subseteq \weyl$, where $p'$ denotes the (formal) derivative of $p$.
  We thus have that
  \[
      \del p - p \del
    = p'
  \]
  for all $p \in k[X]$.
  It follows for $d \defined \deg_X(p_m)$ that $p^{(d)} \in I$ is a non-zero constant polynomial (because $\ringchar(k) \neq 0$).
  This shows that $I$ contains a unit, which shows that $I = \weyl$.
\end{example}


\begin{remark}
  We have given in Remark~\ref{remark: skew polynomial rings} a short overview how the Weyl algebra $\weyl$ can be understood as a skew polyonmial ring over the ground ring $k[X]$ with respect to the $k$-derivation $\del \colon k[X] \to k[X]$.
  
  It can be shown more generally that for a field of characteristic $\ringchar(k) = 0$ and a $k$-algebra $A$ a skew polynomial ring $A[y;\delta]$ (with respect to a $k$-derivation $\delta \colon A \to A$) is simple if and only if $A$ is \emph{$\delta$-simple} and $\delta$ is not an \emph{inner derivation};
  proofs of this can be found in \cite[Theorem~3.15]{Lam1991First} and \cite[Proposition~2.1]{NoncommutativeNoetherian} (where definitions of the above terms can also be found).
  If $\ringchar(k) = 0$ then it also holds for every simple $k$-algebra $A$ and every non-inner derivation $\delta \colon A \to A$ that the skew polynomial ring $A[y;\delta]$ is a nonartinian simple ring; a proof of this can be found in \cite[Corollary~3.16]{Lam1991First}.
  
  This gives a recipe for constructing simple rings which are not semisimple.
\end{remark}


\begin{example}
  \label{example: simple but not semisimple endorphism ring}
  The following example is taken from \cite[3.14]{Lam1991First}:
  Let $V$ be a countable infinite-dimensional $k$-vector space for some (skew) field $k$.
  Then
  \[
              I
    \defined  \{
                f \in \End(V)
              \suchthat
                \text{$f$ has finite rank}
              \}
  \]
  is a two-sided ideal because we have for all $g \in \End(V)$ and $f, f_1, f_2 \in I$ that
  \begin{gather*}
            \rank(f \circ g),
            \rank(g \circ f)
      \leq  \rank(f)
  \shortintertext{and}
            \rank(f_1 + f_2)
      \leq  \rank(f_1) + \rank(f_2) \,.
  \end{gather*}
  The two-sided ideal $I$ is already maxmimal:
  
  Suppose that $f \in \End(V)$ has infinite rank.
  Let $C_1$ be a direct complement of $\im(f)$ and let $C_2$ be a direct complement of $\ker(f)$.
  Then $\im(f)$ is countable infinite-dimensional, so there exists an endomorphism $g_1 \colon V \to V$ such that the restriction $\restrict{g_1}{\im(f)} \to V$ is an isomorphism.
  The endomorphism $f$ restricts to an isomorphism $C_2 \to \im(f)$, so $C_2$ is also countable infinite-dimensional.
  It follows that there exists an endomorphism $g_2 \colon V \to V$ which restricts to an isomorphism $V \to C_2$.
  The composition $g_1 \circ f \circ g_2 \colon V \to V$ is then an isomorphism.
  This shows that the two-sided ideal generated by $f$ is already $\End(V)$ itself.
  
  It follows from the maximality of $I$ that $\End(V)/I$ is simple.
  We now show that $\End(V)/I$ is not noetherian, from which it follows by Corollary~\ref{corollary: semisimple rings are notherian artinian} that $\End(V)/I$ is not semisimple.
  
  We choose a basis $(b_{i,j})_{i,j \geq 0}$ of $V$, and for every $n \geq 0$ we consider the left ideal of $\End(V)$ given by
  \[
              J_n
    \defined  \{
                f \in \End(V)
              \suchthat
                \text{$f(b_{ij}) = 0$ for all $i \geq n$, $j \geq 0$}
              \} \,.
  \]
  We then have that $J_n \ideallneq J_{n+1}$ for all $n \geq 0$. 
  
  \begin{claim}
    For all $n \geq 0$ we have that $I + J_n \ideallneq I + J_{n+1}$.
  \end{claim}
  
  \begin{proof}
    It sufficies to show that $J_{n+1} \idealnleq I + J_n$:
    Consider an element $f \in J_{n+1}$ with
    \[
      f(b_{n+1, j}) = b_{n+1,j}
    \]
    for all $j \geq 0$.
    If $f \in I + J_n$ then there would exist $g \in I$, $f' \in J_n$ with $f = g + f'$.
    It then follows that
    \[
        b_{n+1,j}
      = f(b_{n+1,j})
      = g(b_{n+1,j}) + f'(b_{n+1,j})
      = g(b_{n+1,j})
    \]
    for all $j \geq 0$, and therefore that
    \[
        g(b_{n+1,j})
      = -b_{n+1,j}
    \]
    for all $j \geq 0$.
    But this contradicts $g$ having finite rank.
  \end{proof}

  It follows that
  \[
                I
    =           I + J_0
    \ideallneq  I + J_1
    \ideallneq  I + J_2
    \ideallneq  \dotsb
  \]
  is a strictly increasing sequence of ideals in $\End(V)$ and it follows that
  \[
                0
    =           (I + J_0)/I
    \ideallneq  (I + J_1)/I
    \ideallneq  (I + J_2)/I
    \ideallneq  \dotsb
  \]
  is a strictly increasing sequence of ideals in $\End(V)/I$.
  This shows that $\End(V)/I$ is not noetherian.
\end{example}


\begin{lemma}
  \label{lemma: isotypical components are two sided ideals}
  If $E$ is a simple $R$-module then the isotypical component $R_E$ is a two-sided ideal of $R$.
\end{lemma}


\begin{proof}
  The isotypical component $R_E$ is a submodule of $R$ and therefore a left ideal.
  For every $r \in R$ the map $R \to R$, $x \mapsto xr$ is a homomorphism of $R$-modules and therefore maps $R_E$ into $R_E$ by Lemma~\ref{lemma: functoriality of isotypical components}.
  This shows that $R_E$ is also a right ideal.
\end{proof}


\begin{theorem}[Wedderburn]
  \label{theorem: wedderburns theorem}
  The following conditions are equivalent:
  \begin{enumerate}
    \item
      \label{enumerate: simple and artinian}
      The ring $R$ is simple and (left) artinian.
    \item
      \label{enumerate: simple and minimal left ideal}
      The ring $R$ is simple and contains a minimal nonzero left ideal $I \idealleq R$.
    \item
      \label{enumerate: simple and semisimple}
      The ring $R$ is both simple and semisimple.
    \item
      \label{enumerate: semisimple with unique simple}
      The ring $R$ is semisimple and has only one simple module up to isomorphism.
    \item
      \label{enumerate: matrix algebra over skew field}
      We have that $R \cong \Mat_n(D)$ for some $n \geq 1$ and skew field $D$.
  \end{enumerate}
  The skew field $D$ is then up to isomorphism uniquely determined as $D \cong \End_R(I)^\op$, with $I$ as above being the unique simple $R$-module up to isomorphism.
  The number $n$ is uniquely determined as the multiplicity of $I$ in $R$.
\end{theorem}


\begin{proof}
  \label{proof: wedderburns theorem first proof}
  \leavevmode
  \begin{description}
    \item[\ref*{enumerate: simple and artinian} $\implies$ \ref*{enumerate: simple and minimal left ideal}]
      The set of nonzero left ideals of $R$ is non-empty because $R \neq 0$ and thus contains a minimal element because $R$ is artinian.
    \item[\ref*{enumerate: simple and minimal left ideal} $\implies$ \ref*{enumerate: simple and semisimple}]
      The ideal $I$ is a simple $R$-module and the $I$-isotypical component $R_I$ is a nonzero two-sided ideal by Lemma~\ref{lemma: isotypical components are two sided ideals}.
      It follows from $R$ being simple that $R_I = R$, and therefore that $R$ is semisimple.
    \item[\ref*{enumerate: simple and semisimple} $\implies$ \ref*{enumerate: semisimple with unique simple}]
      Because $R \neq 0$ is a sum of simple $R$-modules it follows that there exists a simple $R$-module $E$.
      The $E$-isotypical component $R_E$ is a two-sided ideal by Lemma~\ref{lemma: isotypical components are two sided ideals} which is nonzero by Lemma~\ref{lemma: simple module of semisimple ring is direct summand}.
      It follows that $R_E = R$, and therefore from Lemma~\ref{lemma: simple module of semisimple ring is direct summand} that $E$ is the unique simple $R$-module up to isomorphism.
    \item[\ref*{enumerate: semisimple with unique simple} $\implies$ \ref*{enumerate: matrix algebra over skew field}]
      This follows from the \hyperref[theorem: artin wedderburn theorem]{theorem of Artin--Wedderburn}.
    \item[\ref*{enumerate: matrix algebra over skew field} $\implies$ \ref*{enumerate: simple and semisimple}]
      That $\Mat_n(D)$ is simple follows from Example~\ref*{example: matrix algebra over skew field is simple}, and that $R$ is semisimple follows from Example~\ref{example: semisimple rings}.
    \item[\ref*{enumerate: simple and semisimple} $\implies$ \ref*{enumerate: simple and artinian}]
      This follows from Corollary~\ref{corollary: semisimple rings are notherian artinian}.
  \end{description}
  The minimal ideal $I$ is a simple submodule of $R$, and thus the unique simple $R$-module up to isomorphism .
  Because $D^n$ is a simple $\Mat_n(D)$-module, so it follows that
  \[
          D^\op
    \cong \End_{\Mat_n(D)}(D^n)
    \cong \End_R(I) \,,
  \]
  and therefore that $D \cong \End_R(I)^\op$.
  The multiplicity of $I$ in $R$ is the same as the multiplicity of $D^n$ in $\Mat_n(D)$, which is $n$.
\end{proof}





\subsection{Alternative Approach}


\begin{fluff}
  We will now show another approach to the theorems of \hyperref[theorem: artin wedderburn theorem]{Artin--Wedderburn} and \hyperref[theorem: wedderburns theorem]{Wedderburn} which illuminates the role that simple ring play in the theory.
  
  We have seen in Lemma~\ref{lemma: isotypical components are two sided ideals} that the isotypical components of a ring $R$ are two-sided ideals.
  We will begin by strengthening this result:
\end{fluff}


\begin{lemma}[{\cite[Lemma~1.14]{FarbDennis1993}}]
  \label{lemma: decomposition of module into End tensor R modules}
  Let $M$ be an $R$-module.
  \begin{enumerate}
    \item
      For every simple $R$-module $E$ the $E$-isotypical component $M_E$ is $\End_R(M)$-in\-vari\-ant, i.e.\ we have that $f(M_E) \subseteq M_E$ for every $f \in \End_R(M)$.
    \item
      If $M$ is semisimple and $N \moduleleq M$ is an $R$-submodule which is $\End_R(M)$-invariant then $N$ is a sum of isotypical components of $M$, i.e.\ there exists some subset $\mc{S} \subseteq \Irr(R)$ with $N = \bigoplus_{[E] \in \mc{S}} M_E$.
  \end{enumerate}
\end{lemma}


\begin{proof}
  \leavevmode
  \begin{enumerate}
    \item
      This follows from Lemma~\ref{lemma: functoriality of isotypical components}.
    \item
      We have that $M = \bigoplus_{[E] \in \Irr(R)} M_E$ and $N = \bigoplus_{[E] \in \Irr(R)} N_E$, so we need to show that for every $E \in \Irr(R)$ with $N_E \neq 0$ we already have that $N_E = M_E$.
      
      Note that $N_E = N \cap M_E$ is $\End_{R}(M)$-invariant and therefore also $\End_{R}(M_E)$-invariant because every $R$-module endomorphism of $M_E$ extends to an endomorphism of $M$ by Corollary~\ref{corollary: endomorphism ring of semisimple module}. 
      It therefore sufficies to consider the case that $M = M_E$ for some $E \in \Irr(R)$, i.e.\ that $M$ is $E$-isotypical.
      
      Then $N$ is also $E$-isotypical and it follows from $N \neq 0$ that there exists a submodule $L \moduleleq N$ with $L \cong E$.
      If $L' \moduleleq M$ is any submodule with $L' \cong E$ then $L \cong L'$ and every isomorphism $f \colon L \to L'$ extends to an $R$-module endomorphism $g \colon M \to M$:
      We may choose direct complements $C, C'$ of $L, L'$ because $M$ is semisimple and define $g$ by
      \[
                g
        \colon  M
        =       L \oplus C
        \xlongrightarrow{ \begin{bmatrix} f & 0 \\ 0 & 0 \end{bmatrix} }
                L' \oplus C'
        =       M \,,
      \]
      i.e.\ $g$ is given by the composition
      \[
                            g
        \colon              M
        =                   L \oplus C
        \projection         L
        \xlongrightarrow{f} L'
        \inclusion          L' \oplus C'
        =                   M \,.
      \]
      It follows that
      \[
                    L'
        =           f(L)
        =           g(L)
        \moduleleq  N
      \]
      because $N$ is $\End_R(M)$-invariant.
      This shows that
      \[
                    M
        =           M_E
        =           \sum_{L' \moduleleq M, L' \cong E} L'
        \moduleleq  N \,,
      \]
      which shows that $M = N$.
    \qedhere
  \end{enumerate}
\end{proof}


\begin{corollary}
  \label{corollary: isotypical components as two sided ideals}
  \leavevmode
  \begin{enumerate}
    \item
      For every simple $R$-module $E$ the $E$-isotypical component $R_E$ is a two-sided ideal of $R$.
    \item
      If $R$ is semisimple and $E_1, \dotsc, E_n$ is a set of representatives for the isomorphism classes of simple $R$-modules (this set is finite by Corollary~\ref{corollary: ss rings have only finitely many simple modules}) then the $E_i$-isotypical components $R_{E_i}$ are minimal two-sided ideals of $R$, and every two-sided ideal of $R$ is a sum of isotypical components.
  \end{enumerate}
\end{corollary}


\begin{proof}
    The two-sided ideal of $R$ are precisely those left ideals which are also invariant under right multiplication with elements of $R$.
    It follows from the isomorphism $R^\op \cong \End_R(R)$ from Lemma~\ref{lemma: End_R(R) = Rop} that for a left ideal, invariance under right multiplication is the same as $\End_R(R)$-invariance.
    The two-sided ideals of $R$ are therefore precisely those left ideals which are $\End_R(R)$-invariant.
    
    With this observation the claims follow from Lemma~\ref{lemma: decomposition of module into End tensor R modules}.
\end{proof}


\begin{corollary}
  \label{corollary: semisimple ring has only finitely many components}
  If $R$ is semisimple then $R$ contains only finite many two-sided ideals, namely $2^n$ many where $n = |{\Irr(R)}|$.
\end{corollary}


\begin{remark}
  Corollary~\ref{corollary: semisimple ring has only finitely many components} also follows from the \hyperref[theorem: artin wedderburn theorem]{theorem of Artin--Wedderburn}:
  We may assume that the semisimple ring $R$ is given by $R = \Mat_{n_1}(D_1) \times \dotsb \times \Mat_{n_r}(D_r)$ with $r \geq 0$, $n_1, \dotsc, n_r \geq 1$ and skew fields $D_1, \dotsc, D_r$.
  Then every two-sided ideal $I \idealleq R$ is of the form $I = I_1 \times \dotsb \times I_r$ for unique two-sided ideals $I_j \idealleq \Mat_{n_j}(D_j)$ by Remark~\ref{remark: right and two-sided ideals in products of rings}.
  It follows for every $j = 1, \dotsc, r$ that either $I_j = 0$ or $I_j = \Mat_{n_j}(D_j)$ because the matrix rings $\Mat_{n_j}(D_j)$ are simple.
  Thus it follows that $R$ contains precisely $2^r$ two-sided ideals, with $r = |{\Irr(R)}|$.
\end{remark}


\begin{fluff}
  \label{fluff: intro to artin wedderburn}
  If $R$ is semisimple then by Corollary~\ref{corollary: ss rings have only finitely many simple modules} there exist only finitely many simple $R$-modules $E_1, \dotsc, E_r$ up to isomorphism.
  The \hyperref[theorem: isotypical decomposition]{isotypical decomposition} then reads
  \[
      R
    = R_{E_1} \times \dotsb \times R_{E_r}
  \]
  and each $R_{E_i}$ is a non-trivial two-sided ideal by Lemma~\ref{lemma: simple module of semisimple ring is direct summand} and Corollary~\ref{corollary: isotypical components as two sided ideals}.
  
  Each $R_{E_i}$ is then itself a ring with respect to the addition and multiplication inherited from $R$ as explained in Proposition~\ref{proposition: factor ideals are again rings}.
  Each $R_{E_i}$ is itself semisimple by Proposition~\ref{proposition: product of semisimple} with $E_i$ being its only simple module up to isomorphism.
  The rings $R_{E_i}$ are simple, as can be seen in two ways:
  \begin{itemize}
    \item
      Every $R_{E_i}$ is a minimal two-sided ideal of $R$, so it cannot contain any nonzero proper ideals.
    \item
      Because $R_{E_i}$ is semisimple and has precisely one isomorphism class of simple modules it follows from Corollary~\ref{corollary: isotypical components as two sided ideals} that $R_{E_i}$ is the unique nonzero two-sided ideal of $R_{E_i}$.
  \end{itemize}
  
  We have thus shown that every semisimple ring has a canonical decomposition into a direct product of rings, each of which is simple and semisimple with only one isomorphism class of simple modules.
  We would now like to rewoke \hyperref[theorem: wedderburns theorem]{Wedderburn’s theorem} to conclude that each factor $R_{E_i}$ is already of the form $R_{E_i} \cong \Mat_{n_i}(D_i)$ for some $n_i \geq 1$ and skew field $D_i$.
  
  But in the proof of \hyperref[theorem: wedderburns theorem]{Wedderburn’s theorem} we actually used the \hyperref[theorem: artin wedderburn theorem]{theorem of Artin--Wedderburn}, which are trying to avoid.
  However, by taking a careful look at the given proof of \hyperref[theorem: wedderburns theorem]{Wedderburn’s theorem} we see that we only used the \hyperref[theorem: artin wedderburn theorem]{theorem of Artin--Wedderburn} for one of the implications.
  We will therefore now give another proof, which does not rely on the \hyperref[theorem: artin wedderburn theorem]{theorem of Artin--Wedderburn}.
  The main tool is the following observation due to \cite{Rieffel}.
%   and can also be found in \cite[Chapter~XVII, Theorem~5.4]{LangAlgebra2005} and \cite[Theorem~3.11]{Lam1991First}.
\end{fluff}


\begin{lemma}[Rieffel]
  \label{lemma: simple ring isomorphic to endomorphism ring of ideal}
  Let $I \idealleq R$ be a left ideal.
  Then $I$ is a left $D$-module for $D \defined \End_R(I)$ via
  \[
      \varphi \cdot x
    = \varphi(x)
  \]
  for all $\varphi \in D$, $x \in I$, and the map
  \[
            \Phi    
    \colon  R
    \to     \End_D(I)\,,
    \quad   r
    \mapsto (x \mapsto rx)
  \]
  is a well-defined ring homomorphism.
  If $R$ is simple and $I$ is nonzero then $\Phi$ is an isomorphism.
\end{lemma}


\begin{proof}
  That $I$ is a left $D$-module follows by direct calculation.
  The $R$-module structure of $I$ corresponds to a ring homomorphism $\Phi' \colon R \to \End_\Integer(I)$, $r \mapsto (x \mapsto rx)$, which restrict to the desired ring homomorphism $\Phi$ because the actions of $R$ and $D$ on $I$ commute (by definition of $D$).
  
  It follows from $I \neq 0$ that $D \neq 0$ and therefore that $\Phi \neq 0$.
  The kernel $\ker(\Phi)$ is a therefore a proper two-sided ideal of $R$, and must be trivial by the simplicity of $R$.
  This shows that $\Phi$ is injective.
  
  The key observation behind the surjectivity of $\Phi$ is that $\Phi(I)$ is a left-ideal in $\End_D(I)$:
  Let $f \in \End_D(I)$ and $x \in I$.
  For every $y \in I$ the map $\rho_y \colon I \to I$, $x' \mapsto x'y$ is a homomorphism of $R$-modules, i.e.\ an element of $D$, and thus commutes with $f$.
  It follows for every $y \in I$ that
  \[
      (f \Phi(x))(y)
    = f(\Phi(x)(y))
    = f(xy)
    = f(\rho_y(x))
    = \rho_y(f(x))
    = f(x)y
    = \Phi(f(x))(y) \,,
  \]
  showing that $f \Phi(x) = \Phi(f(x)) \in \Phi(I)$.
  
  It follows that $\Phi(R)$ is a left ideal in $\End_D(I)$:
  It follows from $IR$ being a nonzero two-sided ideal of $R$ that $R = IR$ by the simplicity of $R$.
  It follows that $\Phi(R) = \Phi(I)\Phi(R)$, and because $\Phi(I)$ is a left ideal in $\End_D(R)$ it further follows that
  \[
              \End_D(I) \Phi(R)
    =         \End_D(I) \Phi(I) \Phi(R)
    \subseteq \Phi(I) \Phi(R)
    =         \Phi(R) \,.
  \]
  Because the left ideal $\Phi(R)$ contains the identity $1_{\End_D(I)} = \Phi(1)$ it follows that $\Phi(R) = \End_D(I)$, showing the surjecitvity of $R$.
\end{proof}


\begin{proof}[Alternative proof of {\hyperref[theorem: wedderburns theorem]{Wedderburn’s theorem}}]
  \leavevmode
  \begin{description}
    \item[\ref*{enumerate: simple and artinian} $\implies$ \ref*{enumerate: simple and minimal left ideal}]
      As in the \hyperref[proof: wedderburns theorem first proof]{first proof}.
    \item[\ref*{enumerate: simple and minimal left ideal} $\implies$ \ref*{enumerate: matrix algebra over skew field}]
      It follows from Lemma~\ref{lemma: simple ring isomorphic to endomorphism ring of ideal} that $R \cong \End_D(I)$ for $D \defined \End_R(I)$.
      It follows from $I$ being simple as an $R$-module that $D$ is a skew field.
      If $I$ were not finite-dimensional as a $D$-vector space then it would follows as in Example~\ref{example: simple but not semisimple endorphism ring} that
      \[
        \{
          f \in \End_D(I)
        \suchthat
          \text{$f$ has finite rank}
        \}
      \]
      is a nonzero proper two-ideal in $\End_D(I)$, which would contradict $R$ being simple.
      We thus find that $I$ is finite-dimensional as a $D$-vector space of dimension $n \defined \dim_D(I)$.
      By linear algebra we find that $I \cong D^n$ as $D$-vector spaces;
      contrary to our usual convention we will regard $D^n$ as the space of \emph{row} vectors of width $n$.
      It then follows from linear algebra that every $D$-endomorphism $f \colon D^n \to D^n$ is given by right multiplication with a matrix $A \in \Mat_n(D)$, resulting in an isomorphism of rings $\End_D(D^n) \cong \Mat_n(D)^\op$.
      It follows that
      \[
              R
        \cong \End_D(I)
        \cong \End_D(D^n)
        \cong \Mat_n(D)^\op
        \cong \Mat_n(D^\op)
      \]
      with $D^\op$ being a skew field.
    \item[\ref*{enumerate: matrix algebra over skew field} $\implies$ \ref*{enumerate: simple and semisimple}]
      We know that $\Mat_n(D)$ is both simple and semisimple.
    \item[\ref*{enumerate: simple and semisimple} $\implies$ \ref*{enumerate: semisimple with unique simple}]
      As in the \hyperref[proof: wedderburns theorem first proof]{first proof}.
    \item[\ref*{enumerate: semisimple with unique simple} $\implies$ \ref*{enumerate: simple and artinian}]
      Every semisimple ring is artinian by Corollary~\ref{corollary: semisimple rings are notherian artinian}, and if $R$ has only a single isomorphism class of simple modules then it follows from Corollary~\ref{corollary: isotypical components as two sided ideals} that $R$ is simple.
  \end{description}
  The uniqueness of $D$ up to isomorphism and the uniqueness of $n$ can be shown as in the \hyperref[proof: wedderburns theorem first proof]{first proof}.
\end{proof}


\begin{fluff}
  We now have an alternative proof for the existence of an Artin--Wedderburn decomposition of a semisimple ring $R$:
  We first decompose $R$ as
  \[
      R
    = R_{E_1} \times \dotsb \times R_{E_n}
  \]
  where $E_1, \dotsc, E_n$ is a set of representatives for the isomorphism classes of simple $R$-modules.
  This is a decomposition into two-sided ideals, and therefore a decomposition of $R$ into a direct products of rings $R_{E_1}, \dotsc, R_{E_n}$.
  Then each factor $R_{E_i}$ is a ring which is both simple and semisimple.
  By \hyperref[theorem: wedderburns theorem]{Wedderburn’s theorem} each factor $R_{E_i}$ isomorphic to a matrix ring $R_{E_i} \cong \Mat_{n_i}(D_i)$ where $n_i$ is the multiplicity of $E_i$ in $R_{E_i}$, which is the same as the multiplicity of $E_i$ in $R$, and $D_i = \End_{R_{E_i}}(E_i)^\op = \End_R(E_i)^\op$ is a skew field.
  
  We can also give an alternative proof of the uniqueness of the Artin--Wedderburn decomposition (up to isomorphism):
\end{fluff}


\begin{lemma}
  \label{lemma: uniqueness of decompositon into simple rings}
  Let $R = I_1 \oplus \dotsb \oplus I_n = J_1 \oplus \dotsb \oplus J_m$ be two decompositions into minimal two-sided ideals.
  Then both decompositions coincide up to permutation of the summands.
\end{lemma}


\begin{proof}[First proof ({\cite[Lemma~3.8]{Lam1991First}})]
  Each $I_i$ inherits the structure of a ring from $R$ by Proposition~\ref{proposition: factor ideals are again rings}, and $R$ is the internal direct product of the rings $I_1, \dotsc, I_n$ in the sense of Definition~\ref{definition: internal direct product of rings}.
  It follows from Remark~\ref{remark: right and two-sided ideals in products of rings} that every two-sided ideal $K \idealleq R$ is of the form
  \[
    K = K_1 \oplus \dotsb \oplus K_n
  \]
  for unique two-sided ideals $K_i \idealleq I_i$;
  the component $K_i$ can equivalently be described as $K_i = K \cap I_i$.
  We thus find for all $j = 1, \dotsc, m$ that
  \[
    J_j = (J_j \cap I_1) \oplus \dotsb \oplus (J_j \cap I_n) \,.
  \]
  The intersections $J_j \cap I_i$ are again two-sided ideals.
  It therefore follows from the minimality of $J_j$ that there exists a unique index $1 \leq \tau(j) \leq n$ with $J_j = J_j \cap I_{\tau(j)}$, which can be rephrased as $J_j \subseteq I_{\tau(j)}$.
  
  We find in the same way that there exists for every $i = 1, \dotsc, n$ some $1 \leq \sigma(i) \leq m$ with $I_i \subseteq J_{\sigma(i)}$.
  It follows for every $i = 1, \dotsc, n$ that
  \[
              I_i
    \subseteq J_{\sigma(i)}
    \subseteq I_{\tau(\sigma(i))} \,,
  \]
  from which it follows that $\tau(\sigma(i)) = i$.
  It then also follows that $I_i = J_{\sigma(i)}$ for every $i = 1, \dotsc, n$.
  That $\sigma(\tau(j)) = j$ and $J_j = I_{\tau(j)}$ for all $j = 1, \dotsc, m$ follows in the same way.
  
  This shows that the mappings $\sigma, \tau$ are mutually inverse bijections, which shows that $n = m$.
  We have also shown that both $I_1, \dotsc, I_n$ and $J_1, \dotsc, J_m = J_n$ agree up to permutation (namely $\sigma$, resp.\ $\tau$).
\end{proof}


\begin{proof}[Second proof ({\cite[Theorem~1.13]{FarbDennis1993}})]
  We have for all $j = 1, \dotsc, m$ that
  \[
      J_j
    = R J_j
    = \bigoplus_{i=1}^n I_i J_j    
  \]
  with $I_i J_j$ being two-sided ideals which are contained in $J_j$.
  It follows from the minimality of $J_j$ that there exists a unique index $\tau(j) \in \{1, \dotsc, n\}$  with $J_j = I_{\tau(i)} J_j$, and thus $J_j \subseteq I_{\tau(i)}$.
  We can now proceed as in the first proof.
\end{proof}


\begin{fluff}
  We can now prove the uniqueness part of the \hyperref[theorem: artin wedderburn theorem]{theorem of Artin--Wedderburn}:
  Suppose that $R$ is semisimple with $R \cong \Mat_{n_1}(D_1) \times \dotsb \times \Mat_{n_r}(D_r)$ for some $r \geq 0$, $n_1, \dotsc, n_r \geq 1$ and skew fields $D_i$.
  Then the factors $\Mat_{n_i}(D_i)$ corresponding precisely to the isotypical components of $R$, as can be seen in two ways:
  \begin{itemize}
    \item
      The factors $\Mat_{n_i}(D_i)$ are the isotyipical components of $\Mat_{n_1}(D_1) \times \dotsb \times \Mat_{n_r}(D_r)$, and thus are mapped by every isomorphism $\Mat_{n_1}(D_1) \times \dotsb \times \Mat_{n_r}(D_r) \to R$ bijectively onto the isotypical components of $R$.
    \item
      The product structure of $\Mat_{n_1}(D_1) \times \dotsb \times \Mat_{n_r}(D_r)$ correspondings to a decomposition $R = I_1 \oplus \dotsb \oplus I_r$ into two-sided ideals.
      The ideals $I_i$ are minimal nonzero two-sided ideals because the rings $\Mat_{n_i}(D_i)$ are simple.
      It follows from Lemma~\ref{lemma: uniqueness of decompositon into simple rings} that the ideals $I_1, \dotsc, I_r$ coincide the with isotypical components of $R$ up to permutation.
  \end{itemize}
  By reordering the factors $\Mat_{n_i}(D_i)$ we may therefore assume that $R_{E_i} \cong \Mat_{n_i}(D_i)$ for all $i = 1, \dotsc, r$, where $E_1, \dotsc, E_r$ is a set of representatives for the isomorphism classes of simple $R$-modules.
  It now follows from \hyperref[theorem: wedderburns theorem]{Wedderburn’s theorem} that the number $n_i$ is uniquely determined as the multiplicity of $E_i$ in $R$, and the skew field $D_i$ is uniquely determined up to isomorphism as $D_i \cong \End_R(E_i)^\op$.
\end{fluff}


\begin{fluff}
  Altogether we have shown that every semisimple ring $R$ has a unique decomposition into a product of simple rings, and that each factor is then isomorphic to matrix ring $\Mat_{n_i}(D_i)$ over a skew field $D_i$ by \hyperref[theorem: wedderburns theorem]{Wedderburn’s theorem}, with $n_i$ being uniquely determined and $D_i$ being unique up to isomorphism.
\end{fluff}


% \begin{remark}
%   \leavevmode
%   \begin{enumerate}
%     \item
%       Note that under an isomorphism of rings $R \cong \Mat_{n_1}(D_1) \times \dotsb \times \Mat_{n_r}(D_r)$ the isotypical components $R_{E_1}, \dotsc, R_{E_r}$ correspond (not necessarily in the same order) to the isotypical components $\Mat_{n_1}(D_1), \dotsc, \Mat_{n_r}(D_r)$.
%       We have therefore proven our claim from \ref{fluff: intro to artin wedderburn} that the factors $R_{E_i}$ are isomorphic to matrix rings over skew fields.
%       Note however that the decomposition
%       \[
%           R
%         = R_{E_1} \times \dotsb \times R_{E_r}
%       \]
%       is canonical, while the decomposition
%       \[
%               R
%         \cong \Mat_{n_1}(D_1) \times \dotsb \times \Mat_{n_r}(D_r)
%       \]
%       depends on the choice of decompositions of $R_{E_i}$ into a direct sums of simple submodules.
% %     \item
% %       Under the isomorphism of $R$-modules $R \cong E_1^{n_1} \oplus \dotsb \oplus E_r^{n_r}$ the isotypical component $R_{E_i}$ corresponds to the direct summand $E_i^{n_i}$.
% %       In the above proof of the \hyperref[theorem: artin wedderburn theorem]{theorem of Artin--Wedderburn} we have therefore actually constructed an isomorphism
% %       \[
% %                                 R^\op
% %         \xlongrightarrow{\sim}  \End_R(R_{E_1}) \times \dotsb \times \End_R(R_{E_r})
% %       \]
% %       which maps $x \in R^\op$ to $(f_1, \dotsc, f_r)$ with $f_i(y) = yx$ for all $i = 1, \dotsc, r$.
% %       
% %       This decomposition of $R^\op$ is canonical and does not depend on the further decomposition of $R_{E_i}$ into a direct sum of simple submodules $R_{E_1} \cong E_i^{\oplus n_i}$, contrary to the identification of $\End_R(R_{E_i})$ with $\Mat_{n_i}(\End_R(E_i))$.
%   \end{enumerate}
% \end{remark}









