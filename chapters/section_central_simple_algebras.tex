\section{Central Simple \texorpdfstring{$k$}{k}-algebras}


\begin{conventions}
  In this subsecton $k$ denotes a field and we abbreviate $\otimes \defined \otimes_k$.
\end{conventions}


\begin{fluff}
  In this subsection we give a short introduction to central simple $k$-algebras.
\end{fluff}


\begin{definition}
  A \emph{central simple $k$-algebra} is a finite-dimensional $k$-algebra $A$ which is simple (as a ring) and for which $\ringcenter(A) = k$.
  The class of central simple $k$-algebras is denoted by $\CSA_k$.
\end{definition}





\subsection{Tensor Products of CSA}


\begin{definition}
  If $R$ is a ring and $S \subseteq R$ a subset, then
  \[
      \centralizer_R(S)
    = \{
        r \in R
      \suchthat
        \text{$rs = sr$ for every $s \in R$}
      \}
  \]
  is the \emph{centralizer of $S$} in $R$.
\end{definition}


\begin{lemma}
  \label{lemma: centralizer componentwise}
  Let $A, B$ be $k$-algebras and let $A' \subseteq A$, $B' \subseteq B$ be subalgebras.
  Then
  \[
      \centralizer_{A \otimes B}(A' \otimes B')
    = \centralizer_A(A') \otimes \centralizer_B(B') \,.
  \]
\end{lemma}


\begin{proof}
  We have for every simple tensor $a \otimes b \in \centralizer_A(A') \otimes \centralizer_B(B')$ that
  \[
      (a \otimes b) (a' \otimes b')
    = (aa') \otimes (bb')
    = (a'a) \otimes (b'b)
    = (a' \otimes b')(a \otimes b)
  \]
  for every simple tensor $a' \otimes b' \in k \otimes B$.
  It follows that $(a \otimes b) x = x (a \otimes b)$ for every $x \in A' \otimes B'$ because every tensor is a sum of simple tensors.
  This shows that
  \[
              \centralizer_A(A') \otimes \centralizer_B(B')
    \subseteq \centralizer_{A \otimes B}(A' \otimes B') \,.
  \]
  
  To show the other inclusion let $x \in \centralizer_{A \otimes B}(A' \otimes B')$.
  We may write $x = \sum_{i=1}^n a_i \otimes b_i$ and assume w.l.o.g.\ that both $a_1, \dotsc, a_n$ and $b_1, \dotsc, b_n$ are linearly independent.
  For every $a' \in A'$ we then have that
  \[
      \sum_{i=1}^n (a' a_i) \otimes b_i
    = (a' \otimes 1) x
    = x (a' \otimes 1)
    = \sum_{i=1}^n (a_i a') \otimes b_i
  \]
  and therefore that $a_i a' = a' a_i$ for every $i = 1, \dotsc, n$ because $b_1, \dotsc, b_n$ are linearly independent (see Recall~\ref{recall: unique representation in tensor product}).
  This shows that $a_1, \dotsc, a_n \in \centralizer_{A}(A')$.
  In the same way we find that $b_1, \dotsc, b_n \in \centralizer_{B}(B')$.
  Together this shows that $x \in \centralizer_A(A') \otimes \centralizer_B(B')$.
\end{proof}


\begin{corollary}
  \label{corollary: center of tensor product}
  Let $A$ and $B$ be $k$-algebras.
  Then
  \[
      Z(A \otimes B)
    = Z(A) \otimes Z(B) \,.
  \]
\end{corollary}


\begin{proof}
  We have that
  \[
      \ringcenter(A \otimes B)
    = \centralizer_{A \otimes B}(A \otimes B)
    = \centralizer_A(A) \otimes \centralizer_B(B)
    = \ringcenter(A) \otimes \ringcenter(B)
  \]
  by Lemma~\ref{lemma: centralizer componentwise}.
\end{proof}


\begin{proposition}
  Let $A, B$ be central simple $k$-algebras.
  Then $A \otimes B$ is again a central simple $k$-algebra.
\end{proposition}


\begin{proof}
  The $k$-algebra $A \otimes B$ is finite-dimensional because both $A, B$ are finite-di\-men\-si\-o\-nal, and by Corollary~\ref{corollary: center of tensor product} we have that
  \[
      \ringcenter(A \otimes B)
    = \ringcenter(A) \otimes \ringcenter(B)
    = k \otimes k
    = k \,.
  \]
  It follows from $A, B \neq 0$ that $A \otimes B \neq 0$.
  It remains to show that the only nonzero two-sided ideal $I \idealeq A \otimes B$ is $A \otimes B$:
  
  Every $u \in I$ can be written as $u = \sum_{i=1}^n a_i \otimes b_i$ where $b_1, \dotsc, b_n \in B$ are linearly independent.
  Let $x \in I$ with $x \neq 0$ for which the number of summands $n$ is minimal with respect to all nonzero elements in $I$.
  Let
  \begin{equation}
    \label{eqn: u as a sum}
      x
    = a_1 \otimes b_1 + a_2 \otimes b_2 + \dotsb + a_n \otimes b_n
  \end{equation}
  be such a sum.
  
  We will modify $x$ such that $a_1 = 1$:
  We have that $n \geq 1$ because $x$ is nonzero and $a_1 \neq 0$ by the minimality of $n$.
  It follows that the two-sided ideal $A a_1 A \idealeq A$ is nonzero, and therefore that $A a_1 A = A$ because $A$ is simple.
  We thus have that $1 \in A a_1 A$ which is why $1 = \sum_{i=1}^m c_i a_1 c'_i$ for suitable coefficients $c_i, c'_i \in C$.
  We then have that $x' \in I$ for
  \[
              x'
    \defined  \sum_{i=1}^m (c_i \otimes 1) x (c'_i \otimes 1)
    =         1 \otimes b_1 + a'_2 \otimes b_2 + \dotsb + a'_n \otimes b_n
  \]
  with $a'_2, \dotsc, a'_n \in A$.
  We have that $x' \neq 0$ because $b_1, \dotsc, b_n$ are linearly independent.
  
  Now we show that $x'$ is already of the form $x' = 1 \otimes b$ for some $b \in B$:
  For every $a \in A$ the element
  \[
        (a \otimes 1) x' - x' (a \otimes 1)
    =     (a a'_2 - a'_2 a) \otimes b_2
        + \dotsb
        + (a a'_n - a'_n a) \otimes b_2
  \]
  is contained in $I$.
  It thus follows from the minimality of $n$ that
  \[
      (a \otimes 1) x' - x' (a \otimes 1)
    = 0 \,.
  \]
  Because $b_2, \dotsc, b_n$ are linearly independent it follows that $a a'_i - a'_i a = 0$ for all $a \in A$ and $i = 2 \dotsc, n$.
  We therefore have that $a'_2, \dotsc, a'_n \in Z(A) = k$.
  It follows that
  \begin{align*}
        x'
    &=  1 \otimes b_1 + a'_2 \otimes b_2 + \dotsb + a'_n \otimes b_n  \\
    &=  1 \otimes b_1 + 1 \otimes (a'_2 b_2) + \dotsb + 1 \otimes (a'_n b_n)  \\
    &=  1 \otimes (b_1 + a'_2 b_2 + \dotsb + a'_n b_n)  \\
    &=  1 \otimes b
  \end{align*}
  with $b \defined b_1 + a'_2 b_2 + \dotsb + a'_n b_n \in B$.
  
  We have that $b \neq 0$ because $x' \neq 0$, and it follows that $BbB = B$ because $B$ is simple.
  We therefore have that
  \[
              I
    \supseteq (1 \otimes B) x' (1 \otimes B)
    =         (1 \otimes B) (1 \otimes b) (1 \otimes B)
    =         1 \otimes (BbB)
    =         1 \otimes B \,.
  \]
  It follows that
  \[
              I
    \supseteq (A \otimes 1) (1 \otimes B)
    =         A \otimes B \,.
  \]
  This shows that $A \otimes B$ is the only non-zero two-sided ideals in $A \otimes B$.
\end{proof}


\begin{remark}
  It can be shown more generally that if $A$ is a central simple $k$-algebra and $B$ is any $k$-algebra then every two-sided ideal of $A \otimes B$ is of the form $A \otimes J$ for a two-sided ideal $J \idealeq B$.
  A proof of this can be found in \cite[Lemma~4.1]{Clark2012NonCA}.
\end{remark}


\begin{fluff}
  We have shown that
  \[
              [A] \cdot [B]
    \defined  [A \otimes B]
  \]
  is a well-defined binary operation on the set of isomorphic classas $\CSA_k\!/{\cong}$.
  This operation is also associative, commutative and we have that
  \[
      [k] \cdot [A]
    = [k \otimes A]
    = [A] \,,
  \]
  which shows that $[k]$ is neutral.
  Altogether we have thus endowed $\CSA_k\!/{\cong}$ with the structure of a commutative monoid.
\end{fluff}


\begin{warning}
  For central $k$-algebras $A, B$ their tensor product $A \otimes B$ does not need to be simple.
  A counterexample is given by
  \begin{align*}
            \Complex \otimes_\Real \Complex
     \cong  \Complex \otimes \Real[X]/(X^2 + 1)
    &\cong  \Complex[X]/(X^2 + 1)
     =      \Complex[X]/((X-1)(X+1))
    \\
    &\cong  \Complex[X]/(X-1) \times \Complex[X]/(X+1)
     \cong  \Complex \times \Complex \,.
  \end{align*}
  where we use the chinese reminder theorem for the second to last isomorphism.
\end{warning}





\subsection{Brauer Equivalence}


\begin{fluff}
  Much of the rest of this subsection is taken from \cite[4.2]{Clark2012NonCA}, except that the author added most of the proofs himself.
\end{fluff}


\begin{definition}
  A division $k$-algebra $D$ is \emph{central} if $\ringcenter(D) = k$.
  The class of finite-dimensional central $k$-algebras is denotey by $\CDA_k$.
\end{definition}


\begin{lemma}
  \label{lemma: center of matrix ring}
  For every $n \geq 0$ the map
  \[
            \ringcenter(R)
    \to     \ringcenter(\Mat_n(R))  \,,
    \quad   z
    \mapsto z I_n
  \]
  is an isomorphism of rings.
\end{lemma}


\begin{fluff}
  By \hyperref[theorem: wedderburns theorem]{Wedderburn’s theorem} every finite-dimensional simple $k$-algebra is isomorphic to $\Mat_n(D)$ where $D$ is a finite-dimensional division $k$-algebra, which is unique up to isomorphism.
  It follows from Lemma~\ref{lemma: center of matrix ring} that
  \[
          \ringcenter(D)
    \cong \ringcenter(\Mat_n(D))
    \cong \ringcenter(A)
    \cong k \,,
  \]
  which shows that $D$ is already a central division $k$-algebra.
  It follows that there exists a well-defined map
  \[
        \CSA_k\!/{\cong}
    \to \CDA_k\!/{\cong}
  \]
  wich maps $[A]$ to $[D]$.
  This map is surjective because every $D \in \CDA_k$ is in particular a central simple $k$-algebra, for which the isomorphism class $[D] = [\Mat_1(D)] \in \CSA_k\!/{\cong}$ is mapped to $[D] \in \CDA_k\!/{\cong}$.
\end{fluff}


\begin{definition}
  Two central simple $k$-algebras $A, B$ are \emph{Brauer equivalent} if for the central divsion $k$-algebras $D_1, D_2$ with $A \cong \Mat_n(D_1)$ and $B \cong \Mat_m(D_2)$ (for suitable $n, m$) we have that $D_1 \cong D_2$.
  Brauer equivalence is denoted by $\sim$.
\end{definition}


\begin{corollary}
  Brauer equivalence is an equivalence relation on $\CSA_k$ and the map
  \[
        \CSA_k\!/{\sim}
    \to \CDA_k\!/{\cong} \,,
    \quad   [\Mat_n(D)]
    \mapsto [D]
  \]
  is a well-defined bijection.
\end{corollary}


\begin{proof}
  For any $A, B \in \CSA_k$ we have that $A \sim B$ if and only if $A$ and $B$ are mapped to the same element by the composition
  \[
        \CSA_k
    \to \CSA_k\!/{\cong}
    \to \CDA_k\!/{\cong} \,.
  \]
  It follows that $\sim$ is an equivalence relation.
  The above composition is surjective and thus descends to the desired bijection.
\end{proof}


\begin{remark}
  Isomorphic central simple $k$-algebras are Brauer equivalent.
  By abuse of notation we will to the refer to the equivalence relation induced by $\sim$ on $\CSA_k\!/{\cong}$ also as Brauer equivalence and write $[A] \sim [B]$ if $A \sim B$ for all $[A], [B] \in \CSA_k\!/{\cong}$.
\end{remark}


\begin{fluff}
  We will now show that $\sim$ is a congruence relation on $\CSA_k\!/{\cong}$, and that the induced monoid structure on $(\CSA_k\!/{\cong})/{\sim} \cong \CSA_k\!/{\sim}$ is already a group structure.
\end{fluff}


\begin{lemma}
  Let $n, m \geq 0$.
  \begin{enumerate}
    \item
      If $D$ is a $k$-algebra then
      \[
              \Mat_n(D)
        \cong D \otimes \Mat_n(k) \,.
      \]
      as $k$-algebras.
    \item
      We have that $\Mat_n(k) \otimes \Mat_m(k) \cong \Mat_{nm}(k)$ as $k$-algebras.
  \end{enumerate}
\end{lemma}


\begin{proof}
  \leavevmode
  \begin{enumerate}
    \item
      There exists a unique $k$-linear map $\varphi \colon D \otimes \Mat_n(k) \to \Mat_n(D)$ which is on simple tensors given by $d \otimes M \mapsto dM$ by the universal property of the tensor product.
      Let $E_{ij}$ be the standard $D$-basis of $\Mat_n(D)$ and let $E'_{ij}$ be the standard $k$-basis of $\Mat_n(k)$.
      Then
      \[
              D \otimes \Mat_n(k)
        =     D \otimes \bigoplus_{i,j=1}^n k E_{ij}
        \cong \bigoplus_{i,j=1}^n D \otimes (k E_{ij})
      \]
      as $k$-vector spaces and $\varphi$ maps $D \otimes (k E_{ij})$ isomorphically onto $D E_{ij}$.
      This shows that $\varphi$ is an isomorphism of $k$-vector spaces.
      The multiplicativity of $\varphi$ can be checked on simple tensors, for which we have that 
      \begin{align*}
            \varphi( (d \otimes M) (d' \otimes M') )
         =  \varphi( (d d') \otimes (M M') )
        &=  d d' M M' \\
        &=  d M d' M'
         =  \varphi(d \otimes M) \varphi(d' M') \,.
      \end{align*}
      We also have that $\varphi(1 \otimes I) = I$.
    \item
      We have that
      \begin{align*}
                \Mat_n(k) \otimes \Mat_m(k)
        &\cong  \End_k(k^n) \otimes \End_k(k^m) \\
        &\cong  \End_k(k^n \otimes k^m)
         \cong \End_k(k^{nm})
         \cong \Mat_{nm}(k)
      \end{align*}
      where the second isomorphism is given by $f \otimes g \mapsto f \otimes g$.
    \qedhere
  \end{enumerate}
\end{proof}


\begin{remark}
  The above isomorphism $\Mat_n(k) \otimes \Mat_m(k) \to \Mat_{nm}(k)$ can be expressed on simple tensors by the Kronecker product:
  \[
            \varphi
    \colon  \Mat_n(k) \otimes \Mat_m(k)
    \to     \Mat_{nm}(k),
    \quad   A \otimes B
    \mapsto \begin{bmatrix}
              A_{11} B  & \cdots  & A_{1n} B  \\
              \vdots    & \ddots  & \vdots    \\
              A_{n1} B  & \cdots  & A_{nn} B
            \end{bmatrix}.
  \]
  It can also be checked by hand that this $k$-linear map is an isomorphis of $k$-algebras:
  The basis $E_{ij} \otimes E_{i'j'}$ with $i,j = 1, \dotsc, n$ and $i', j' = 1, \dotsc, m$ of $\Mat_n(k) \otimes \Mat_m(k)$ is mapped bijectively onto the analogous standard basis of $\Mat_{mn}(k)$, which shows that $\varphi$ is bijective.
  We have that
  \begin{align*}
      &\,  \varphi(A \otimes B) \varphi(A' \otimes B'))
    \\
    =&\,  \begin{bmatrix}
            A_{11} B  & \cdots  & A_{1n} B  \\
            \vdots    & \ddots  & \vdots    \\
            A_{n1} B  & \cdots  & A_{nn} B
          \end{bmatrix}
          \cdot
          \begin{bmatrix}
            A'_{11} B'  & \cdots  & A'_{1n} B'  \\
            \vdots      & \ddots  & \vdots      \\
            A'_{n1} B'  & \cdots  & A'_{nn} B'
          \end{bmatrix}
    \\
    =&\,  \begin{bmatrix}
            \sum_{j=1}^n A_{1j} B A'_{j1} B'  & \cdots  & \sum_{j=1}^n A_{1j} B A'_{jn} B'  \\
            \vdots                            & \ddots  & \vdots                            \\
            \sum_{j=1}^n A_{nj} B A'_{j1} B'  & \cdots  & \sum_{j=1}^n A_{nj} B A'_{jn} B'
          \end{bmatrix}
    \\
    =&\,  \begin{bmatrix}
            \sum_{j=1}^n A_{1j} A'_{j1} B B'  & \cdots  & \sum_{j=1}^n A_{1j} A'_{jn} B B'  \\
            \vdots                            & \ddots  & \vdots                            \\
            \sum_{j=1}^n A_{nj} A'_{j1} B B'  & \cdots  & \sum_{j=1}^n A_{nj} A'_{jn} B B'
          \end{bmatrix}
          \\
    =&\,  \begin{bmatrix}
            (A A')_{11} B B' & \cdots  & (A A')_{1n} B B' \\
            \vdots           & \ddots  & \vdots           \\
            (A A')_{n1} B B' & \cdots  & (A A')_{nn} B B'
          \end{bmatrix}
    \\
    =&\,  \varphi((A A') \otimes (B B'))
    =     \varphi((A \otimes B) (A' \otimes B'))
  \end{align*}
  for all simple tensors $A \otimes B, A' \otimes B' \in \Mat_n(k) \otimes \Mat_m(k)$, which shows that $\varphi$ is multiplicative.
  We also have that $\varphi(I \otimes I) = I$.
\end{remark}


\begin{lemma}
  \label{lemma: characterization of Brauer equivalence}
  For simple central $k$-algebras $A, B$ the following are equivalent:
  \begin{enumerate}
    \item
      \label{enumerate: matrices over isomorphic division algebra}
      We have that $A \sim B$, i.e.\ there exist $n, m \geq 1$ and division $k$-algebras $D_1 \cong D_2$ with $A \cong \Mat_n(D_1)$ and $B \cong \Mat_m(D_2)$ as $k$-algebras.
    \item
      \label{enumerate: matrices over same division algebra}
      There exist $n, m \geq 1$ and a division $k$-algebra $D$ such that $A \cong \Mat_n(D)$ and $B \cong \Mat_m(D)$ as $k$-algebras.
    \item
      \label{enumerate: up to tensor with a matrix ring}
      There exists $n', m' \geq 1$ with $A \otimes \Mat_{n'}(k) \cong B \otimes \Mat_{m'}(k)$ as $k$-algebras.
  \end{enumerate}
\end{lemma}



\begin{proof}
  \leavevmode
  \begin{description}
    \item[\ref*{enumerate: matrices over isomorphic division algebra} $\implies$ \ref*{enumerate: matrices over same division algebra}]
      Choose $D = D_1$.
    \item[\ref*{enumerate: matrices over same division algebra} $\implies$ \ref*{enumerate: up to tensor with a matrix ring}]
      We can choose $n' = m$ and $m' = n$ because
      \begin{align*}
              A \otimes \Mat_m(k)
        &\cong \Mat_n(D) \otimes \Mat_m(k)
         \cong D \otimes \Mat_n(k) \otimes \Mat_m(k)  \\
        &\cong D \otimes \Mat_m(k) \otimes \Mat_n(k)
         \cong \Mat_m(D) \otimes \Mat_n(k)
         \cong B \otimes \Mat_n(k) \,.
      \end{align*}
    \item[\ref*{enumerate: up to tensor with a matrix ring} $\implies$ \ref*{enumerate: matrices over isomorphic division algebra}]
      Let $D_1, D_2$ be division $k$-algebras with $A \cong \Mat_n(D_1)$ and $B \cong \Mat_m(D_2)$ as $k$-algebras.
      Then
      \begin{align*}
                A \otimes \Mat_{n'}(k)
         \cong  \Mat_n(D_1) \otimes \Mat_{n'}(k)
        &\cong  D_1 \otimes \Mat_n(k) \otimes \Mat_{n'}(k)  \\
        &\cong  D_1 \otimes \Mat_{nn'}(k)
         \cong  \Mat_{nn'}(D_1)
      \end{align*}
      and similarly
      \[
              B \otimes \Mat_{m'}(k)
        \cong \Mat_{mm'}(D_2) \,.
      \]
      It follows from $\Mat_{nn'}(D_1) \cong \Mat_{mm'}(D_2)$ and the \hyperref[theorem: artin wedderburn theorem]{theorem of Artin--Wedderburn} that $D_1 \cong D_2$ as $k$-algebras (and $nn' = mm'$).
    \qedhere
  \end{description}
\end{proof}


\begin{corollary}
  \label{corollary: Brauer equivalence is a congruence relation}
  Brauer equivalence is an congruence relation on $\CSA_k\!/{\cong}$, i.e.\ for $A, A', B, B' \in \CSA_k$ with $A \sim A'$ and $B \sim B'$ we have that $A \otimes B \sim A' \otimes B'$.
\end{corollary}


\begin{proof}
  There then exist $n, n', m, m' \geq 1$ with
  \[
          A \otimes \Mat_n(k)
    \cong A' \otimes \Mat_{n'}(k)
    \quad\text{and}\quad
          B \otimes \Mat_m(k)
    \cong B'  \otimes \Mat_{m'}(k)
  \]
  by Lemma~\ref{lemma: characterization of Brauer equivalence}.
  It follows that
  \begin{align*}
            A \otimes B \otimes \Mat_{nm}(k)
    &\cong  A \otimes B \otimes \Mat_{n}(k) \otimes \Mat_{m}(k) \\
    &\cong  A \otimes \Mat_{n}(k) \otimes B \otimes \Mat_{m}(k) \\
    &\cong  A' \otimes \Mat_{n'}(k) \otimes B' \otimes \Mat_{m'}(k) \\
    &\cong  A' \otimes B' \otimes \Mat_{n'}(k) \otimes \Mat_{m'}(k) \\
    &\cong  A' \otimes B' \otimes \Mat_{n'm'}(k) \,,
  \end{align*}
  which shows that $A \otimes B \sim A' \otimes B'$ by Lemma~\ref{lemma: characterization of Brauer equivalence}.
\end{proof}


\begin{lemma}
  Let $A$ be a central simple $k$-algebra.
  \begin{enumerate}
    \item
      The algebra $A^\op$ is again a central simple $k$-algebra.
    \item
      We have that $A \otimes A^\op \cong \End_k(A)$ as $k$-algebras.
  \end{enumerate}
\end{lemma}


\begin{proof}
  \leavevmode
  \begin{enumerate}
    \item
      The two-sided ideals of $A^\op$ are precisely the two-sided ideals of $A$, of which there are precisely two.
      We have that $\ringcenter(A^\op) = \ringcenter(A) = k$ and the dimension $\dim_k(A^\op) = \dim_k(A)$ is finite.
    \item
      We denote the multiplication of $A^\op$ by $*$.
      
      The left $A$-module structure of $A$ itself corresponds to a $k$-algebra homomorphism $f \colon A \to \End_k(A)$ with $f(a)(x) = ax$ for all $a \in A$, $x \in A$.
      The right $A$-module structure of $A$ itself corresponds to a left $A^\op$-module structure of $A$, which in turn corresponds to a $k$-algebra homomorphism $f \colon A^\op \to \End_k(A)$ given by $f(b)(x) = xb$ for all $b \in A^\op$, $x \in A$.
      Then $f, g$ induce a well-defined $k$-linear map
      \[
                \varphi
        \colon  A \otimes A^\op
        \to     \End_k(A)
      \]
      which is given on simple tensors by
      \[
          \varphi(a \otimes b)(x)
        = (f(a) \circ g(b))(x)
        = a x a'
      \]
      for all $a \otimes b \in A \otimes A^\op$, $x \in A$.
      For all simple tensors $a \otimes b, a' \otimes b' \in A \otimes A^\op$ we have that
      \begin{align*}
            \varphi(a \otimes a') \varphi(b \otimes b')
        &=  f(a) \circ g(b) \circ f(a') \circ g(b')
         =  f(a) \circ f(a') \circ g(b) \circ g(b') \\
        &=  f(a a') \circ g(b * b')
         =  \varphi((a a') \otimes (b * b'))
         =  \varphi((a \otimes b) (a' \otimes b'))
      \end{align*}
      where we use for the second equality that $g(b)$ and $f(a')$ commute by the associativity of the multiplication of $A$.
      We also have that $\varphi(1 \otimes 1) = \id_A$.
      
      Altogether this shows that $\varphi$ is a homomorphism of $k$-algebras.
      The kernel $\ker(\varphi)$ is a nonzero two-sided ideal in the central simple $k$-algebra $A \otimes A^\op$.
      It follows that $\ker(\varphi) = 0$ which shows that $\varphi$ is injective.
      We have that
      \[
          \dim_k (A \otimes A^\op)
        = \dim_k(A) \dim_k(A^\op)
        = \dim_k(A)^2
        = \dim_k \End_k(A) \,,
      \]
      so it follows that $\varphi$ is already an isomorphism.
    \qedhere
  \end{enumerate}
\end{proof}


\begin{theorem}
  \label{theorem: existence of brauer group}
  The binary operation
  \[
              [A] \cdot [B]
    \defined  [A \otimes B]
  \]
  defines on $\CSA_k\!/{\sim}$ the structure of an abelian groups.
\end{theorem}


\begin{proof}
  Brauer equivalence is a congruence relation on $\CSA_k\!/{\cong}$ by Corollary~\ref{corollary: Brauer equivalence is a congruence relation}, from which it follows that this binary operation is well-defined.
  Since $\CSA_k\!/{\cong}$ was is commutative monoid the same goes for $\CSA_k\!/{\sim}$.
  For every $[A] \in \CSA_k\!/{\sim}$ with $n = \dim_k A$ we have that
  \[
      [A] \cdot [A^\op]
    = [A \otimes A^\op]
    = [\End_k(A)]
    = [\Mat_n(k)]
    = [k] \,,
  \]
  which shows that $[A^\op]$ is inverse to $[A]$.
\end{proof}


\begin{definition}
  The group $\Br(k) \defined \CSA_k\!/{\sim}$ as described in Theorem~\ref{theorem: existence of brauer group} is the \emph{Brauer group} of the field $k$.
\end{definition}


\begin{example}
  \leavevmode
  \begin{enumerate}
    \item
      If $k$ is algebraically closed then every finite-dimensional division $k$-algebra is already $k$ itself.
      It follows that the Brauer group $\Br(k)$ is trivial.
    \item
      It is a classical result due to Frobenius that the only finite-dimensional division $\Real$-algebras are $\Real, \Complex, \Quaternion$.
      Both $\Real$ and $\Quaternion$ are central, while $\Complex$ is not.
      It follows that $\CDA\!/{\cong} = {[\Real], [\Quaternion]}$ has two elements.
      The Brauer group $\Br(\Real)$ is therefore the cyclic group of order two.
    \item
      A theorem of Wedderburn (which is not to be confused with \hyperref[theorem: wedderburns theorem]{Wedderburn’s theorem}) states that every finite skew field is already commutative, and thus a field.
      It follows that for a finite field $k$ the only \emph{central} finite-dimensional $k$-division ring is $k$ itself.
      It follows that the Brauer group $\Br(k)$ is trivial.
  \end{enumerate}
\end{example}






\subsection{The Skolem--Noether Theorem}


\begin{fluff}
  The main ideas of this section are taken from \cite[4.3]{Clark2012NonCA}
\end{fluff}


\begin{example}
  \label{example: every automorphism of matrix ring is inner}
  As a motivation for the upcoming theorem and its proof we first consider a special case:
  
  Let $n \geq 1$ and let $\alpha \colon \Mat_n(k) \to \Mat_n(k)$ be a $k$-algebra automorphism.
  The usual matrix-vector multiplication makes $k^n$ into a simple $\Mat_n(k)$-module which we will denote by $M$.
  By using the automorphism $\alpha$ we can \enquote{twist} this module structure, resulting in a $\Mat_n(k)$-module $M_\alpha$ whose underlying $k$-vector space is again $k^n$ but whose multiplication is given by
  \[
              A * x
    \defined  \varphi(A) x
  \]
  for all $A \in \Mat_n(k)$, $x \in k^n$.
  It follows from the surjectivity of $\alpha$ that $M_\alpha$ is also simple.
  
  The $k$-algebra $\Mat_n(k)$ has only one simple module up to isomorphism (namely $M$), so it follows that $M \cong M_\alpha$.
  If $f \colon M \to M_\alpha$ is such an isomorphism then $f$ is in particular $k$-linear and therefore given by multiplication with some invertible matrix $S \in \GL_n(k)$.
  The inverse $f^{-1}$ is then given by multiplication with $S^{-1}$
  
  It follows for every $A \in \Mat_n(k)$ that
  \[
      \alpha(A) x
    = A * x
    = f(A f^{-1}(x))
    = S A S^{-1} x
  \]
  for every $x \in k^n$, and therefore that $\alpha(A) = S A S^{-1}$.
  We have thus found that $\alpha$ is an inner automorphism, given by conjugation with the unit $S \in \Mat_n(k)^\times$.
\end{example}


\begin{lemma}
  \label{lemma: isomorphic iff same dimension}
  If $A$ is a finite-dimensional simple $k$-algebra and $M, N$ are finite-dimensional $A$-module then $M, N$ are isomorphic as $A$-modules if and only if they have the same $k$-dimension.
\end{lemma}


\begin{proof}
  The algebra $A$ is semisimple and there exists only one simple $A$-module $E$ up to isomorphism, for which it follows from the finite-dimensionality of $A$ that it is also finite-dimensional.
  It follows for $M \cong E^{\oplus m}$ and $N \cong N^{\oplus n}$ that
  \[
    \dim_k M = m \dim_k E
    \qquad\text{and}\qquad
    \dim_k N = n \dim_k E
  \]
  and therefore that
  \[
          M \cong N
    \iff  m = n
    \iff  \dim_k M = \dim_k N
  \]
  by the uniqueness of multiplicities.
\end{proof}


\begin{corollary}
  \label{corollary: skolem noether for into matrix rings}
  If $A$ is a finite-dimensional simple $k$-algebra then any two $k$-algebra homomorphisms $f, g \colon A \to \Mat_n(k)$ are conjugated, i.e.\ there exists some $S \in \GL_n(k)$ with $g(a) = S f(a) S^{-1}$ for every $a \in A$.
\end{corollary}


\begin{proof}
  The homomorphisms $f, g$ correspond to $A$-module structures on $k^n$ given by
  \[
    a \cdot x = f(a)x
    \qquad\text{and}\quad
    a * x = g(a)x
  \]
  for all $a \in A$, $x \in k^n$.
  We denote the resulting $A$-modules by $M_f$ and $M_g$.
  It follows from Lemma~\ref{lemma: isomorphic iff same dimension} that $M_f$ and $M_g$ are isomorphic as $A$-modules.
  It follows in the same way as in Example~\ref{example: every automorphism of matrix ring is inner} that there exists some $S \in \GL_n(k)$ with $g(a) = S f(a) S^{-1}$ for every $a \in A$.
\end{proof}


\begin{theorem}[Skolem--Noether]
  If $A$ is a simple $k$-algebra and $B$ is a central simple $k$-algebra then any two $k$-algebra homomorphisms $f, g \colon A \to B$ are conjugated, i.e.\ there exists a unit $u \in B^\times$ with $g(a) = u f(a) u^{-1}$ for every $a \in A$.
\end{theorem}


\begin{proof}
  Note first that $A$ must be finite-dimensional:
  The kernel $\ker(f)$ is a proper two-sided ideal of $A$ because $B \neq 0$.
  It follows that $\ker(f) = 0$ and therefore that $\dim_k B \leq \dim_k A$ by the injectivity of $f$
  
  By observing that $B \otimes B^\op \cong \End_k(B)$ we can now apply Corollary~\ref{corollary: skolem noether for into matrix rings} to the extended $k$-algebra homomorphisms
  \[
            f \otimes \id, g \otimes \id
    \colon  A \otimes B^\op
    \to     B \otimes B^\op
  \]
  to conclude that $f \otimes \id$ and $g \otimes \id$ are conjugated:
  There exists some $x \in B \otimes B^\op$ with
  \begin{equation}
    \label{equation: noether skolem formula}
      g(a) \otimes b
    = x ( f(a) \otimes b ) x^{-1}
  \end{equation}
  for all $a \in A$, $b \in B^\op$.
  By setting $a = 1$ we find that
  \[
      1 \otimes b
    = x (1 \otimes b) x^{-1}
  \]
  for all $b \in B$, which shows that $x, x^{-1} \in \centralizer{B \otimes B^\op}(1 \otimes B^\op) = \centralizer{B \otimes B^\op}(k \otimes B^\op)$.
  It follows from Lemma~\ref{lemma: centralizer componentwise} that
  \[
      \centralizer{B \otimes B^\op}(k \otimes B^\op)
    = \centralizer{B}(k) \otimes \centralizer{B^\op}(B^\op)
    = \centralizer{B}(k) \otimes \ringcenter(B^\op)
    = B \otimes k \,.
  \]
  We therefore have that $x = u \otimes 1$ and $x^{-1} = u' \otimes 1$ for some $u, u' \in B$.
  It follows from
  \[
      1 \otimes 1
    = x x^{-1}
    = (u \otimes 1) (u' \otimes 1)
    = (u u') \otimes 1
  \]
  that $u u' = 1$ and similarly that $u' u = 1$.
  This shows that $u \in B^\times$ is a unit with $u' = u^{-1}$.
  By setting $b = 1$ in \eqref{equation: noether skolem formula} we find that
  \[
      g(a) \otimes 1
    = (u \otimes 1) (f(a) \otimes 1) (u^{-1} \otimes 1)
    = (u f(a) u^{-1}) \otimes 1
  \]
  and therefore $g(a) = u f(a) u^{-1}$ for every $a \in A$.
\end{proof}


\begin{corollary}
  Every $k$-algebra automorphism of a central simple $k$-algebra $A$ is inner, i.e.\ given by conjugation with an element $u \in A^\times$.
\end{corollary}


\begin{proof}
  Every automorphism is conjugated to the identity $\id_A$ by the Skolem--Noether theorem.
\end{proof}
