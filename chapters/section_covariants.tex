\section{Covariants}
% TODO: Move this section.


\begin{fluff}
  \label{fluff: induced action on polynomial algebra}
  Let $G$ be a group acting on a set $X$
  Then $G$ acts on linearly on $\Maps(X,k)$ via
  \[
              (g.f)(x)
    \defined  f(g^{-1}.x)
  \]
  for all $g \in G$, $f \in \Maps(X,k)$, $x \in X$.
  
  If $G$ acts linearly on a finite-dimensional $k$-vector space $V$ then it follows that $G$ acts linearly on $\Maps(V,k)$ in the above way.
  The $k$-linear subspace $\mc{P}(V) \subseteq \Maps(V,k)$ is then a subrepresentation (because the precomposition of a polynomial function by a linear function is again polynomial).
  
  This shows that the linear action of $G$ on $V$ induces a linear action of $G$ on $\mc{P}(V)$ given by $(g.f)(v) = f(g^{-1}.v)$.
  Note that this is already an action by $k$-algebra automorphisms because additionaly
  \begin{align*}
        (g.(f_1 f_2))(v)
    &=  (f_1 f_2)(g^{-1}.v)
     =  f_1(g^{-1}.v) f_2(g^{-1}.v) \\
    &=  (g.f_1)(v) \, (g.f_2)(v)
     =  ((g.f_1) (g.f_2))(v)
  \end{align*}
  for every $v \in V$, and thus $g.(f_1 f_2) = (g.f_1)(g.f_2)$, as well as
  \[
      \left( g.1_{\mc{P}(V)} \right)(v)
    = 1_{\mc{P}(V)}(g^{-1}.v)
    = 1
  \]
  for every $v \in V$, and thus $g.1_{\mc{P}(V)} = 1_{\mc{P}(V)}$.
  It follows in particular that $\mc{P}(V)^G$ is a $k$-subalgebra of $\mc{P}(V)$.
\end{fluff}


\begin{conventions}
  In the following, $U, V, W$ will denote finite-dimensional representations of a group $G$ over an infinite field $k$.
\end{conventions}


\begin{definition}
  A map $f \colon V \to W$ is \emph{covariant} if it is both polynomial and $G$-equivariant.
  The space of covariant functions $V \to W$ is denoted by $\Cov_k(V,W)$, or just by $\Cov(V,W)$.
\end{definition}


\begin{example}
  \leavevmode
  \begin{enumerate}
    \item
      For every $r \geq 0$ the map
      \[
                \beta
        \colon  V \to V^{\tensor r},
        \quad   v
        \mapsto v \tensor \dotsb \tensor v
      \]
      is polynomial, as seen in Example~\ref{example: polynomial maps}.
      It is also $G$-equivariant since
      \[
          \beta(g.v)
        = (g.v) \tensor \dotsb \tensor (g.v)
        = g.(v \tensor \dotsb \tensor v)
        = g.\beta(v)
      \]
      for all $g \in G$, $v \in W$.
    \item
      The group $\GL_n(k)$ on $\Mat_n(k)$ via conjugation, i.e.\ via
      \[
                  g.A
        \defined  gAg^{-1}
      \]
      for all $g \in \GL_n(k)$, $A \in \Mat_n(k)$.
      Then the map
      \[
                \beta_i 
        \colon  \Mat_n(k) 
        \to     \Mat_n(k),
        \quad   A
        \mapsto A^i
      \]
      is covariant for every $i \geq 1$. 
  \end{enumerate}
\end{example}


\begin{fluff}
%   Notice that since we have $\Hom_k(W,V) \subseteq \Pol_k(W,V)$ we also have
%   \[
%               \Hom_G(W,V)
%     =         \Hom_k(W,V)^G
%     \subseteq \Pol_k(W,V)^G
%     =         \Cov(W,V) \,.
%   \]
%   Therefore every morphism of representations is covariant.
% 
% 
  We know that $\Maps(V,W)$ is a $G$-set via
  \begin{equation}
    \label{equation: action on mapsVW}
      (g.f)(v)
    = g.f\left( g^{-1}.v \right)
  \end{equation}
  for all $g \in G$, $v \in V$.
  As in~\ref{fluff: induced action on polynomial algebra} we find that $\Pol(V,W) \subseteq \Maps(V,W)$ is a subrepresentation.
  Thus $G$ acts linearly on $\Pol(V,W)$ via~\eqref{equation: action on mapsVW}.
  
  Note that it follows in particular that $\Cov(V,W) = \Pol_k(V,W)^G$ is a $k$-linear subspace of $\Pol_k(V,W)$.
  
  Let $\beta \colon V \to W$ be a covariant map.
  Then the induced $k$-algebra homomorphism $\beta^* \colon \mc{P}(W) \to \mc{P}(V)$ is also $G$-equivariant because
  \begin{align*}
        (g.\beta^*(f))(v)
    &=  \beta^*(f)(g^{-1}.v)
     =  f(\beta(g^{-1}.v))  \\
    &=  f(g^{-1}.\beta(v))
     =  (g.f)(\beta(v))
     =  \beta^*(g.f)(v)
  \end{align*}
  for all $g \in G$, $f \in \mc{P}(W)$, $v \in V$, and therefore $g.\beta^*(f) = \beta^*(g.f)$ for all $g \in G$, $f \in \mc{P}(W)$.
  It follows that $\beta^*(\mc{P}(W)^G) \subseteq \mc{P}(V)^G$, so that $\beta^*$ restricts to a $k$-algebra homomorphism $\mc{P}(W)^G \to \mc{P}(V)^G$.
\end{fluff}


\begin{proposition}
  The $\mc{P}(V)$-module structure of $\Pol(V,W)$ restrict to a $\mc{P}(V)^G$-module structure on $\Cov(V,W)$ by restriction.
\end{proposition}
\begin{proof}
  The $\mc{P}(V)$-module structure of $\Pol(V,W)$ restricts to a $\mc{P}(V)^G$-module structure.
  We thus need to show that $\Cov(V,W)$ is preserved under the action of $\mc{P}(V)^G$.
  
  We already know that $\Cov(W,V)$ is a $k$-linear subspace of $\Pol_k(W,V)$.
  For all $f \in \mc{P}(W)^G$, $\beta \in \Cov(W,V)$ we have for all $g \in G$, $v \in V$ that
  \begin{align*}
        (g.(f \cdot \beta))(v)
    &=  g.\left( (f \cdot \beta) \left( g^{-1}.v \right) \right)
     =  g.\left(
                f\left( g^{-1}.v \right)
          \cdot \beta\left( g^{-1}.v \right)
        \right) \\
    &=        f\left( g^{-1}.v \right)
        \cdot \left(
                g.\beta\left( g^{-1}.v \right)
              \right)
     =  (g.f)(v) \cdot (g.\beta)(v) \\
    &=  f(v) \cdot \beta(v)
     =  (f \cdot \beta)(v) \,,
  \end{align*}
  where we used in the second to last equality that $g.f = f$ and $g.\beta = \beta$ (note that $\beta \in \Cov(V,W)$ is $G$-equivariant, und thus $G$-invariant).
  This shows the claim.
\end{proof}
