\section{The Zariski Topology}


\begin{fluff}
  In this section we show that the Zariski closed subsets define a topology on $V$ and use this to extend the $1$:$1$-correspondences from Corollary~\ref{corollary: general correspondence for algebraic subsets and points} and Corollary~\ref{corollary: algebraically closed correspondence for algebraic subsets and points} to include prime ideals as well.
\end{fluff}


\begin{lemma}
  \label{lemma: intersections and unions of Zariski closed sets}
  \leavevmode
  \begin{enumerate}
    \item
      Let $(I_j)_{j \in J}$ be a family of ideals $I_j \idealeq \mc{P}(V)$.
      Then $\bigcap_{j \in J} \mc{V}(I_j) = \mc{V}( \sum_{j \in J} I_j )$.
    \item
      Let $I_1, I_2 \idealeq \mc{P}(V)$ be ideals.
      Then $\mc{V}(I_1) \cup \mc{V}(I_2) = \mc{V}(I_1 \cap I_2) = \mc{V}(I_1 I_2)$.
  \end{enumerate}
\end{lemma}
\begin{proof}
  \leavevmode
  \begin{enumerate}
    \item
      We have that $I_i \subseteq \sum_{j \in J} I_j$ for every $i \in J$, therefore $\mc{V}( \sum_{j \in J} I_j ) \subseteq \mc{V}(I_i)$ for every $i \in I$, and thus $\mc{V}( \sum_{j \in J} I_j ) \subseteq \bigcap_{j \in J} \mc{V}(I_j)$.
      
      For $x \in \bigcap_{j \in J} \mc{V}(I_j)$ we have that $x \in \mc{V}(I_j)$ for every $j \in J$, and thus $f(x) = 0$ for every $j \in J$, $f \in I_j$.
      It follows that $x \in \mc{V}( \bigcup_{j \in J} I_j ) = \mc{V}( (\bigcup_{j \in J} I_j) ) = \mc{V}( \sum_{j \in J} I_j )$.
      This shows the inclusion $\bigcap_{j \in J} \mc{V}(I_j) \subseteq \mc{V}( \sum_{j \in J} I_j )$.
    \item
      It follows from $I_1 \cap I_2 \subseteq I_1, I_2$ that $\mc{V}(I_1), \mc{V}(I_2) \subseteq \mc{V}(I_1 \cap I_2)$ and therefore that $\mc{V}(I_1) \cup \mc{V}(I_2) \subseteq \mc{V}(I_1 \cap I_2)$.
      
      It follows from $I_1 I_2 \subseteq I_1 \cap I_2$ that $\mc{V}(I_1 \cap I_2) \subseteq \mc{V}(I_1 I_2)$.
      
      For $x \notin \mc{V}(I_1) \cup \mc{V}(I_2)$ we have that $x \notin \mc{V}(I_1), \mc{V}(I_2)$, so there exist $f_1 \in I_1$, $f_2 \in I_2$ with $f_1(x), f_2(x) \neq 0$.
      Then $f_1 f_2 \in I_1 I_2$ with $(f_1 f_2)(x) \neq 0$, so that $x \notin \mc{V}(I_1 I_2)$.
      This shows that $\mc{V}(I_1 I_2) \subseteq \mc{V}(I_1) \cup \mc{V}(I_2)$.
    \qedhere
  \end{enumerate}
\end{proof}


\begin{corollary}
  There exists a topology on $V$ whose closed subsets are precisely the Zariski closed subsets.
\end{corollary}


\begin{proof}
  The subsets $\emptyset, V \subseteq V$ are Zariski closed by Example~\ref{example: examples of algebraic subsets} and are closed under arbitrary intersections and finite unions by Lemma~\ref{lemma: intersections and unions of Zariski closed sets}.
\end{proof}


\begin{definition}
  The \emph{Zariski topology \textup(on $V$\textup)} is the topology on $V$ whose closed subsets are the Zariski closed subsets of $V$.
\end{definition}


\begin{proposition}
  \label{proposition: characterization of Zariski closed and Zariski dense}
  Let $X \subseteq Y \subseteq V$ be subsets.
  \begin{enumerate}
    \item
      We have that $\closure{X} = \mc{V}(\mc{I}(X))$.
    \item
      \label{enumerate: Zariski density is Zariski density}
      The subset $X$ is dense in $Y$ with respect to the Zariski topology if and only if $X$ is Zariski dense in $Y$ (in the sense of Definition~\ref{definition: Zariski density}).
    \item
      The closure $\closure{X}$ is the biggest subset of $V$ in which $X$ is Zariski dense.
  \end{enumerate}
\end{proposition}


\begin{proof}
  \leavevmode
  \begin{enumerate}
    \item
      We have that
      \begin{align*}
            \closure{X}
        &=  \bigcap \{ C \suchthat \text{$C \subseteq V$ is Zariski closed, $X \subseteq C$} \}
         =  \bigcap \{ \mc{V}(I) \suchthat I \idealeq \mc{P}(V), X \subseteq \mc{V}(I) \} \\
        &=  \bigcap \{ \mc{V}(I) \suchthat I \idealeq \mc{P}(V), \mc{I}(X) \supseteq I \}
         =  \bigcap_{\substack{I \idealeq \mc{P}(V) \\ I \subseteq \mc{I}(X)}} \mc{V}(I)
         =  \mc{V}\left( \sum_{\substack{I \idealeq \mc{P}(V) \\ I \subseteq \mc{I}(X)}} I \right)
         =  \mc{V}(\mc{I}(X)) \,.
      \end{align*}
    \item
      We have that
      \begin{align*}
              \text{$X$ is dense in $Y$}
        &\iff Y \subseteq \closure{X}
         \iff Y \subseteq \mc{V}(\mc{I}(X)) \\
        &\iff \mc{I}(Y) \supseteq \mc{I}(X)
         \iff \text{$Y$ is Zariski dense in $X$} \,.
      \end{align*}
    \item
      The closure $\closure{X}$ is the biggest subset of $V$ in which $X$ is dense so the claim follows from part~\ref*{enumerate: Zariski density is Zariski density}.
    \qedhere
  \end{enumerate}
\end{proof}


\begin{remark}
  Propositon~\ref{proposition: characterization of Zariski closed and Zariski dense} shows that Zariski density in the sense of Definition~\ref{definition: Zariski density} can be understood as a topological kind of density.
  The Zariski topology is the unique topology on $V$ with this property:
  
  Suppose we are given any topology on $V$ such that Zariski density coincides with topological density.
  Then for every subset $X \subseteq V$ the closure $\overline{X}$ is the maximal subset of $V$ in which $X$ is Zariski dense, which shows that $\closure{X}$ is uniquely determined.
  A subset $X \subseteq V$ is closed if and only if $X = \closure{X}$ so it further follows that the closed subsets are uniquely determined.
  But this already determines the topology itself.
\end{remark}


\begin{remark}
  It follows from Lemma~\ref{lemma: properties of V and I} that the composition
  \[
              c
    \defined  \mc{V} \circ \mc{I}
    \colon    \{ \text{subsets $X \subseteq V$} \}
    \to       \{ \text{subsets $X \subseteq V$} \}
  \]
  is idempotent and monotone with $X \subseteq c(X)$ for every subset $X \subseteq V$.
  The map $c$ is therefore a closure operator.
  We also have that
  \[
      c(\emptyset)
    = \mc{V}(\mc{I}(\emptyset))
    = \mc{V}((1))
    = \emptyset,
  \]
  and for all subsets $X, Y \subseteq V$ we have that
  \[
      c(X \cup Y)
    = \mc{V}(\mc{I}(X \cup Y))
    = \mc{V}(\mc{I}(X) \cap \mc{I}(Y))
    = \mc{V}(\mc{I}(X)) \cup \mc{V}(\mc{I}(Y))
    = c(X) \cup c(Y) \,.
  \]
  This shows that $c$ is a Kuratowski closure operator, and thus defines a topology on $V$ whose sets are precisely those subsets $X \subseteq V$ for which $c(X) = X$.
  By Proposition~\ref{proposition: characterization of Zariski closed and Zariski dense} this topology is precisely the Zariski topology.
\end{remark}


\begin{remark}
  Building on the ideals of Remark~\ref{remark: (weak) nullstellensatz} one can think about the \hyperref[theorem: nullstellensatz]{Nullstellensatz} as the existence of a partition of unity, as explained in \cite{SBS}:
  
  Recall that for an open covering $(U_j)_{j \in J}$ of a topological space $X$ a partition of unity subordinate to this covering is a familiy $(\varphi_i)_{i \in I}$ of continuous maps $f_i \colon X \to \Real$ such that for every $i \in I$ there exists some $j \in J$ with $\supp(\varphi_i) \subseteq U_j$, the family $(\supp(\varphi)_i)_{i \in I}$ is a locally finite covering of $X$, and $1 = \sum_{i \in I} \varphi_i$.
  (Here $\supp$ denotes the support $\supp(\varphi) = \closure{\{x \in X \suchthat \varphi(x) \neq 0\}}$.)
  
  For every ideal $I \subseteq \mc{P}(V)$ we now set $U(I) \defined V \smallsetminus \mc{V}(I)$.
  Then the sets $U(I)$ with $I \idealeq \mc{P}(V)$ are precisely the Zariski open subsets of $V$, and
  \begin{itemize}
    \item
      we have that $U(S) = U((S))$ for every subset $S \subseteq \mc{P}(V)$,
    \item
      for all $S \subseteq T \subseteq \mc{P}(V)$ it follows from $S \subseteq T$ that $U(S) \subseteq U(T)$,
    \item
      for every family $(I_j)_{j \in J}$ of ideals $I_j \idealeq \mc{P}(V)$ we have that $\bigcup_{j \in J} U(I_j) = U( \sum_{j \in J} I_j )$.
  \end{itemize}
  
  Suppose now that $k$ is algebraically closed and that $(U_j)_{j \in J}$ is an open covering of $V$.
  Every set $U_j \subseteq V$ is Zariski open, and thus of the form $U_j = U(I_j)$ for some ideal $I_j \idealeq \mc{P}(V)$.
  Then $U_j$ consists of all those $a \in V$ which are not a common zero of all $f \in I_j$.
  The condition $V = \bigcup_{j \in J} U_j$ is therefore equivalent to the polynomial functions $f \in I \defined \sum_{j \in J} I_j$ having no common roots.
  This can also seen by using that
  \[
      V
    = \bigcup_{j \in J} U_j
    = \bigcup_{j \in J} U(I_j)
    = U\left( \sum_{j \in J} I_j \right)
    = U(I) \,,
  \]
  from which it follows that $\mc{V}(I) = V \smallsetminus U(I) = \emptyset$.
  
  It then follows from the \hyperref[theorem: nullstellensatz]{Nullstellensatz} (as explained in part~\ref*{enumerate: partition of unity formulation of NS} of Remark~\ref{remark: (weak) nullstellensatz}) that $1 = g_1 f_1 + \dotsb + g_n f_n$ for some $j_1, \dotsc, j_n \in J$, $f_i \in I_{j_i}$ and $g_i \in \mc{P}(V)$.
  Then the polynomials functions $\varphi_i \defined g_i f_i \colon V \to k$ satisfy $1 = \varphi_1 + \dotsb + \varphi_n$, and for every $i = 1, \dotsc, n$ we have that
  \[
              U(\varphi_i)
    =         U(g_i f_i)
    \subseteq U(f_i)
    \subseteq U(I_{j_i})
    =         U_{j_i} \,.
  \]
  We can therefore regard the functions $\varphi_1, \dotsc, \varphi_n$ as a partition of unity subordinate to the open covering $(U_j)_{j \in J}$.
\end{remark}


\begin{definition}
  A topological space $X$ is \emph{irreducible} (or \emph{hyperconnected}) if it is non-empty and cannot be written as $X = C_1 \cup C_2$ for proper closed subsets $C_1, C_2 \subsetneq X$.
  Otherwise $X$ is \emph{reducible}.
\end{definition}


\begin{remark}
  Let $X$ be a topological space.
  \begin{enumerate}
    \item
      By taking complements one find that the following conditions are equivalent:
      \begin{enumerate}
        \item
          The space $X$ is irreducible.
        \item
          The space $X$ is non-empty and every two non-empty open subsets of $X$ intersect non-trivially.
        \item
          The space $X$ is non-empty and every non-empty open subset of $X$ is dense.
      \end{enumerate}
    \item
      A non-empty subspace $C \subseteq X$ is irreducible (when endowed with the subspace topology) if and only if for all closed subsets $C_1, C_2 \subseteq X$ with $C \subseteq C_1 \cup C_2$ it follows that $C \subseteq C_1$ or $C \subseteq C_2$.
      
      We will use this observations throughout the rest of this section whenever we need to show that a subspace is irreducible.
  \end{enumerate}
\end{remark}


% TODO: Figure out if this is true, or what kind of connectedness this is.
% \begin{remark}
%   Let $X$ be a topological space.
%   The use of \emph{hyperconnectivity} instead of \emph{irreducibility} can be explained by the fact that $X$ is irreducible if and only if every non-empty open subset $U \subseteq X$ is connected:
%   \begin{itemize}
%     \item
%       Every irreducible space is in particular connected.
% 
%       If $X$ is irreducible then every non-empty open subset $U \subseteq X$ is again irreducible:
%       Let $C_1, C_2 \subseteq X$ be closed subsets with $U \subseteq C_1 \cup C_2$.
%       Then $C_1 \cup C_2 = X$ because $U$ is dense in $X$ (because $X$ is irreducible).
%       Then $C_1 = X$ or $C_2 = X$ because $X$ is irreducible.
%       This shows that $U \subseteq C_1$ or $U \subseteq C_2$.
% 
%       It follows that every non-empty open subset of $X$ is again connected.
%     \item
%       Suppose on the other hand that every non-empty open subset of $X$ is connected.
%       Suppose that $X = C_1 \cup C_2$ for some proper closed subsets $C_1, C_2 \subsetneq X$.
%       
%   \end{itemize}
% \end{remark}


\begin{lemma}
  \label{lemma: trivial prime avoidance}
  Let $R$ be a commutative ring, $\mf{p} \idealneq R$ a prime ideal and let $I_1, I_2 \idealeq R$ be ideals with $I_1 I_2 \subseteq \mf{p}$.
  Then $I_1 \subseteq \mf{p}$ or $I_2 \subseteq \mf{p}$.
\end{lemma}


\begin{proof}
  If $I_1, I_2 \subsetneq \mf{p}$ then there exist $x_j \in I_j$ with $x_j \notin \mf{p}$ for $j = 1,2$.
  Then $x_1 x_2 \notin \mf{p}$ because $\mf{p}$ is prime, but $x_1 x_2 \in I_1 I_2$, which contradicts $I_1 I_2 \subseteq \mf{p}$.
\end{proof}


\begin{lemma}
  \label{lemma: X is irreducible iff I(X) is prime}
  Let $X \subseteq V$ be a Zariski closed subset with corresponding vanishing ideal $\mf{p} \idealeq \mc{P}(V)$, i.e.\ $X = \mc{V}(\mf{p})$ and $\mf{p} = \mc{I}(X)$.
  Then $X$ is irreducible if and only if $\mf{p}$ is prime.
\end{lemma}


\begin{proof}
  That $X$ is non-empty is equivalent to $\mc{I}(X) = \mf{p}$ being a proper ideal.
  
  Suppose that $X$ is irreducible and let $f_1, f_2 \in \mc{P}(V)$ with $f_1 f_2 \in I = \mc{I}(X)$.
  It follows from Lemma~\ref{lemma: galois connection for vanishing ideals and zero sets} that
  \[
              X
    \subseteq \mc{V}(f_1 f_2)
    =         \mc{V}(f_1) \cup \mc{V}(f_2) \,.
  \]
  It follows that $X \subseteq \mc{V}(f_j)$ for some $j = 1,2$ because $X$ is irreducible, and it then follows from Lemma~\ref{lemma: galois connection for vanishing ideals and zero sets} that $f_j \in \mc{I}(X) = \mf{p}$.
  This shows that the ideal $\mf{p}$ is prime.
  
  Suppose on the other hand that $\mf{p}$ is prime and that $X = C_1 \cup C_2$ for some closed subsets $C_1, C_2 \subseteq V$.
  Then $C_1, C_2$ are also closed in $V$ because $X$ is a closed subset of $V$, so there exist ideals $I_1, I_2 \idealeq \mc{P}(V)$ with $C_j = \mc{V}(I_j)$ for $j = 1,2$.
  It then follows from
  \[
      X
    = C_1 \cup C_2
    = \mc{V}(I_1) \cup \mc{V}(I_2)
    = \mc{V}(I_1 I_2)
  \]
  and Lemma~\ref{lemma: galois connection for vanishing ideals and zero sets} that $\mf{p} = \mc{I}(X) \supseteq I_1 I_2$.
  It follows from Lemma~\ref{lemma: trivial prime avoidance} that $I_j \subseteq \mf{p}$ for some $j = 1,2$ and therefore that $C_j = \mc{V}(I_j) \supseteq \mc{V}(\mf{p}) = X$.
  This shows that $X$ is irreducible.
\end{proof}


\begin{theorem}
  \label{theorem: big correspondence theorems}
  \leavevmode
  \begin{enumerate}
    \item
      The maps $\mc{I}, \mc{V}$ restrict to the following mutually inverse bijections:
      \[
        \begin{matrix}
            \left\{
              \begin{tabular}{c}
                  algebraic subsets \\
                  $X \subseteq V$
              \end{tabular}
            \right\}
          & \begin{tikzcd}[column sep = large]
                {}
                \arrow[shift left]{r}{\mc{I}}
              & {}
                \arrow[shift left]{l}{\mc{V}}
            \end{tikzcd}
          & \left\{
              \begin{tabular}{c}
                vanishing ideals \\
                $I \idealeq \mc{P}(V)$
              \end{tabular}
            \right\}
          \\
            {}
          & {}
          & {}
          \\
            \rotatebox[origin=c]{90}{$\subseteq$}
          & {}
          & \rotatebox[origin=c]{90}{$\subseteq$}
          \\
            {}
          & {}
          & {}
          \\
            \left\{
              \begin{tabular}{c}
                  irreducible \\
                  algebraic subsets \\
                  $X \subseteq V$
              \end{tabular}
            \right\}
          & \begin{tikzcd}[column sep = large]
                {}
                \arrow[shift left]{r}{\mc{I}}
              & {}
                \arrow[shift left]{l}{\mc{V}}
            \end{tikzcd}
          & \left\{
              \begin{tabular}{c}
                vanishing ideals \\
                $\mf{p} \idealeq \mc{P}(V)$ \\
                which are prime
              \end{tabular}
            \right\}
          \\
            {}
          & {}
          & {}
          \\
            \rotatebox[origin=c]{90}{$\subseteq$}
          & {}
          & \rotatebox[origin=c]{90}{$\subseteq$}
          \\
            {}
          & {}
          & {}
          \\
            \left\{
              \begin{tabular}{c}
                points $a \in V$
              \end{tabular}
            \right\}
          & \begin{tikzcd}[column sep = large]
                {}
                \arrow[shift left]{r}{\mc{I}}
              & {}
                \arrow[shift left]{l}{\mc{V}}
            \end{tikzcd}
          & \left\{
              \begin{tabular}{c}
                vanishing ideals \\
                $\mf{m} \idealeq \mc{P}(V)$ \\
                which are maximal
              \end{tabular}
            \right\}
        \end{matrix}
      \]
    \item
      If $k$ is algebraically closed then the maps $\mc{I}, \mc{V}$ restrict to the following mutually inverse bijections:
      \[
        \begin{matrix}
            \left\{
              \begin{tabular}{c}
                  algebraic subsets \\
                  $X \subseteq V$
              \end{tabular}
            \right\}
          & \begin{tikzcd}[column sep = large]
                {}
                \arrow[shift left]{r}{\mc{I}}
              & {}
                \arrow[shift left]{l}{\mc{V}}
            \end{tikzcd}
          & \left\{
              \begin{tabular}{c}
                radical ideals \\
                $I \idealeq \mc{P}(V)$
              \end{tabular}
            \right\}
          \\
            {}
          & {}
          & {}
          \\
            \rotatebox[origin=c]{90}{$\subseteq$}
          & {}
          & \rotatebox[origin=c]{90}{$\subseteq$}
          \\
            {}
          & {}
          & {}
          \\
            \left\{
              \begin{tabular}{c}
                  irreducible \\
                  algebraic subsets \\
                  $X \subseteq V$
              \end{tabular}
            \right\}
          & \begin{tikzcd}[column sep = large]
                {}
                \arrow[shift left]{r}{\mc{I}}
              & {}
                \arrow[shift left]{l}{\mc{V}}
            \end{tikzcd}
          & \left\{
              \begin{tabular}{c}
                prime ideals \\
                $\mf{p} \idealeq \mc{P}(V)$
              \end{tabular}
            \right\}
          \\
            {}
          & {}
          & {}
          \\
            \rotatebox[origin=c]{90}{$\subseteq$}
          & {}
          & \rotatebox[origin=c]{90}{$\subseteq$}
          \\
            {}
          & {}
          & {}
          \\
            \left\{
              \begin{tabular}{c}
                points $a \in V$
              \end{tabular}
            \right\}
          & \begin{tikzcd}[column sep = large]
                {}
                \arrow[shift left]{r}{\mc{I}}
              & {}
                \arrow[shift left]{l}{\mc{V}}
            \end{tikzcd}
          & \left\{
              \begin{tabular}{c}
                maximal ideals \\
                $\mf{m} \idealeq \mc{P}(V)$
              \end{tabular}
            \right\}
        \end{matrix}
      \]
  \end{enumerate}
\end{theorem}


\begin{proof}
  These are upgraded of Corollary~\ref{corollary: general correspondence for algebraic subsets and points} and Corollary~\ref{corollary: algebraically closed correspondence for algebraic subsets and points} via Lemma~\ref{lemma: X is irreducible iff I(X) is prime}.
  For part~b) we also use that every prime ideal is already radical.
\end{proof}


\begin{fluff}
  We will now show that every topological space is the union of its irreducible components:
\end{fluff}


\begin{definition}
  Let $X$ be a topological space.
  A subset $C \subseteq X$ is an \emph{irreducible component} of $X$ if $C$ is a maximal irreducible subset of $X$, i.e.\ $C$ is irreducible, and for every irreducible subset $C' \subseteq X$ with $C \subseteq C'$ it follows that $C = C'$.
\end{definition}


\begin{lemma}
  Let $X$ be a topological space and let $Y \subseteq X$ be irreducible.
  Then the closure $\closure{Y}$ is also irreducible.
\end{lemma}


\begin{proof}
  Let $C_1, C_2 \subseteq Y$ be closed subsets with $\closure{Y} \subseteq C_1 \cup C_2$.
  Then $Y \subseteq C_1 \cup C_2$ and it follows that $Y \subseteq C_j$ for some $j = 1,2$ because $Y$ is irreducible.
  It then follows that $\closure{Y} \subseteq C_j$ because $C_j$ is closed.
\end{proof}


\begin{corollary}
  The irreducible components of a topological space $X$ are closed.
\end{corollary}


\begin{proof}
  If $C \subseteq X$ is an irreducible component then $\closure{C} \subseteq X$ is an irreducible subspace with $C \subseteq \closure{C}$
  It follows that $\closure{C} \subseteq C$ because $C$ is maximal among all irreducible subspaces.
  We thus have that $C = \closure{C}$, which shows that $C$ is closed.
\end{proof}


\begin{proposition}
  \label{proposition: irreducible components for alls topological spaces}
  Let $X$ be a topological space.
  \begin{enumerate}
    \item
      Let $(C_i)_{i \in I}$ be non-empty family of irreducible subsets $C_i \subseteq X$ which is linearly ordered with respect to inclusion, i.e.\ for all $i, j \in I$ we have that $C_i \subseteq C_j$ or $C_j \subseteq C_i$.
      Then $C \defined \bigcup_{i \in I} C_i$ is again irreducible.
    \item
      Every irreducible subset $C \subseteq X$ is contained in an irreducible component of $X$.
    \item
      Every $x \in X$ is contained in an irreducible component, i.e.\ $X$ is the union of its irreducible components.
    \item
      If $C, C' \subseteq X$ are two distinct irreducible components of $X$ then $C \nsubseteq C'$.
  \end{enumerate}
\end{proposition}


\begin{proof}
  \leavevmode
  \begin{enumerate}
    \item
      Let $C'_1, C'_2 \subseteq X$ be two distinct closed subsets with $C \subseteq C'_1 \cup C'_2$ and $C \nsubseteq C'_2$.
      It follows from $C \nsubseteq C'_2$ that there exists some $j \in I$ with $C_j \nsubseteq C'_2$.
      It also follows from $C \subseteq C'_1 \cup C'_2$ that $C_i \subseteq C'_1 \cup C'_2$ for every $i \in I$, and therefore for every $i \in I$ that $C_i \subseteq C'_1$ or $C_i \subseteq C'_2$ by the irreducibility of $C_i$.
      
      For $i = j$ we have that $C_j \subseteq C'_1$ because $C_j \nsubseteq C'_2$.
      For every other $i \in I$ we distinguish between two cases:
      \begin{itemize}
        \item
          If $C_i \subseteq C_j$ then it follows that $C_i \subseteq C'_1$.
        \item
          If $C_i \supseteq C_j$ then it follows that $C_i \subseteq C'_1$ because otherwise $C_j \subseteq C_i \subseteq C'_2$, which would contradicts the choice of $j$.
      \end{itemize}
      Alltogether this shows that $C_i \subseteq C'_1$ for every $i \in I$, so that $C = \bigcup_{i \in I} C_i \subseteq C'_1$.
    \item
      We consider the set
      \[
          \mc{C}
        = \{
            C' \subseteq X
          \suchthat
            \text{$C'$ is irreducible with $C' \supseteq C$}
          \} \,.
      \]
      This set is non-empty because it contains $C$.
      It follows from part a) of this proposition that the partially ordered set $(\mc{C},\subseteq)$ is inductive, i.e.\ we can apply Zorn’s Lemma.
      It follows that there exists a maximal element $C' \in \mc{C}$.
      Then $C'$ is in particular a maximal irreducible subspace of $X$, and thus an irreducible component of $X$.
      We have that $C \subseteq C'$ because $C' \in \mc{C}$.
    \item
      This follows from part c) of this proposition because $\{x\}$ is an irreducible subspace of $X$.
    \item
      If $C \subseteq C'$ then it follows that already $C' = C$ because $C$ is a maxmial irreducible subset of $X$.
    \qedhere
  \end{enumerate}
\end{proof}


\begin{example}
  Consider the real line $\Real$ with the standard (i.e.\ euclidian) topology.
  If $C \subseteq \Real$ contains at least two distinct points $x, y \in C$ then for $z =  (x+y)/2$ we have that $C \subseteq (-\infty,z] \cup [z,\infty)$ but $C \nsubseteq (-\infty,z], [z,\infty)$.
  This shows that the only irreducible subspaces of $\Real$ are the singletons $\{x\}$, $x \in \Real$.
  These are in particular the irreducible components of $\Real$.
\end{example}


% TODO: Better examples with pictures


\begin{fluff}
  It follows from Proposition~\ref{proposition: irreducible components for alls topological spaces} that every Zariski closed subset $V \subseteq X$ is the union of its irreducible components, each of which is closed, and which are not contained in each other.
  We will now show that a Zariski closed set has only finitely many irreducible components.
\end{fluff}


\begin{definition}
  A topological space $X$ is \emph{noetherian} if every ascending sequence
  \[
              U_1
    \subseteq U_2
    \subseteq U_3
    \subseteq \dotsb
  \]
  of open subsets $U_i \subseteq X$ stabilizes;
  equivalently, every descending chain
  \[
              C_1
    \supseteq C_2
    \supseteq C_3
    \supseteq \dotsb
  \]
  of closed subsets $C_i \subseteq X$ stabilizes.
\end{definition}


\begin{lemma}
  \label{lemma: noetherian via max min elements of collections}
  A topological space $X$ is noetherian if and only if every non-empty collection $\mc{U}$ of open subsets has a maximal element;
  equivalently, every non-empty family of closed subsets has a minimal element.
\end{lemma}


\begin{proof}
  Suppose that there exists a non-empty collection $\mc{U}$ of open subsets of $X$ which does not have a maximal element.
  Starting with any $U_1 \in \mc{U}$ there then exists for every $n \geq 1$ some $U_{n+1} \in \mc{U}$ with $U_n \subsetneq U_{n+1}$.
  It then follows that
  \[
                U_1
    \subsetneq  U_2
    \subsetneq  U_3
    \subsetneq  \dotsb
  \]
  is an increasing sequence of open subsets of $X$ which does not stabilize.
  This contradicts $X$ being noetherian.
  
  Suppose on the other hand that
  \[
              U_1
    \subseteq U_2
    \subseteq U_3
    \subseteq \dotsb
  \]
  is an increasing sequence of open subsets $U_n \subseteq X$.
  Then $\mc{U} = \{U_n \suchthat n \geq 1\}$ has a maximal element, i.e.\ there exists some $m \geq 1$ with $U_m \supseteq U_n$ for every $n \geq 1$.
  It then follows for all $n \geq m$ that $U_n = U_m$, which shows that the above sequence stabilizes.
  This shows that $X$ is noetherian.
\end{proof}


\begin{proposition}
  \label{proposition: irreducible components of noetherian space alternative construction}
  Let $X$ be a noetherian topological space.
  Then there exist closed irreducible subsets $C_1, \dotsc, C_n \subseteq X$ with $X = C_1 \cup \dotsb \cup C_n$ and $C_i \nsubseteq C_j$ for all $i \neq j$.%, and the $C_i$ are unique up to permutation.
\end{proposition}


\begin{proof}
  Suppose that $X$ is not the union of finitely many closed irreducible subsets.
  Then the set
  \[
      \mc{C}
    = \left\{
        C \subseteq X
      \suchthat*
        \begin{tabular}{c}
          $C$ is closed and not the union of \\
          finitely many closed irreducible subsets
        \end{tabular}
      \right\}
  \]
  contains $X$ and is therefore non-empty.
  It follows from Lemma~\ref{lemma: noetherian via max min elements of collections} that $\mc{C}$ contains a minimal element $C \in \mc{C}$ because $X$ is noetherian.
  Note that $\emptyset \notin \mc{C}$ because $\emptyset$ is the union of zero closed irreducible subsets.
  The set $C$ is therefore non-empty.
  
  The set $C$ cannot be irreducible because otherwise $C \notin \mc{C}$.
  Because $C$ is non-empty it follows that there exist proper closed subsets $C_1, C_2 \subsetneq C$ with $C = C_1 \cup C_2$.
  The sets $C_1, C_2$ cannot be the union of finitely many closed irreducible subspaces because otherwise the same would hold for $C = C_1 \cup C_2$, which would contradict $C \in \mc{C}$.
  Both $C_1, C_2$ are closed in $X$ because they are closed in $C$ which is closed in $X$.
  This shows that $C_1, C_2 \in \mc{C}$.
  But this contradicts the minimality of $C$.
  
  It follows that $X = C_1 \cup \dotsb \cup C_m$ for some closed irreducible subsets $C_i \subseteq X$.
  If $C_i \subseteq C_j$ for some $i \neq j$ then we may eliminate $C_i$ from the collection $C_1, \dotsc, C_m$ without losing the property that $X = C_1 \cup \dotsb \cup C_m$.
  After finitely many eliminations we arrive at closed irreducible subsets $C_1, \dotsc, C_n \subseteq X$ with $X = C_1 \cup \dotsb \cup C_n$ and $C_i \nsubseteq C_j$ for all $i \neq j$.
%   
%   Let $C'_1, \dotsc, C'_m$ be another collection of closed irreducible subsets $C'_j \subseteq X$ with $X = C'_1 \cup \dotsb \cup C'_m$ and $C'_i \nsubseteq C'_j$ for all $i \neq j$.
%   For every $i = 1, \dotsc, n$ it follows from $C_i \subseteq X = C'_1 \cup \dotsb \cup C'_m$ and the irreducibility of $C_i$ that $C_i \subseteq C'_{\sigma(i)}$ for some $\sigma(i)$.
%   In the same way we find that for every $j = 1, \dotsc, m$ we have $C'_j \subseteq C_{\tau(j)}$ for some $\tau(j)$.
%   
%   It then follows for every $i = 1, \dotsc, n$ that
%   \[
%               C_i
%     \subseteq C'_{\sigma(i)}
%     \subseteq C_{\tau(\sigma(i))}
%   \]
%   and therefore that $i = \tau(\sigma(i))$.
%   In the same way we find that $\sigma(\tau(j)) = j$ for every $j = 1, \dotsc, m$.
%   This shows that the maps
%   \[
%             \tau \circ \sigma
%     \colon  \{1, \dotsc, n\}
%     \colon  \{1, \dotsc, m\}
%     \qquad\text{and}\quad
%             \sigma \circ \tau
%     \colon  \{1, \dotsc, m\}
%     \colon  \{1, \dotsc, n\}
%   \]
%   are mutually inverse bijections.
%   It follows in particular $n = m$.
%   For every $i = 1, \dotsc, n$ we have that
%   \[
%               C_i
%     \subseteq C'_{\sigma(i)}
%     \subseteq C_{\tau(\sigma(i))}
%     =         C_i
%   \]
%   and therefore $C_i = C'_{\sigma(i)}$.
%   This shows that the two families
%   \[
%     C_1, \dotsc, C_n
%     \quad\text{and}\quad
%     C'_1, \dotsc, C'_m = C'_n \,.
%   \]
%   are the same up to permutation.
\end{proof}


\begin{lemma}
  \label{lemma: recognizing irreducible components}
  Let $X$ be a topological space.
  If $X = C_1 \cup \dotsb \cup C_n$ for closed irreducible subsets $C_1, \dotsc, C_n \subseteq X$ with $C_i \nsubseteq C_j$ for all $i \neq j$ then $C_1, \dotsc, C_n$ are the irreducible components of $X$.
\end{lemma}


\begin{proof}
  Let $C'$ be an irreducible subset of $X$.
  Then $C' \subseteq X = C_1 \cup \dotsb \cup C_n$ and it follows from the irreducibilty of $C'$ that $C' \subseteq C_i$ for some $i$.
  This shows that every irreducible subset of $X$ is contained in some $C_i$.
  
  If $C'$ is an irreducible component of $X$ then it follows that $C' \subseteq C_i$ for some $i = 1, \dotsc, n$.
  It then follows that $C' = C_i$ because $C'$ is a maximal irreducible subset of $X$ and $C_i$ is irreducible.
  This shows that every irreducible component occurs as some $C_i$.
  
  Fix some $i = 1, \dotsc, n$.
  If $C' \subseteq X$ is an irreducible subset with $C_i \subseteq C'$ then $C'$ is contained in some $C_j$ because $C'$ is irreducible.
  It follows that $C_i \subseteq C_j$ and therefore that $i = j$.
  Then $C_i \subseteq C' \subseteq C_i$ and thus $C_i = C$.
  This shows that the $C_i$ are maximal irreducible subsets of $X$, i.e.\ irreducible components of $X$.
\end{proof}


\begin{corollary}
  \label{corollary: noetherian spaces have only finitely many irreducible components}
  A noetherian topological space $X$ has only finitely many irreducible components.
\end{corollary}


\begin{corollary}
  There exist closed irreducible subsets $C_1, \dotsc, C_n \subseteq X$ such that $X = C_1 \cup \dotsb \cup C_n$ and $C_i \nsubseteq C_j$ for all $i \neq j$ by Proposition~\ref{proposition: irreducible components of noetherian space alternative construction}, and the $C_i$ are the irreducible components of $X$ by Lemma~\ref{lemma: recognizing irreducible components}.
\end{corollary}


\begin{lemma}
  \label{lemma: algebraic spaces are noetherian}
  \leavevmode
  \begin{enumerate}
    \item
      The space $V$ (together with the Zariski topology) is noetherian.
    \item
      If $X$ is a noetherian topological space then every subspace $Y \subseteq X$ is noetherian.
  \end{enumerate}
\end{lemma}


\begin{proof}
  \leavevmode
  \begin{enumerate}
    \item
      Let
      \begin{equation}
      \label{equation: increasing chain of closed subsets}
                  C_1
        \supseteq C_2
        \supseteq C_3
        \supseteq \dotsb
      \end{equation}
      be a decreasing sequence of closed subsets $C_n \subseteq X$.
      Then
      \[
                  \mc{I}(C_1)
        \subseteq \mc{I}(C_2)
        \subseteq \mc{I}(C_3)
        \subseteq \dotsb
      \]
      is an increasing sequence of ideals in $\mc{P}(V) \cong k[X_1, \dotsc, X_{(\dim V)}]$, which is noetherian.
      It follows that this chain stabilizes, so there exists some $m \geq 1$ with $\mc{I}(C_n) = \mc{I}(C_m)$ for every $n \geq m$.
      For every $n \geq m$ it then follows that
      \[
          C_m
        = \mc{V}(\mc{I}(C_m))
        = \mc{V}(\mc{I}(C_n))
        = C_n \,.
      \]
      This shows that the sequence~\eqref{equation: increasing chain of closed subsets} stabilizes.
    \item
      Let $\mc{U} = \{U_i \suchthat i \in I\}$ be a collection of open subsets of $Y$.
      Then for every $i \in I$ there exists an open subset $V_i \subseteq X$ with $U_i = V_i \cap X$, and $\mc{V} = \{V_i \suchthat i \in I\}$ is a collection of open subsets of $X$.
      Then $\mc{V}$ contains a maximal element because $X$ is noetherian, i.e.\ there exists some $j \in I$ with $V_j \supseteq V_i$ for every $i \in I$.
      It follows that $U_j \supseteq U_i$ for every $i \in I$.
      This shows that $Y$ is noetherian.
    \qedhere
  \end{enumerate}  
\end{proof}


\begin{corollary}
  Every $X \subseteq V$ has only finitely many irreducible components.
\end{corollary}


\begin{proof}
  It follows from Lemma~\ref{lemma: algebraic spaces are noetherian} that $X$ is noetherian, so the statement follows from Corollary~\ref{corollary: noetherian spaces have only finitely many irreducible components}.
\end{proof}



% TODO: Every radical ideal is intersection of maximal ideals

% TODO: Every prime ideal is intersection of maxmial ideals

% TODO: Jacobson rings



