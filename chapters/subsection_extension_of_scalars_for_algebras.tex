\subsection{Extension of Scalars for Algebras}


\begin{fluff}
  \label{fluff: tensor product of algebras}
  For $k$-algebras $A$ and $B$ their tensor product $A \tensor_k B$ can again be endowed with the structure of a $k$-algebra, with the multiplication being given on simple tensors by
  \[
      (a_1 \tensor b_1) \cdot (a_2 \tensor b_2)
    = (a_1 a_2) \tensor (b_1 b_2).
  \]
  for all $a_1, a_2 \in A$, $b_1, b_2 \in B$.
  If both $A, B$ are unital then $A \tensor_k B$ is again unital with $1_{A \tensor B} = 1_A \tensor 1_B$.
  
  To see that this multiplication is well-defined note that the map
  \begin{align*}
              A \times B \times A \times B
    &\to      A \tensor_k B \,, \\
              (a_1, b_1, a_2, b_2)
    &\mapsto  (a_1 a_2) \tensor (b_1 b_2)
  \end{align*}
  is well-defined and $k$-multilinear, and thus induces a well-defined $k$-linear map
  \begin{align*}
              A \tensor_k B \tensor_k A \tensor_k B
    &\to      A \tensor_k B \,, \\
              a_1 \tensor b_1 \tensor a_2 \tensor b_2
    &\mapsto  (a_1 a_2) \tensor (b_1 b_2) \,,
  \end{align*}
  which in turn corresponds to a $k$-bilinear map
  \begin{align*}
              (A \tensor_k B) \times (A \tensor_k B)
    &\to      A \tensor_k B \,, \\
              (a_1 \tensor b_1, a_2 \tensor b_2)
    &\mapsto  (a_1 a_2) \tensor (b_1 b_2) \,.
  \end{align*}
  The various algebra axioms can now be checked on simple tensors.
\end{fluff}


\begin{fluff}
  Let $A$ be a $k$-algebra.
  By considering $L$ as a $k$-algebra it now follows that $A_L = L \tensor_k A$ carries the structure of a $k$-algebra, with the multiplication being given on simple tensors by
  \[
      (l_1 \tensor a_1) \cdot (l_2 \tensor a_2)
    = (l_1 l_2) \tensor (a_1 a_2)
  \]
  for all $l_1, l_2 \in L$, $a_1, a_2 \in A$.
  The $k$-bilinear map
  \begin{align*}
              (L\tensor_k A) \times (L \tensor_k A)
    &\to      L \tensor_k A \,, \\
              (l_1 \tensor a_1, l_2 \tensor a_2)
    &\mapsto  (l_1 l_2) \tensor (a_1 a_2)
  \end{align*}
  is already $L$-bilinear, so the $L$-vector space structure of $A_L$ makes the $k$-algebra $A_L$ into an $L$-algebra.
\end{fluff}


\begin{remark}
  Let $A$ be a $k$-algebra.
  \begin{enumerate}
    \item
    If $A$ is unital then so is $A_L$ with $1_{A_L} = 1 \tensor 1_A$.
    \item
    The canonical homomorphism $\can_A \colon A \to A_L$ is already a homomorphism of $k$-algebras.
  \end{enumerate}
\end{remark}


\begin{lemma}
  If $f \colon A \to B$ is a homomorphism of $k$-algebras, then the induced $L$-linear map $f_L \colon A_L \to B_L$ is a homomorphism of $L$-algebras.
\end{lemma}
\begin{proof}
  The map $f_L$ is multiplicative on simple tensors because
  \begin{align*}
     &\,  f_L((l_1 \tensor a_1)(l_2 \tensor a_2))
    =     f_L((l_1 l_2) \tensor (a_1 a_2))
    =     (l_1 l_2) \tensor f(a_1 a_2) \\
    =&\,  (l_1 l_2) \tensor (f(a_1)f(a_2))
    =     (l_1 \tensor f(a_1)) (l_2 \tensor f(a_2))
    =     f_L(l_1 \tensor a_1) f_L(l_2 \tensor a_2)
  \end{align*}
  for all $l_1, l_2 \in L$, $a_1, a_2 \in A$.
  It follows that $f_L$ is multiplicative because these simple tensors generate $L \tensor_k A_1$ as a vector space i.
\end{proof}
  
  
\begin{remark}
  With this we have seen that the extension of scalars defines a functor $(-)_L \colon \cAlg{k} \to \cAlg{L}$.
\end{remark}


\begin{lemma}
  \label{lemma: universal property of extension of scalars for algebras}
  Let $A$ be a $k$-algebra, $B$ an $L$-algebra and $f \colon A \to B$ a homomorphism of $k$-algebras.
  Then the corresponding $L$-linear map $\overline{f} \colon A_L \to B$ is a homomorphism of $L$-algebras.
  \begin{proof}
  The map $\overline{f}$ is multiplicative on simple tensors because
    \begin{align*}
          \overline{f}((l_1 \tensor a_1)(l_2 \tensor a_2))
      &=  \overline{f}((l_1 l_2) \tensor (a_1 a_2))
       =  l_1 l_2 f(a_1 a_2)    \\
      &=  l_1 l_2 f(a_1) f(a_2)
       =  l_1 f(a_1) l_2 f(a_2)
       =  \overline{f}(l_1 \tensor a_1) \overline{f}(l_2 \tensor a_2)
    \end{align*}
    for all $l_1, l_2 \in L$, $a_1, a_2 \in A$.
    It follows that $\overline{f}$ is multiplicative because the simple tensors generate $A_L$ as a vector space.
  \end{proof}
\end{lemma}


\begin{remark}
  It follows from Lemma~\ref{lemma: universal property of extension of scalars for algebras} that for a $k$-algebra $A$ and an $L$-algebra $B$ the bijection
  \[
            \Phi_{A,B}
    \colon  \Hom_L(A_L, B)
    \to     \Hom_k(A, B),
    \quad   f
    \mapsto f \circ \can_A
  \]
  restricts to a bijection
  \[
            \Psi_{A,B}
    \colon  \Hom_{\cAlg{L}}(A_L, B)
    \to     \Hom_{\cAlg{k}}(A, B) \,.
  \]
  This can also be formulated by giving a variation of Theorem~\ref{theorem: universal property of extension of scalars} for algebras.
  
  It follows that the functor $(-)_L \colon \cAlg{k} \to \cAlg{L}$ is left adjoint to the forgetful functor $R \colon \cAlg{L} \to \cAlg{k}$.
\end{remark}


\begin{fluff}
  We have seen in Corollary~\ref{corollary: inclusion to bijection vector spaces} that it is possible to realize the extension of scalars of a $k$-vector $V$ as a given $L$-vector space $W$ with $V \subseteq W$ under suitable conditions.
  This also generalizes to algebras:
\end{fluff}


\begin{corollary}
\label{corollary: inclusion to bijection algebras}
  Let $B$ be an $L$-algebra and $A \subseteq B$ a $k$-subalgebra.
  Suppose that $X \subseteq A$ is both a $k$-basis of $A$ and an $L$-basis of $B$.
  Then the isomorphism of $L$-vector spaces $\varphi \colon A_L \to B$ from Corollary~\ref{corollary: inclusion to bijection vector spaces}, which is given on simple tensors by
  \[
      \varphi(l \tensor a)
    = l a
  \]
  for all $l \in L$, $a \in A$, is an isomorphism of $L$-algebras.
\end{corollary}
\begin{proof}
  The inclusion $A \hookrightarrow B$ is a homomorphism of $k$-algebras, so it follows from Lemma~\ref{lemma: universal property of extension of scalars for algebras} that the induced $L$-linear map $A_L \to B$, which is precisely the isomorphism $\varphi$, is a homomorphism of $L$-algebras.
\end{proof}


\begin{example}
  The isomorphisms of $L$-vector spaces
  \begin{gather*}
    k[X_1, \dotsc, X_n]_L \to L[X_1, \dotsc, X_n] \,,
    \quad
    \Mat_n(k)_L \to \Mat_n(L) \,,
    \quad
    k[G]_L \to L[G]
  \end{gather*}
  from Example~\ref{example: recognizing extension of scalar} are all isomorphisms of $L$-algebras.
\end{example}


\begin{lemma}
  Let $A$ be a $k$-algebra and let $I \idealleq A$ be a left-ideal (resp.\ right-ideal, resp.\ both sided ideal).
  Then $I_L$ is a left-ideal (resp.\ right-ideal, resp.\ both sided ideal) in $A_L$.
\end{lemma}
\begin{proof}
  It follows from $I$ being a $k$-linear subspace of $A$ that $I_L$ is an $L$-linear subspace of $A_L$.
  For all simple tensors $l \tensor a \in A_L$, $l' \tensor x \in I_L$ with $l, l' \in L$, $a \in A$, $x \in I$ we have that
  \[
        (l \tensor a) \cdot (l' \tensor x)
    =   (l l') \tensor (a x)
    \in L \tensor_k I
    =   I_L \,.
  \]
  It follows that $A_L I_L \subseteq I_L$ because every tensor is a linear combination of simple tensors.
  This shows that $I_L$ is a left ideal in $A_L$.
  The case of $I$ being a right ideal can be treated in the same way, and the case of $I$ being a two-sided ideal follows from the previous two cases.
\end{proof}


\begin{lemma}
  Let $A$ be a $k$-algebra and let $I \idealleq A$ be an ideal generated by elements $(b_j)_{j \in J}$.
  Then the ideal $I_L \idealleq A_L$ is generated by the elements $(1 \tensor b_j)_{j \in J}$.
\end{lemma}
\begin{proof}
  Let $I_0$ be the ideal in $A_L$ generated by the elements $(1 \tensor b_j)_{j \in J}$.
  We have that $I_0 \subseteq I_L$ because $I_L$ is an ideal with $1 \tensor b_j \in I_L$ for every $j \in J$.
  To show that $I_L \subseteq I_0$ we consider the preimage
  \[
              I'
    \defined  \{
                a \in A
              \mid
                1 \tensor a \in I_0
              \}
    =         \can_A^{-1}(I_0) \,.
  \]
  This is an ideal in $A$ because $\can_A$ is a homomorphism of $k$-algebras.
  We have that $b_j \in I'$ for every $j \in J$, and thus $I \subseteq I'$.
  It follows that $1 \tensor a \in I_0$ for all $a \in I$, and it further follows that $I_L \subseteq I_0$ because these simple tensors generate $I_L$ as an $L$-vector space.
\end{proof}


\begin{warning}
  The ideal $(X^2+1)_{\Real[X]} \subseteq \Real[X]$ is a prime ideal, but the ideal \mbox{$(X^2+1)_{\Complex[X]} \subseteq \Complex[X]$} is not.
  This shows that for a prime (resp.\ maxmial) ideal $P \idealleq A$ the ideal $P_L \idealleq A_L$ is not necessarily prime (resp.\ maximal).
\end{warning}




