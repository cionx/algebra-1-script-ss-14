\section{Schur--Weyl Duality}



\begin{example}
Let $e_1$, $e_2$ be the standard basis of $\Complex^2$.
The usual action of $\GL_2(\Complex)$ on $\Complex^2$ induces an action of $\GL_2(\Complex)$ on $V \coloneqq \Complex^2 \otimes_\Complex \Complex^2$ with
  \[
      A.(v \otimes w)
    = (Av) \otimes (Aw)
  \]
  for all matrices $A \in \GL_2(\Complex)$ and simple tensors $v \otimes w \in V$.
  We also have an action of $S_2 = \{e, s\}$ on $V$ with
  \[
      s.(v \otimes w)
    = w \otimes v
  \]
  for every simple tensor $v \otimes w \in V$.
  It is clear that both actions commute and therefore induces an action of $\GL_2(\Complex) \times S_2$ on $V$ with
  \[
      (A,\sigma).v
    = A.\sigma.v
    = A.(\sigma.v)
    = \sigma.(A.v)
  \]
  for every $(A,\sigma) \in \GL_2(\Complex) \times S_2$ and $v \in V$.
  
  $V$ is completely reducible as a representation of $S_2$ with
  \[
                  V
    \cong         \Complex
          \oplus  \Complex
          \oplus  \Complex
          \oplus  \sgn
  \]
  where $\Complex$ is the one-dimensional trivial representation of $S_2$ and $\sgn$ the sign-representation of $S_2$.
  (Notice that these are, up to isomorphism, the only irreducible representations of $S_2$ by Corollary \ref{corollary: number of irreducible representations of finite abelian group} because $|\Irr(S_2)| = |S_2| = 2$.)
  The corresponding trivial subrepresentations of $V$ are spanned by $e_1 \otimes e_1$, $e_2 \otimes e_2$ and $e_1 \otimes e_2 + e_2 \otimes e_1$ respectively.
  The corresponding sign-subrepresentation of $V$ is spanned by $e_1 \otimes e_2 - e_2 \otimes e_1$.
  
  $V$ is also completely reducible as a representations of $\GL_2(\Complex)$:
  The usual isomorphism of $\Complex$-vector spaces
  \[
            V
    \cong   S^2\left( \Complex^2 \right)  \oplus  \Lambda^2\left( \Complex^2 \right),
    \quad   v \otimes w
    \mapsto (v \cdot w, v \wedge w)
  \]
  is clearly $\GL_2(\Complex)$-equivariant and thus an isomorphism of representations of $\GL_2(\Complex)$.
  
  \begin{claim}
    Both $S^2(\Complex^2)$ and $\Lambda^2(\Complex^2)$ are irreducible as representations of $\GL_2(\Complex)$.
  \end{claim}
  \begin{proof}[Proof of the claim]
    It is clear that $\Lambda^2(\Complex^2)$ is irreducible as it is one-dimensional.
    To show that $S^2(\Complex^2)$ is irreducible let $x \in \Lambda^2(\Complex^2)$ with $x \neq 0$ and $U \subseteq S^2(\Complex^2)$ be the subrepresentation generated by $x$.
    Since $e_1^2$, $e_1 e_2$, $e_2^2$ is a basis of $S^2(\Complex^2)$ we can write
    \[
        x
      =   \lambda_1 e_1^2
        + \lambda_2 e_1 e_2
        + \lambda_3 e_2^2
    \]
    with unique $\lambda_1, \lambda_2, \lambda_3 \in \Complex$.
    
    We first notice that $e_1 e_2 \in U$:
    If $\lambda_2 \neq 0$ we
    \[
        e_1 e_2
      = \frac{1}{2 \lambda_2} \cdot (x - A.x)
      \in U
    \]
    for
    \[
                A
      \coloneqq \begin{bmatrix}
                  1 &  0 \\
                  0 & -1
                \end{bmatrix}
      \in       \GL_2(\Complex).
    \]
    If $\lambda_2 = 0$ we have $x = \lambda_1 e_1^2 + \lambda_3 e_2^2$.
    In the case of $\lambda_3 = 0$ we then have $\lambda_1 \neq 0$, thus $e_1^2 \in U$ and therefore
    \[
          e_1 e_2
      =   \frac{1}{2} \left(
                        B.\left( e_1^2 \right) - e_1^2 - C.\left( e_1^2 \right)
                      \right)
      \in U
    \]
    for the matrices $B, C \in \GL_2(\Complex)$ with
    \[
        B
      = \begin{bmatrix}
          1 & 0 \\
          1 & 1
        \end{bmatrix}
      \text{ and }
        C
      = \begin{bmatrix}
          0 & 1 \\
          1 & 0
        \end{bmatrix}
    \]
    If $\lambda_3 \neq 0$ we have
    \[
          e_1 e_2
      =   \frac{1}{4 \lambda_3} E.(D.x - x)
      \in U
    \]
    for the matrices $D, E \in \GL_2(\Complex)$ with
    \[
        D
      = \begin{bmatrix}
          1 & 1 \\
          0 & 1
        \end{bmatrix}
      \text{ and }
        E
      = \begin{bmatrix}
          2 & -1 \\
          0 &  1
        \end{bmatrix}.
    \]
    Since $e_1 e_2 \in U$ we also have
    \begin{gather*}
          e_1^2
      =   A.(e_1 e_2) - e_1 e_2
      \in U
    \shortintertext{for}
                A
      \coloneqq \begin{bmatrix}
                  1 & 1 \\
                  0 & 1
                \end{bmatrix}
      \in       \GL_2(\Complex)
    \end{gather*}
    as well as
    \begin{gather*}
          e_2^2
      =   B.(e_1 e_2) - e_1 e_2
      \in U
    \shortintertext{for}
                B
      \coloneqq \begin{bmatrix}
                  1 & 0 \\
                  1 & 1
                \end{bmatrix}
      \in       \GL_2(\Complex) \,.
    \end{gather*}
    So $U$ contains a basis of $S^2(\Complex^2)$ and therefore $U = \Complex^2$.
  \end{proof}
  
  As a representation of $\GL_2(\Complex) \times S_2$ we now have
  \[
          V
    \cong         S^2\left( \Complex \right) \boxtimes_\Complex \Complex
          \oplus  \Lambda^2\left( \Complex \right) \boxtimes_\Complex \sgn.
  \]
\end{example}


In this section let $k$ be an infinite field.
For a $k$-vector space $V$ and $d \geq 1$ we let $\GL(V)$ act on $V^{\otimes d}$ in the usual way, i.e.\
\[
    \psi.(v_1 \otimes \dotsb \otimes v_d)
  = (\psi.v_1) \otimes \dotsb \otimes (\psi.v_d)
\]
for all $\psi \in \GL(V)$ and simple tensors $v_1 \otimes \dotsb \otimes v_d \in V^{\otimes d}$.
This turns $V^{\otimes d}$ into a representation of $\GL(V)$.
We can also turn $V^{\otimes d}$ into a representation of $S_d$ with
\[
    \pi.(v_1 \otimes \dotsb \otimes v_d)
  = v_{\pi(1)} \otimes \dotsb \otimes v_{\pi(d)}
\]
for every permutation $\pi \in S_d$ and simple tensors $v_1 \otimes \dotsb \otimes v_d \in V^{\otimes d}$. 
Is is clear that the actions of $\GL(V)$ and $S_d$ commute.

\begin{theorem}(Schur--Weyl duality)
  Let $\gen{\GL(V)}$ denote the image of the algebra homomorphism
  \[
            k \GL(V)
    \to     \End_k\left( V^{\otimes d} \right),
    \quad   g
    \mapsto (x \mapsto g.x)
  \]
  and $\gen{S_d}$ the image of the algebra homomorphism
  \[
            k S_d
    \to     \End_k\left( V^{\otimes d} \right),
    \quad   \sigma
    \mapsto (x \mapsto \sigma.x).
  \]
  \begin{enumerate}[label=\emph{(\alph*)}, leftmargin=*]
    \item \label{enum: end sd = gl}
      $\End_{S_d}(V^{\otimes d}) = \gen{\GL(V)}$.
    \item \label{enum: end gl = sd}
      If $\kchar k = 0$ or $\kchar k > d$ then $\End_{\GL(V)}(V^{\otimes d}) = \gen{S_d}$.
  \end{enumerate}
\end{theorem}
\begin{proof}
  We first prove \ref{enum: end gl = sd} assuming \ref{enum: end sd = gl}:
  $V^{\otimes d}$ is a finite-dimenisonal representation of $S_d$ over $k$, where $\kchar k = 0$ or $\kchar k > d$.
  The group algebra $k S_d$ is semisimple by Maschke’s theorem, so $V^{\otimes d}$ is completely reducible as a $k S_d$-module.
  Since $k S_d$ is semisimple and $A \coloneqq \gen{S_d} \subseteq \End_k(V^{\otimes d})$ is a quotient of $k S_d$ we find that $A$ is also semisimple.
  By \ref{enum: end sd = gl} we have $A' = \gen{GL(V)}$ where $A'$ denotes the commutator of $A$ in $\End_k(V^{\otimes d})$.
  Applying the double centralizer theorem we find that
  \[
      \End_{\GL(V)}\left( V^{\otimes d} \right)
    = \gen{\GL(V)}'
    = A''
    = A
    = \gen{S_d}.
  \]
  
  Now we show \ref{enum: end sd = gl}:
  We have an isomorphism of $k$-vector spaces
  \begin{align*}
              \Phi
    \colon    (\End_k(V))^{\otimes d}
    &\to      \End_k\left( V^{\otimes d} \right), \\
              f_1 \otimes \dotsb \otimes f_d
    &\mapsto  f_1 \otimes \dotsb \otimes f_d \,.
  \end{align*}
  Now both $\End_k(V)^{\otimes d}$ and $\End_k\left( V^{\otimes d} \right)$ are representations of $S_d$ in the usual way, i.e.\
  \[
      \pi.(f_1 \otimes \dotsb \otimes f_d)
    = f_{\pi(1)} \otimes \dotsb \otimes f_{\pi(d)}
  \]
  for all permutations $\pi \in S_d$ and simple tensors $f_1 \otimes \dotsb \otimes f_d \in \End_k(V)^{\otimes d}$, and
  \[
      (\pi.f)(x)
    = \pi.f\left( \pi^{-1}.x \right)
  \]
  for all permutations $\pi \in S_d$, $f \in \End_k(V^{\otimes d})$ and $x \in V^{\otimes d}$.
  $\Phi$ is also $G$-equivarint, since for every permutation $\pi \in S_d$ and simple tensors $f_1 \otimes \dotsb \otimes f_d \in \End_k(V)^{\otimes d}$, $v_1 \otimes \dotsb \otimes v_d \in V^{\otimes_d}$
  \begin{align*}
     &\,  \Phi(\pi.(f_1 \otimes \dotsb \otimes f_d))(v_1 \otimes \dotsb \otimes v_d) \\
    =&\,  \Phi\left( f_{\pi(1)} \otimes \dotsb \otimes f_{\pi(d)} \right)(v_1 \otimes \dotsb \otimes v_d) \\
    =&\,  f_{\pi(1)}(v_1) \otimes \dotsb \otimes f_{\pi(d)}(v_d)
  \shortintertext{and}
     &\,  (\pi.\Phi(f_1 \otimes \dotsb \otimes f_d))(v_1 \otimes \dotsb \otimes v_d) \\
    =&\,  \pi.\Phi(f_1 \otimes \dotsb \otimes f_d)\left( \pi^{-1}.(v_1 \otimes \dotsb \otimes v_d) \right) \\
    =&\,  \pi.\Phi(f_1 \otimes \dotsb \otimes f_d)\left( v_{\pi^{-1}(1)} \otimes \dotsb \otimes v_{\pi^{-1}(d)} \right) \\
    =&\,  \pi.\left( f_1\left(v_{\pi^{-1}(1)}\right) \otimes \dotsb \otimes f_d\left(v_{\pi^{-1}(d)}\right) \right) \\
    =&\,  f_{\pi(1)}(v_1) \otimes \dotsb \otimes f_{\pi(d)}(v_d)0\,.
  \end{align*}
  So $\Phi$ is an isomorphism of representations of $S_d$. It follows that $\Phi$ induces an isomorphism
  \[
          \left( \End_k(V)^{\otimes d} \right)^{S_d}
    \cong \End_k \left(V^{\otimes d}\right)^{S_d}.
  \]
  Hence
  \begin{align*}
          \End_{S_n}\left( V^{\otimes d} \right)
    &=    \left( \End_k\left( V^{\otimes d} \right) \right)^{S_n}
    \cong \left( \End_k(V)^{\otimes d} \right)^{S_n} \\
    &=    \text{symmetric tensors in $\End_k(V)^{\otimes d}$} \,.
  \end{align*}
  
  Now $\gen{\GL(V)} \subseteq \End_k(V^{\otimes d})$ is generated as an $k$-vector space by the image of the group homomorphism $\GL(V) \to \GL(V^{\otimes d})$.
  (Since $\GL(V)$ is a $k$-basis of $k\GL(V)$, the image of $\GL(V)$ under the algebra homomorphism $k \GL(V) \to \End_k(V)$ generated $\gen{\GL(V)}$ as a $k$-vector space.
  The image of $\GL(V)$ under this algebra homomorphism is precisely the image of $\GL(V)$ under the group homomorphism.)
  Since the image of $\psi \in \GL(V)$ under this group homomorphism is given by $\psi \otimes \dotsb \otimes \psi$ we need to show that
  \[
      \left( \End_k(V)^{\otimes d} \right)^{S_d}
    = \vspan_k  \{
                  \psi \otimes \dotsb \otimes \psi
                \mid
                  \psi \in \GL(V)
                \} \,.
  \]
  Since $\GL_k(V) \subseteq \End_k(V)$ is Zariski dense over $k$ this will follow from Lemma \ref{lemma: symmetric tensors and zariski dense subsets}
\end{proof}


\begin{lemma}\label{lemma: symmetric tensors and zariski dense subsets}
  Let $k$ be an infinite field, $d \geq 1$, $E$ a finite-dimensional $k$-vector space and $X \subseteq E$ Zariski-dense over $k$.
  Then the symmetric tensors in $E^{\otimes d}$ are generated as a $k$-vector space by the tensors $x \otimes \dotsb \otimes x$ where $x \in X$.
\end{lemma}
\begin{proof}
  Let $e_1, \dotsc, e_n$ be a $k$-basis of $E$, $S \subseteq E^{\otimes d}$ the vector subspace of symmetric tensors (i.e.\ $S = (E^{\otimes d})^{S_n}$) and
  \[
              U
    \coloneqq \vspan_k \{ x \otimes \dotsb \otimes x \mid x \in X \}
    \subseteq E^{\otimes d} \,.
  \]
  For every partition $\lambda \in \Natural^n$ with $|\lambda| = d$ we write
  \[
      e^\lambda
    =         \underbrace{e_1 \otimes \dotsb \otimes e_1}_{\lambda_1}
      \otimes \underbrace{e_2 \otimes \dotsb \otimes e_2}_{\lambda_2}
      \otimes \dotsb
      \otimes \underbrace{e_n \otimes \dotsb \otimes e_n}_{\lambda_n} \,.
  \]
  as well as
  \[
      a^\lambda
    = \sum_{y \in S_d e^\lambda} y
    = \sum \text{distinct permutations of $e^\lambda$} \,,
  \]
  where $S_d e^\lambda$ denotes the orbit of $e^\lambda$.
  It is easy to see that $\{ a^\lambda \mid \lambda \in \Natural^n, |\lambda| = d \}$ is a $k$-basis of $S$.
  It is clear that $U \subset S$ and to show the other inclusion is suffices to show that every $k$-linear map $f \colon S \to k$ which vanishes on $U$ is the zero map.
  
  To show this let $p \in k[X_1, \dotsc, X_d]$ be defined as
  \[
      p(X_1, \dotsc, X_d)
    = \sum_{\substack{\lambda \in \Natural^n \\ |\lambda| = d}}
        f\left( a^\lambda \right)
        X_1^{\lambda_1} \dotsm X_n^{\lambda_n}
  \]
  and $\tilde{\lambda} \colon E \to k$ as the corresponding polynomial function
  \[
      \tilde{\lambda}\left( \sum_{i=1}^n \mu_i e_i \right)
    = p(\mu_1, \dotsc, \mu_n) \,.
  \]
  It is clear that $\tilde{\lambda} \in \Pol_k(E)$.
  For every $x = \sum_{i=1}^n x_i e_i \in X$ we have
  \[
      \tilde{\lambda}(x)
    = p(x_1, \dotsc, x_n)
    = \sum_{\substack{\lambda \in \Natural^n \\ |\lambda| = d}}
        x_1^{\lambda_1} \dotsm x_n^{\lambda_n}
        f\left( a^\lambda \right) 
    = f(x \otimes \dotsb \otimes x)
    = 0 \,.
  \]
  Since $X$ is Zariski dense in $E$ we find that $\tilde{\lambda} = 0$.
  Therefore $p = 0$ and thus $f(a^\lambda) = 0$ for every $\lambda \in \Natural^n$ with $|\lambda| = d$.
  So $f_{|S} = 0$.
\end{proof}


\begin{corollary}
  Let $k$ be a field with $\kchar k = 0$ (in particular $k$ is infinite) and $V$ a finite-dimensional $k$-vector space.
  Then $V^{\otimes d}$ is a representation of $S_d \times \GL(V)$ and we have a decomposition into irreducible representations of $S_d \times \GL(V)$ with
  \[
          V^{\otimes d}
    \cong \bigoplus_{\lambda \in \Delta} S_\lambda \otimes V_\lambda
  \]
  where the $S_\lambda$ are irreducible representations of $S_d$, the $V_\lambda$ are irreducible representations of $\GL(V)$ and the $S_\lambda \otimes V_\lambda$ are irreducible representations of $S_d \times \GL(V)$, where $\Delta$ is a complete set of representatives of $\Irr(S_d)$.
\end{corollary}
\begin{proof}
  This follows directly from the Schur--Weyl Duality and the Double Centralizer Theorem.
\end{proof}

(The last few lectures, in which some basic facts about the irreducible representations of the symmetric group $S_n$ were presented without (much) proof, are missing from these notes.)
