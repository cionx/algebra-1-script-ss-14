\section{Some results on the number of generators of \texorpdfstring{$\mc{P}(V)^G$}{P(V)\^G}}


\begin{lemma}\label{lemma: technical lemma about generating}
  Let $A = \bigoplus_{d \geq 0} A_d$ be a graded $k$-algebra which is commutative.
  If the ideal $A^+ = \bigoplus_{d > 0} A_d$ is finitely generated over $A$ then $A$ is finitely generated as an $A_0$-algebra by homogeneous elements.
  (By this we mean that there are finitely many homogeneous elements $a_1, \dotsc, a_n \in A^+$ such that $A = A_0[a_1, \dotsc, a_n]$.)
\end{lemma}


% TODO: Figure out a proof.


\begin{lemma}\label{lemma: projection reynold operator}
  Let $A$ be a commutative $k$-algebra and let $G$ be a group acting on $A$ by algebra automorphisms.
  In particular, $A$ is a representation of $G$.
  Assume that $A = \bigoplus_{i \in I} L_i$ where $L_i \subseteq A$ is an irreducible subrepresentation of $G$ for all $i \in I$.
  Let
  \[
              J
    \coloneqq \{
                i \in I
              \mid
                      L_i
                \cong k
                \text{ as representations of $G$}
              \}.
  \]
  \begin{enumerate}
    \item
      We have that
      \[
          A^G
        = \bigoplus_{j \in J} L_j \,.
      \]
    \item
      We have that $A = A^G \oplus N$ where $N = \bigoplus_{i \in I \smallsetminus J} L_i$.
      For the projection $\pi \colon A \to A^G$ along this decomposition we have that
      \[
          \pi(hf)
        = h\pi(f) \,.
      \]
      for all $h \in A^G$, $f \in A$.
  \end{enumerate}
\end{lemma}


% TODO: Figure out a proof.


\begin{remark}
  REYNOLD OPERATOR
\end{remark}


\begin{theorem}(Hilbert)
  Let $G$ be a group and $W$ a finite-dimensional representation of $G$ over $k$.
  Then $\mc{P}(W)$ is a representation of $G$ in the usual way (i.e.\ $(g.f)(w) = f\left( g^{-1}.w \right)$ for all $g \in G$, $f \in \mc{P}(W)$ and $w \in W$).
  Suppose that
  \[
      \mc{P}(W)
    = \bigoplus_{i \in I} L_i
  \]
  where $L_i \subseteq \mc{P}(W)$ is an irreducible representation of $G$ for all $i \in I$.
  Then $\mc{P}(W)^G$ is finitely generated as a $k$-algebra.
\end{theorem}


\begin{proof}
  We write $A \coloneqq \mc{P}(W)$. By Lemma \ref{lemma: projection reynold operator} we can decompose
  \[
      A
    = A^G \oplus N
  \]
  as representations of $G$, and the projection
  \[
            \pi
    \colon  A
    \to     A^G
  \]
  is $G$-equivariant with
  \[
     \pi(h)
    = h
    \text{ for all }
    h \in A^G
  \]
  and
  \[
      \pi(hf)
    = h\pi(f)
    \text{ for all }
    h \in A^G,
    f \in A \,.
  \]
  
  It is clear that $A^G \subseteq A$ is a $k$-subalgebra.
  Let $I \subseteq A^G$ be an ideal. Then
  \begin{equation}\label{eqn: reynold ideal}
      \pi(A I)
    = \pi(I A)
    = I \pi(A)
    = I A^G
    = I \,.
  \end{equation}
  Now consider the special case $I = \mf{m}_0 \coloneqq A^G_+$ where
  \[
              A^G_+
    \coloneqq \bigoplus_{d > 0} A^G_d \,.
  \]
  To see that $A^G_+$ is an $A^G$-ideal notice that $A_+ \coloneqq \bigoplus_{i > 0} A_d$ is an $A$-ideal and that $A^G = \bigoplus_{d \geq 0} A^G_d$ and therefore
  \[
              A^G A^G_+
    =         A^G \left( A^G \cap A_+ \right)
    \subseteq A^G \cap A^G A_+
    \subseteq A^G \cap A A_+
    \subseteq A^G \cap A^+
    =         A^G_+ \,.
  \]
  
  Let $A \mf{m}_0$ be the ideal in $A$ generated by $\mf{m}_0$.
  Because $A = \mc{P}(W)$ is noetherian there exists $f_1, \dotsc, f_n \in \mf{m}_0$ with
  \[
      A \mf{m}_0
    = (f_1, \dotsc, f_n)_A \,.
  \]
  By \eqref{eqn: reynold ideal} we find that $\mf{m}_0$ is generated by $f_1, \dotsc, f_s$ as in $A^G$-module because
  \begin{align*}
      \mf{m}_0
    = \pi(A \mf{m_0})
    = \pi( (f_1, \dotsc, f_s)_A )
    = \left( f_1, \dotsc, f_s \right)_{A^G} \,.
  \end{align*}
  Since $\mf{m}_0$ is finitely generated as an $A$-module we find by by Lemma \ref{lemma: technical lemma about generating} that $A$ is finitely generated as a $A_0^G$-algebra, where $A_0^G = A_0 = k$.
\end{proof}
