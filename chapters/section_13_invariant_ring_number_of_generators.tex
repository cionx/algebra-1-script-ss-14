\section{Finiteness Results on Invariant Rings}


\begin{fluff}
  We now return to the topic of invariant rings.
  Throughout the previous sections we have determined the following examples:
  \begin{center}
    \begingroup
    \renewcommand{\arraystretch}{2}
    \setlength{\tabcolsep}{3pt}
    \begin{tabular}{|ccccc|}
      \hline
        \textbf{Group}
      & \begingroup
        \renewcommand{\arraystretch}{1}
        \begin{tabular}{c}
          \textbf{Repre-}  \\
          \textbf{sentation}
        \end{tabular}
        \endgroup
      & \textbf{Action}
      & \begingroup
        \renewcommand{\arraystretch}{1}
        \begin{tabular}{c}
          \textbf{Invariant Ring}  \\
          \textbf{(up to iso.)}
        \end{tabular}
        \endgroup
      & \textbf{generators}
      \\
      \hline
        $S_n$
      & $k^n$
      & \begingroup
        \renewcommand{\arraystretch}{1}
        \begin{tabular}{c}
          permutation \\
          of coordinates
        \end{tabular}
        \endgroup
      & $k[X_1, \dotsc, X_n]$
      & \begingroup
        \renewcommand{\arraystretch}{1}
        \begin{tabular}{c}
          $e_1, \dotsc, e_n$, \\
          $h_1, \dotsc, h_n$, \\
          $p_1, \dotsc, p_n$\phantom{,} \\
          (for suitable $k$)
        \end{tabular}
        \endgroup
      \\
      \hline
        $\GL_n(k)$
      & \multirow{2}{*}{
        $\Mat_n(k)$
        }
      & \multirow{2}{*}{
        conjugation
        }
      & \multirow{2}{*}{
        $k[X_1, \dotsc, X_n]$
        }
      & \multirow{2}{*}{
        \begingroup
        \renewcommand{\arraystretch}{1}
        \begin{tabular}{c}
          $s_1, \dotsc, s_n$,     \\
          $\tr_1, \dotsc, \tr_n$  \\
          (for suitable $k$)
        \end{tabular}
        \endgroup
        }
      \\
        $\SL_n(k)$
      & {}
      & {}
      & {}
      & {}
      \\
      \hline
        $\GL_n(k)$
      & \multirow{2}{*}{
        $\Mat_n(k)$
        }
      & \multirow{2}{*}{
          (left) mult.\
        }
      & $k$
      & $\emptyset$
      \\
        $\SL_n(k)$
      & {}
      & {}
      & $k[X]$
      & $\det$
      \\
      \hline
    \end{tabular}
    \endgroup
  \end{center}
  We finish this chapter by giving two finiteness results on the invariant ring $\mc{P}(V)^G$, one by Hilbert and one by E.\ Noether.
  The main results of this section are taken from \cite[\S 1.6]{InvariantPrimer} and generalize some aspects of the above examples.
\end{fluff}


\begin{conventions}
  In the following $k$ denotes an infinite field.
\end{conventions}


\begin{fluff}
  Let $V$ be finite-dimensional representation of a group $G$.
  The main observation behind both theorems is that the invariant ring $\mc{P}(V)^G$ inherits a grading from $\mc{P}(V)$:
  
  If $f \in \mc{P}(V)$ is homogenous of degree $d \geq 0$, then for every $g \in G$ the polynomial map $g.f \in \mc{P}(V)$ is again polynomial of degree $d$ because
  \[
      (g.f)(\lambda v)
    = f(g^{-1}.(\lambda v))
    = f(\lambda g^{-1}.v)
    = \lambda^d f(g^{-1}.v)
    = \lambda^d (g.f)(v)
  \]
  for all $\lambda \in k$, $v \in V$.
  It follows for the grading $\mc{P}(V) = \bigoplus_{d \geq 0} \mc{P}(V)_d$ that $\mc{P}(V)_d$ is a subrepresentation for every $d \geq 0$.
  From this it follows that
  \[
      \mc{P}(V)^G
    = \left( \bigoplus_{d \geq 0} \mc{P}(V)_d \right)^G
    = \bigoplus_{d \geq 0} \mc{P}(V)^G_d \,.
  \]
  with $\mc{P}(V)^G_d = \mc{P}(V)^G \cap \mc{P}(V)_d$.
  That $\mc{P}(V)^G_d \mc{P}(V)^G_{d'} \subseteq \mc{P}(V)^G_{d+d'}$ is a combination of
  \[
    \mc{P}(V)_d \mc{P}(V)_{d'} \subseteq \mc{P}(V)_{d+d'}
    \quad\text{and}\quad
    \mc{P}(V)^G \mc{P}(V)^G \subseteq \mc{P}(V)^G \,.
  \]
\end{fluff}





\input{chapters/subsection_13.1_hilberts_theorem}
\input{chapters/subsection_13.2_noethers_theorem}




