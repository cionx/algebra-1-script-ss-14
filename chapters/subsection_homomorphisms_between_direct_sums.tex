\subsection{Homomorphims Between Direct Sums}
\label{appendix: homomorphisms between direct sums}


\begin{conventions}
  We denote by $R$ a ring and abbreviate $\Hom_R$ as $\Hom$.
  In the following, $M_1, \dotsc, M_t, M, N_1, \dotsc, N_t, N$ are $R$-modules.
\end{conventions}


\begin{fluff}
  In this section we will explain how $R$-module homomorphisms between finite direct sums can be represented  by matrices.
  We keep our treatment elementary and will not (explicitely) use the categorical notions of coproducts, products or biproducts.
  We encourage the reader who is familiar with these notions to generalize the contents of this subsection to additive categorie
\end{fluff}


\begin{fluff}
  Let
  \[
          \pi_i
  \colon  N_1 \oplus \dotsb \oplus N_t
  \to     N_i
  \]
  denotes the projection onto the $i$-th summand for $i = 1, \dotsc, s$, and let
  \[
            \iota_j
    \colon  M_j
    \to     M_1 \oplus \dotsb \oplus M_t
  \]
  denote the inclusion of the $j$-th summand for every $j = 1, \dotsc t$.
\end{fluff}


\begin{definition}
  For every homomorphism of $R$-modules
  \[
            f
    \colon  M_1 \oplus \dotsb \oplus M_s
    \to     N_1 \oplus \dotsb \oplus N_t
  \]
  its \emph{$(ij)$-th component} is given by
  \[
              [f]_{ij}
    \defined  \pi_i \circ f \circ \iota_j
  \]
  for all $i = 1, \dotsc, s$, $j = 1, \dotsc, t$, and we set
  \[
              [f]
    \defined  \begin{bmatrix}
                [f]_{11}  & \cdots  & [f]_{1t}  \\
                \vdots    & \ddots  & \vdots    \\
                [f]_{s1}  & \cdots  & [f]_{st}
              \end{bmatrix}
    \in       \begin{bmatrix}
                \Hom(M_1, N_1)  & \cdots  & \Hom(M_t, N_1)  \\
                \vdots          & \ddots  & \vdots          \\
                \Hom(M_1, N_s)  & \cdots  & \Hom(M_t, N_s)
              \end{bmatrix}.
  \]
\end{definition}


\begin{theorem}
  \label{theorem: bijection homomorphisms matrices}
  The map
  \begin{align*}
    \Hom(M_1 \oplus \dotsb \oplus M_t, N_1 \oplus \dotsb \oplus N_s)
    &\longto
    \begin{bmatrix}
      \Hom(M_1, N_1)  & \cdots  & \Hom(M_t, N_1)  \\
      \vdots          & \ddots  & \vdots          \\
      \Hom(M_1, N_s)  & \cdots  & \Hom(M_t, N_s)
    \end{bmatrix},
    \\
    f
    &\longmapsto
    [f]
  \end{align*}
  is an isomorphism of abelian groups, and an isomorphism of $k$-vector spaces if $R$ is a $k$-algebra.
\end{theorem}


\begin{proof}
  A homomorphisn $f \colon M_1 \oplus \dotsb \oplus M_t \to N_1 \oplus \dotsb \oplus N_s$ is uniquely determined by the colllection of its restrictions $f \circ \iota_j \colon M_j \to N_1 \oplus \dotsb \oplus N_s$, which is turns it uniquely determined by the collections of its components $\pi_i \circ f \circ \iota_j \colon M_j \to N_i$.
  This shows that $[\,\cdot\,]$ is injective.
  
  For all $j = 1, \dotsc, t$, $i = 1, \dotsc, s$ we have that
  \begin{align*}
        [f+g]_{ij}
    &=  \pi_i \circ (f+g) \circ \iota_j
     =  \pi_i \circ ((f \circ \iota_j) + (g \circ \iota_j)) \\
    &=  (\pi_i \circ f \circ \iota_j) + (\pi_i \circ g \circ \iota_j)
     =  [f]_{ij} + [g]_{ij} \,,
  \end{align*}
  which shows that $[\,\cdot\,]_{ij}$ is additive.
  It follows that $[\,\cdot\,]$ additive.
  If $R$ is a $k$-algebra then the components $[\,\cdot\,]_{ij} = \pi_i \circ (\,\cdot\,) \circ \iota_j$ are already $k$-linear for all $i, j$, and it follows that $[\,\cdot\,]$ is $k$-linear.
  
  To show that $[\,\cdot\,]$ is surjective let $\pi'_j \colon M_1 \oplus \dotsb \oplus M_t \to M_j$ be the projection onto the $j$-th summand for every $j = 1, \dotsc, t$, and let $\iota_i \colon N_i \to N_1 \oplus \dotsb \oplus N_s$ be the inclusion of the $i$-th summand.
  For a collection $(f_{ij})^{j = 1, \dotsc, t}_{i = 1, \dotsc, s}$ of homomorphisms $f_{ij} \colon M_j \to N_i$ we then have the homomorphism $f \colon M \to N$ given by
  \[
      f
    = \sum_{\substack{j' = 1, \dotsc, t \\ i' = 1, \dotsc, s}} (\iota_{i'} \circ f_{i'j'} \circ \pi'_{j'}) \,,
  \]
  whose $(ij)$-th component is given by
  \begin{align*}
        [f]_{ij}
     =  \pi_i \circ f \circ \iota_j
    &=  \pi_i
        \circ
        \left(
          \sum_{\substack{j' = 1, \dotsc, t \\ i' = 1, \dotsc, s}} (\iota_{i'} \circ f_{i'j'} \circ \pi'_{j'})
        \right)
        \circ
        \iota_j \\
    &=  \sum_{\substack{j' = 1, \dotsc, t \\ i' = 1, \dotsc, s}}
        \big(
        \pi_i \circ \iota_{i'} \circ f_{i'j'} \circ \pi'_{j'} \circ \iota_j
        \big) \\
    &=  \sum_{\substack{j' = 1, \dotsc, t \\ i' = 1, \dotsc, s}}
        \big(
        (\delta_{i,i'} \id) \circ f_{i'j'} \circ (\delta_{j,j'} \id)
        \big)
     =  f_{ij} \,.
  \end{align*}
  This shows that $[\,\cdot\,]$ is surjective.
\end{proof}


\begin{fluff}
  As a consequence of Theorem~\ref{theorem: bijection homomorphisms matrices} we can represent every homomorphism between finite direct sums of $R$-modules as a matrix.
  We want to highlight the following special cases:
\end{fluff}


\begin{corollary}
  \label{corollary: Hom on direct sums}
  The maps
  \begin{align*}
              \Hom(M, N_1 \oplus \dotsb \oplus N_s)
    &\longto  \Hom(M, N_1) \times \dotsb \times \Hom(M, N_s), \\
    \quad         f
    &\longmapsto  (\pi_1 \circ f, \dotsc, \pi_s \circ f)
  \intertext{and}
              \Hom(M_1 \oplus \dotsb \oplus M_t, N)
    &\longto  \Hom(M_1, N) \times \dotsb \times \Hom(M_t, N), \\
    \quad         f
    &\longmapsto  (f \circ \iota_1, \dotsc, f \circ \iota_t)
  \end{align*}
  are isomorphism of abelian groups, and isomorphism of $k$-vector spaces if $R$ is a $k$-algebra.
\end{corollary}


\begin{remark}
  The second isomorphism of Corollary~\ref{corollary: Hom on direct sums} holds for arbitrary direct sums, while the first isomorphism holds for arbitrary \emph{products} instead of finite direct sums.
\end{remark}



\begin{fluff}
  We may write the elements of $M_1 \oplus \dotsb \oplus M_t$ as column vectors
  \[
    \vect{m_1 \\ \vdots \\ m_t}
  \]
  with $m_j \in M_j$ for every $j = 1, \dotsc, t$;
  the elements of $N_1 \oplus \dotsb \oplus N_s$ can be represented as column vectors in the same way.
  For every $R$-module homomorphism $f \colon M \to N$ with
  \[
      f
    = \begin{bmatrix}
        f_{11}  & \cdots  & f_{1t}  \\
        \vdots  & \ddots  & \vdots  \\
        f_{s1}  & \cdots  & f_{st}
      \end{bmatrix}
  \]
  we then have that
  \begin{align*}
        f(m)
     =  \vect{ \pi_1(f(m)) \\ \vdots \\ \pi_s(f(m)) }
    &=  \vect{
          \pi_1( f( \iota_1(m_1) + \dotsb + \iota_t(m_t) ) )
          \\
          \vdots
          \\
          \pi_s( f( \iota_1(m_1) + \dotsb + \iota_t(m_t) ) )
        }
    \\
    &=  \vect{
          \pi_1(f(\iota_1(m_1))) + \dotsb + \pi_1(f(\iota_t(m_t)))
          \\
          \vdots
          \\
          \pi_s(f(\iota_1(m_1))) + \dotsb + \pi_s(f(\iota_t(m_t)))
        }
    \\
    &=  \vect{
          f_{11}(m_1) + \dotsb + f_{1t}(m_t)
          \\
          \vdots
          \\
          f_{s1}(m_1) + \dotsb + f_{st}(m_t)
        }
    = \begin{bmatrix}
        f_{11}  & \cdots  & f_{1t}  \\
        \vdots  & \ddots  & \vdots  \\
        f_{s1}  & \cdots  & f_{st}
      \end{bmatrix}
      \vect{m_1 \\ \vdots \\ m_t} \,,
  \end{align*}
  % TODO: Fix the spacing of the vertical dots in the vectors.
  where the matrix-vector product in the last step is taken in the naive sense.
  
  This shows that when representing a homomorphism by a matrix, we can represent application of this homomorphism by matrix-vector multiplication.
\end{fluff}


\begin{example}
  Let $k$ be a field and let $V, W$ be finite-dimensional $k$-vector spaces.
  A choice of a basis $b_1, \dotsc, b_t$ of $V$ is then the same as an isomorphism $\varphi \colon V \to k^t$, and a choice of a basis $c_1, \dotsc, c_s$ of $W$ is the same as an isomorphism $\psi \colon W \to k^s$.
  We also have that $k \xrightarrow{\sim} \Hom(k,k)$ where for $\lambda \in k$ the linear map $k \to k$ is given by $x \mapsto \lambda x$.
  Alltogether this results in an isomorphism
  \begin{align*}
      \Hom(V,W)
    \xlongrightarrow{\sim}
      \Hom(k^s, k^t)
    \longto
      \begin{bmatrix}
        \Hom(k,k) & \cdots  & \Hom(k,k) \\
        \vdots    & \ddots  & \vdots    \\
        \Hom(k,k) & \cdots  & \Hom(k,k)
      \end{bmatrix}
    \xlongrightarrow{\sim}&
      \begin{bmatrix}
        k       & \cdots  & k       \\
        \vdots  & \ddots  & \vdots  \\
        k       & \cdots  & k
      \end{bmatrix}
    \\
    =&\,
      \Mat(t \times s, k)
  \end{align*}
  which associates to $f \in \Hom(V, W)$ its representing matrix with respect to the bases $b_1, \dotsc, b_t$ and $c_1, \dotsc, c_s$ in the usual way.
\end{example}


\begin{proposition}
  \label{proposition: representing matrix is multiplicative}
  Let
  \[
    M_1 \oplus \dotsb \oplus M_t
    \xlongrightarrow{f}
    N_1 \oplus \dotsb \oplus N_s
    \xlongrightarrow{g}
    P_1 \oplus \dotsb \oplus P_r
  \]
  be $R$-module homomorphisms.
  Then
  \[
      [g \circ f]
    = [g] \cdot [f]
  \]
  where the matrix multiplication on the right hand side is taken in the naive sense.
\end{proposition}


\begin{proof}
  We denote the various projections and inclusions by
  \begin{align*}
                              M_j
    \xlongrightarrow{\iota_j} M_1 \oplus &\dotsb \oplus M_t \,,
    \\
                                N_i
    \xlongrightarrow{\iota'_i}  N_1 \oplus &\dotsb \oplus N_s
    \xlongrightarrow{\pi'_i}    N_i \,,
    \\
                              P_1 \oplus &\dotsb \oplus P_r
    \xlongrightarrow{\pi''_r} P_r \,.
  \end{align*}
  For all $j = 1, \dotsc, t$, $k = 1, \dotsc, r$ we then have that
  \begin{align*}
        (g \circ f)_{kj}
    &=  \pi_k \circ g \circ f \circ \iota_j
     =  \pi_k \circ g \circ \id_N \circ f \circ \iota_j
     =  \pi_k \circ g \circ \left( \sum_{i=1}^s \iota'_i \circ \pi'_i \right) \circ f \circ \iota_j \\
    &=  \sum_{i=1}^s ( \pi_k \circ g \circ \iota'_i \circ \pi'_i \circ f \circ \iota_j )
     =  \sum_{i=1}^s ( g_{ki} \circ f_{ij} ) \,,
  \end{align*}
  which is precisely the $(kj)$-th entry of $[g] \cdot [f]$.
\end{proof}


\begin{corollary}
  \label{corollary: endomorphisms of direct sum}
  The map
  \begin{align*}
              \End_R(M_1 \oplus \dotsb \oplus M_t)
    &\longto  \begin{bmatrix}
                \Hom(M_1, M_1)  & \cdots  & \Hom(M_t, M_1)  \\
                \vdots          & \ddots  & \vdots          \\
                \Hom(M_1, M_t)  & \cdots  & \Hom(M_t, M_t)
              \end{bmatrix}
    \\
                  f
    &\longmapsto  [f]
  \end{align*}
  is an isomorphism of rings, and an isomorphism of $k$-algebras if $R$ is a $k$-algebra.
\end{corollary}


\begin{example}
  We determine the automorphisms of the $\Integer$-module $\Integer \oplus (\Integer/3)$:
  If $f \colon \Integer \oplus (\Integer/3) \to \Integer \oplus (\Integer/3)$ is an endomorphism then
  \[
      f
    = \begin{bmatrix}
        f_{11}  & f_{12}  \\
        f_{21}  & f_{22}
      \end{bmatrix}
  \]
  for homomorphisms
  \[
    \begin{array}{ll}
      f_{11}  \colon  \Integer    \to \Integer \,,
      &
      f_{12}  \colon  \Integer/3  \to \Integer \,,
      \\
      f_{21}  \colon  \Integer    \to \Integer/3 \,,
      &
      f_{22}  \colon  \Integer/3  \to \Integer/3 \,.
    \end{array}
  \]
  There exists no nonzero homomorphism $\Integer/3 \to \Integer$ so we have that
  \[
      f
    = \begin{bmatrix}
        f_{11}  & 0       \\
        f_{21}  & f_{22}
      \end{bmatrix}.
  \]
  For $f, g \colon \Integer \oplus (\Integer/3) \to \Integer \oplus (\Integer/3)$ we have that
  \[
      fg
    = \begin{bmatrix}
        f_{11}  & 0       \\
        f_{21}  & f_{22}
      \end{bmatrix}
      \begin{bmatrix}
        g_{11}  & 0       \\
        g_{21}  & g_{22}
      \end{bmatrix}
    = \begin{bmatrix}
        f_{11} g_{11}                 & 0             \\
        f_{21} g_{11} + f_{22} g_{21} & f_{22} g_{22}
      \end{bmatrix}.
  \]
  It follows that $f$ is an automorphism (with inverse $g$) if and only if both $f_{11}, f_{22}$ are automorphisms (with inverses $g_{11}, g_{22}$).
  There exist two automorphisms $\Integer \to \Integer$, two automorphisms $\Integer/3 \to \Integer/3$, and three homomorphisms $\Integer \to \Integer/3$.
  It follows that $\Integer \oplus (\Integer/3)$ has
  \[
      2 \cdot 2 \cdot 3
    = 12
  \]
  automorphisms, as described above.
\end{example}


\begin{corollary}
  \label{corollary: decomposition of endomorphisms for orthogonal modules}
  \leavevmode
  \begin{enumerate}
    \item
      For $n \geq 0$ the map
      \begin{align*}
                  \End(M^{\oplus n})
        &\longto  \begin{bmatrix}
                    \End(M) & \cdots  & \End(M) \\
                    \vdots  & \ddots  & \vdots  \\
                    \End(M) & \cdots  & \End(M)
                  \end{bmatrix}
        =        \Mat_n( \End(M) )
        \\
                      f
        &\longmapsto  [f]
      \end{align*}
      is an isomorphism of rings, and an isomorphism of $k$-algebras if $R$ is a $k$-algebra.
    \item
      If more generally $M_1, \dotsc, M_t$ there exist no nonzero homomorphism $M_i \to M_j$ for all $i, j$ then for $n_1, \dotsc, n_t \geq 0$ the map
      \begin{align*}
                  \End(M_1^{\oplus n_1} \oplus \dotsb \oplus M_t^{\oplus n_t})
        &\longto  \begin{bmatrix}
                      \Mat_{n_1}( \End(M_1) )
                    & {}
                    & {}
                    \\
                      {}
                    & \ddots
                    & {}
                    \\
                      {}
                    & {}
                    & \Mat_{n_t}( \End(M_t) )
                  \end{bmatrix}
        \\
                      f
        &\longmapsto  [f]
      \end{align*}
      is a well-defined isomorphism of rings, and an isomorphism of $k$-algebras if $R$ is a $k$-algebra.
  \end{enumerate}
\end{corollary}





\subsubsection{Compatibility with Module Structures}


\begin{fluff}
  The abelian group $\Hom(M_1 \oplus \dotsb \oplus M_t, N_1 \oplus \dotsb \oplus N_s)$ carries the structure of a left $\End(N_1 \oplus \dotsb \oplus N_s)$-module via postcomposition
  \[
      g \cdot f
    = g \circ f
  \]
  for all $g \in \End(N_1 \oplus \dotsb \oplus N_s)$, $f \in \Hom(M_1 \oplus \dotsb \oplus M_t, N_1 \oplus \dotsb \oplus N_s)$.
  Similarly, the abelian group
  \begin{equation}
    \label{equation: homomorphism matrices}
    \begin{bmatrix}
      \Hom(M_1, N_1)  & \cdots  & \Hom(M_t, N_1)  \\
      \vdots          & \ddots  & \vdots          \\
      \Hom(M_1, N_s)  & \cdots  & \Hom(M_t, N_s)
    \end{bmatrix}
  \end{equation}
  carries the structure of a left $E$-module for
  \[
              E
    \defined  \begin{bmatrix}
                \Hom(N_1, N_1)  & \cdots  & \Hom(N_s, N_1)  \\
                \vdots          & \ddots  & \vdots          \\
                \Hom(N_1, N_s)  & \cdots  & \Hom(N_s, N_s)
              \end{bmatrix}
  \]
  via  matrix multiplication.
  It follows from Proposition~\ref{proposition: representing matrix is multiplicative} that the isomorphism of abelian groups from Theorem~\ref{theorem: bijection homomorphisms matrices} and the isomorphism of rings from Corollary~\ref{corollary: endomorphisms of direct sum} are compatible with this module structures.
  
  In the same way we can consider the right $\End(M_1 \oplus \dotsb \oplus M_t)$-module structure on $\Hom(M_1 \oplus \dotsb \oplus M_t, N_1 \oplus \dotsb \oplus N_s)$ given by precomposition, and the $E'$-module structure on \eqref{equation: homomorphism matrices} for
  \[
              E'
    \defined  \begin{bmatrix}
                \Hom(M_1, M_1)  & \cdots  & \Hom(M_t, M_1)  \\
                \vdots          & \ddots  & \vdots          \\
                \Hom(M_1, M_t)  & \cdots  & \Hom(M_t, M_t)
              \end{bmatrix}
  \]
  given by matrix multiplication, and find by Proposition~\ref{proposition: representing matrix is multiplicative} that the isomorphism of abelian groups from Theorem~\ref{theorem: bijection homomorphisms matrices} and the isomorphism of rings from Corollary~\ref{corollary: endomorphisms of direct sum} are compatible with this right module structures.
  
  Each left module structure is also compatible with the corresponding right module structure, resulting in compatibilities of bimodule structures.
  We thus arrive at the following results:
\end{fluff}


\begin{proposition}
  The isomorphism of abelian groups from Theorem~\ref{theorem: bijection homomorphisms matrices} and isomorphism of rings from Corollary~\ref{corollary: endomorphisms of direct sum} are compatible with the $\End(N_1 \oplus \dotsb \oplus N_s)$-$\End(M_1 \oplus \dotsb \oplus M_t)$-bimodule structure of $\Hom(M_1 \oplus \dotsb \oplus M_t, N_1 \oplus \dotsb \oplus N_s)$ which is given by postcomposition-precomposition and the $E$-$E'$-bimodule structure of
  \[
    \begin{bmatrix}
      \Hom(M_1, N_1)  & \cdots  & \Hom(M_t, N_1)  \\
      \vdots          & \ddots  & \vdots          \\
      \Hom(M_1, N_s)  & \cdots  & \Hom(M_t, N_s)
    \end{bmatrix}
  \]
  for the rings
  \begin{align*}
              E
    &\defined \begin{bmatrix}
                \Hom(N_1, N_1)  & \cdots  & \Hom(N_s, N_1)  \\
                \vdots          & \ddots  & \vdots          \\
                \Hom(N_1, N_s)  & \cdots  & \Hom(N_s, N_s)
              \end{bmatrix},
    \\
              E'
    &\defined \begin{bmatrix}
                \Hom(M_1, M_1)  & \cdots  & \Hom(M_t, M_1)  \\
                \vdots          & \ddots  & \vdots          \\
                \Hom(M_1, M_t)  & \cdots  & \Hom(M_t, M_t)
              \end{bmatrix}
  \end{align*}
  which is given by left-right matrix multiplication.
\end{proposition}




