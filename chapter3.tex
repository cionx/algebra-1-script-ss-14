\chapter{Invariants for matrix actions and Zariski dense sets}





\section{Invariants for matrix actions}
For this section we require all fields to be infinite.


\begin{thrm}
 Let $\SL_n(k)$ act on $\Mat_n(k)$ by left multiplication. Then
 \[
  \det : \Mat_n(k) \to k
 \]
 generates $\mc{P}(M_n(k))^{\SL_n(k)}$ as a $k$-algebra and is algebraically independent, i.e.
 \[
  \mc{P}(M_n(k))^{\SL_n(k)} \cong k[X] \text{ via } \det \mapsfrom X.
 \]
\end{thrm}
\begin{proof}
 We choose the basis $E_{ij}, 1 \leq i,j \leq n$ of $M_n(k)$ where the $E_{ij}$ is the matrix with $1$ as the entry in the $i$-th row and $j$-th colomn and $0$ otherwise. (So $E_{ij} e_k = \delta_{jk}e_i$ where $e_1, \ldots, e_n$ is the standard basis of $k^n$.)  For a matrix $A = (a_{ij})_{1 \leq i,j \leq n}$ we then have
 \[
  \det A
  = \det\left( a_{ij} E_{ij}\right)
  = \sum_{\sigma \in S_n} \sgn(\sigma) a_{1 \sigma(1)} \cdots a_{n \sigma(n)}.
 \]
 So $\det \in \mc{P}(\Mat_n(k))$. Since for every $S \in \SL_n(k)$ and $A \in \Mat_n(k)$
 \[
  (S.\det)(A) = \det\left(S^{-1}.A\right) = \det\left(S^{-1}\right) \det(A) = \det(A)
 \]
 we also find that $\det$ is $\SL_n(k)$-invariant. So $\det \in \mc{P}(\Mat_n(k))^{\SL_n(k)}$.
 
 Next we show that $\det$ is algebraically independent. Let $P \in k[X]$ with $P(\det) = 0$. Then we have
 \[
  P(\det(A)) = P(\det)(A) = 0
 \]
 for all $A \in \Mat_n(k)$. Since $\det$ is surjective this means that $P(\lambda) = 0$ for all $\lambda \in k$. Because $k$ is infinite this means that $P$ has infinitely many zeroes. Therefore $P = 0$.
 
 All that’s left to show is that $\det$ generates $\mc{P}(\Mat_n(k))^{\SL_n(k)}$ as a $k$-algebra. For this fix $f \in \mc{P}(\Mat_n(k))^{\SL_n(k)}$. Since $f$ is a polynomial map there exists $p \in k[X_{11}, \ldots, X_{nn}]$ such that
 \[
  f(A) = p(a_{11}, \ldots, a_{nn})
 \]
 for all $A = (a_{ij})_{1 \leq i,j \leq n} \in \Mat_n(k)$. We define $\bar{p} \in k[t]$ as $\varphi(p)$, where
 \[
  \varphi : k[X_{11}, \ldots, X_{nn}] \to k[t]
 \] 
 is the algebra homomorphism defined by 
 \[
  \varphi(X_{ij}) =
  \begin{cases}
   0 & \text{ if } i \neq j, \\
   1 & \text{ if } i = j \neq 1, \\
   t & \text{ if } i = j = 1.
  \end{cases}
 \]
 It is clear that
 \[
  \bar{p}(\lambda) :=
  f\left(
   \begin{pmatrix}
    \lambda &   &        &   \\
            & 1 &        &   \\
            &   & \ddots &   \\
            &   &        & 1 \\
   \end{pmatrix}
  \right)
 \]
 for all $\lambda \in k$.
 
 Now let $A \in \GL_n(k)$. Since $\det A \neq 0$ we have
 \[
  B :=
  \begin{pmatrix}
   \det A &   &        &   \\
          & 1 &        &   \\
          &   & \ddots &   \\
          &   &        & 1 \\
  \end{pmatrix}
  \in \GL_n(k).
 \]
 Since $\det A = \det B$ we also have $S := AB^{-1} \in \SL_n(k)$. Combining this we find that
 \[
  f(A) = f(SB) = \left(S^{-1}.f\right)(B) = f(B) = \bar{p}(\det A) = \bar{p}(\det)(A).
 \]
 Therefore we have $f - \bar{p}(\det) = 0$ when restricted to $\GL_n(k)$, where it is clear that $\bar{p}(\det) \in \mc{P}(\Mat_n(k))^{\SL_n}$.
 
 \begin{claim}[Zariski density property 1]
  Let $h \in \mc{P}(\Mat_n(k))$ such that $h_{|\GL_n(k)} = 0$. Then $h = 0$.
 \end{claim}
 
 From this claim and the previous observations it follows that $f - \bar{p}(\det) = 0$ and thus $f = \bar{p}(\det) \in k[\det]$. This shows that $\det$ generates $\mc{P}(\Mat_n(k))^{\SL_n(k)}$ as a $k$-algebra. (We will prove the claim later and end this proof here.)
\end{proof}


Recall from linear algebra that the characteristic polynomial of a matrix $A \in \Mat_n(k)$ is defined as
\[
 \chi_A(t) = \det(t E_n - A)
\]
where $E_n \in \Mat_n(k)$ is the identity matrix, and that this results in
\[
 \chi_A(t) = t^n - s_1(A) t^{n-1} + s_2(A) t^{n-2} + \ldots + (-1)^n s_n(A)
\]
where $s_1, \ldots, s_n \in \mc{P}(\Mat_n(k))$. In particular $s_1 = \tr$ and $s_n = \det$.

In the case that $A$ is a diagonal matrix $A = \diag(d_1, \ldots, d_n)$ we have
\[
 \chi_A(t)
 = \prod_{i=1}^n (t-d_i)
 = t^n - e_1(d_1, \ldots, d_n) t^{n-1} + \ldots + (-1)^n e_n(d_1, \ldots, d_n).
\]
Therefore
\[
 s_i(\diag(d_1, \ldots, d_n)) = e_i(\diag(d_1, \ldots, d_n))
\]
for all $1 \leq i \leq n$ and every diagonal matrix $\diag(d_1, \ldots, d_n)$.


\begin{thrm}
 Let $\GL_n(k)$ act on $\Mat_n(k)$ by conjugation, i.e.
 \[
  S.A = SAS^{-1}
 \]
 for all $A \in \Mat_n(k)$ and $S \in \GL_n(k)$. Then $s_1, \ldots, s_n$ generate $\mc{P}(\Mat_n(k))^{\GL_n(k)}$ as a $k$-algebra and are algebraically independent. Therefore
 \[
  \mc{P}(\Mat_n(k))^{\GL_n(k)} \cong k[X_1, \ldots, X_n] \text{ via } s_i \mapsfrom X_i.
 \]
\end{thrm}
\begin{proof}
 We already know that $s_1, \ldots, s_n \in \mc{P}(\Mat_n(k))$. That $s_1, \ldots, s_n$ are $\GL_n(k)$-invariant directly follows from the fact that the characteristic polynomial of a matrix is invariant under conjugation.
 
 Next we show that $s_1, \ldots, s_n$ are algebraically independent. For this let $p \in k[X_1, \ldots, X_n]$ with $p(s_1, \ldots, s_n) = 0$. This means that
 \[
  p(s_1, \ldots, s_n)(A) = 0 \text{ for all } A \in \Mat_n(k).
 \]
 For
 \[
  D := \{\diag(d_1, \ldots, d_n) \mid d_1, \ldots, d_n \in k\}
 \]
 we therefore have
 \[
  p(s_1, \ldots, s_n)(A) = 0 \text{ for all } A \in D.
 \]
 By the observations above this means that 
 \[
  p(e_1, \ldots, e_n)(d_1, \ldots, d_n) = 0 \text{ for all } d_1, \ldots, d_n \in k.
 \]
 Because $k$ is infinite this mean that $p(e_1, \ldots, e_n) = 0$. Since $e_1, \ldots, e_n$ are algebraically independent we find that $p =0$.
 
 All that’s left to show is that $s_1, \ldots, s_n$ generate $\mc{P}(\Mat_n(k))^{\GL_n(k)}$ as a $k$-algebra. For this let $f \in \mc{P}(\Mat_n(k))^{\GL_n(k)}$. Because $f$ is a polynomial map there exists $p \in k[X_{11}, \ldots, X_{nn}]$ such that
 \[
  f(A) = p(a_{11}, \ldots, a_{nn}) \text{ for all } A = (a_{ij})_{1 \leq i,j \leq n} \in \Mat_n(k).
 \]
 We define $\bar{p} \in k[t_1, \ldots, t_n]$ as $\bar{p} := \varphi(p)$ where
 \[
  \varphi : k[X_{11}, \ldots, X_{nn}] \to k[t_1, \ldots, t_n]
 \]
 is the algebra homomorphism defined by
 \[
  \varphi(X_{ij}) =
  \begin{cases}
   t_i & \text{if } i = j, \\
     0 & \text{otherwise}.
  \end{cases}
 \]
 By definition of $\bar{p}$ we have
 \[
  f(\diag(d_1, \ldots, d_n)) = \bar{p}(d_1, \ldots, d_n) \text{ for all } d_1, \ldots, d_n \in k.
 \]
 
 We claim that $\bar{p}$ is symmetric, i.e. $\bar{p} \in k[t_1, \ldots, t_n]^{S_n}$. Since $k$ is infinite we have the usual $S_n$-equivariant algebra isomorphism
 \[
 \Phi: k[t_1, \ldots, t_n] \to \mc{P}(k^n).
 \]
 Thus it is enough to show that the corresponding polynomial function $\Phi(\bar{p})$ is $S_n$-invariant. This follows from the fact that $f$ is $\GL_n(k)$-invariant: For all $\pi \in S_n$ we define the corresponding permutation matrix $A_\pi \in \GL_n(k)$ via
 \[
  A_\pi e_i = e_{\pi(i)}.
 \]
 For all $\pi \in S_n$ and $d_1, \ldots, d_n \in k$ we then have
 \[
  A_\pi \diag(d_1, \ldots, d_n) A_\pi^{-1}
  = \diag\left(d_{\pi^{-1}(1)}, \ldots, d_{\pi^{-1}(n)}\right).
 \]
 Therefore we have for all $\pi \in S_n$ and $(d_1, \ldots, d_n) \in k^n$
 \begin{align*}
  &\,(\pi.\Phi(\bar{p}))((d_1, \ldots, d_n)) \\
  =&\, \Phi(\bar{p})\left(\pi^{-1}.(d_1, \ldots, d_n)\right)
  = \Phi(\bar{p})\left(\left(d_{\pi(1)}, \ldots, d_{\pi(n)}\right)\right) \\
  =&\, \bar{p}\left(d_{\pi(1)}, \ldots, d_{\pi(n)}\right)
  = f\left(\diag\left(d_{\pi(1)}, \ldots, d_{\pi(n)}\right)\right) \\
  =&\, f\left(A_{\pi^{-1}} \diag(d_1, \ldots, d_n) A_\pi\right)
  = f\left(A_\pi^{-1} \diag(d_1, \ldots, d_n) A_\pi\right) \\
  =&\, f\left(A_\pi^{-1}.\diag(d_1, \ldots, d_n)\right)
  = (A_\pi.f)(\diag(d_1, \ldots, d_n)) \\
  =&\, f(\diag(d_1, \ldots, d_n))
  = \bar{p}(d_1, \ldots, d_n) \\
  =&\, \Phi(\bar{p})((d_1, \ldots, d_n)).
 \end{align*}
 This shows that $\Phi(\bar{p})$ is $S_n$-equivariant. So $\bar{p}$ is symmetric.
 
 Since $e_1, \ldots, e_n$ generate $k[t_1, \ldots, t_n]^{S_n}$ there exists $q \in k[X_1, \ldots, X_n]$ such that
 \[
  \bar{p} = q(e_1, \ldots, e_n),
 \]
 therefore
 \begin{align*}
  f(\diag(d_1, \ldots, d_n))
  &= q(e_1, \ldots, e_n)(d_1, \ldots, d_n) \\
  &= q(s_1, \ldots, s_n)(\diag(d_1, \ldots, d_n))
 \end{align*}
 for all $d_1, \ldots, d_n \in k$. This show that $f-q(s_1, \ldots, s_n) = 0$ when restricted to $D$.
 
 \begin{claim}[Zariski density property 2]
  If $h \in \mc{P}(\Mat_n(k))^{\GL_n(k)}$ (where $\GL_n(k)$ acts on $\Mat_n(k)$ by conjugation) and $h_{|D} = 0$ then $h = 0$.
 \end{claim}

 From this claim and the results so far we find that $f - q(s_1, \ldots, s_n) = 0$ and thus $f = q(s_1, \ldots, s_n)$. (As before we will leave the proof of the claim for later and end this proof here.)
\end{proof}


Another ``famous'' family of polynomial functions in $\mc{P}(\Mat_n(k))^{\GL_n(k)}$ (where $\GL_n(k)$ acts on $\Mat_n(k)$ by conjugation) are the so called power traces. The $m$-th power trace is defined as
\[
 \tr_m : \Mat_n(k) \to k, A \mapsto \tr A^m.
\]
for all $m \geq 0$. It is clear that $\tr_m \in \mc{P}(\Mat_n(k))^{\GL_n(k)}$ for all $m \geq 0$ and that $\tr_m$ is $\GL_n(k)$-invariant follows from the fact that for all $S \in \GL_n(k)$ and $A \in \Mat_n(k)$
\begin{align*}
 \left(S.\tr_m\right)(A)
 &= \tr_m\left(S^{-1}.A\right)
 = \tr_m\left(S^{-1} A S\right) \\
 &= \tr \left(S^{-1} A S\right)^m
 = \tr \left(S^{-1} A^m S\right) \\
 &= \tr A^m
 = \tr_m(A).
\end{align*}


\begin{thrm}
 Let $n \geq 1$ and let $k$ be an infinite field with either $\kchar k = 0$ or $\kchar k > n$. Let $\GL_n(k)$ act on $\Mat_n(k)$ by conjugation. Then $\tr_1, \ldots, \tr_n$ generate $\mc{P}(\Mat_n(k))^{\GL_n(k)}$ as a $k$-algebra and are algebraically independent, i.e.
 \[
  \mc{P}(\Mat_n(k))^{\GL_n(k)} \cong k[X_1, \ldots, X_n] \text{ via } \tr_i \mapsfrom X_i.
 \]
\end{thrm}
\begin{proof}
 As before let 
 \[
  D := \{\diag(d_1, \ldots, d_n) \mid d_1, \ldots, d_n \in k\}.
 \]
 For every diagonal matrix $\diag(d_1, \ldots, d_n) \in D$ we have
 \[
  \tr_m(\diag(d_1, \ldots, d_n)) = d_1^m + \ldots + d_n^m = p^{(n)}_m(d_1, \ldots, d_n)
 \]
 for all $m \geq 0$. Since $k$ is a field with $\kchar k = 0$ or $\kchar k > n$ we know that $p^{(n)}_1, \ldots, p^{(n)}_n$ generate $k[X_1, \ldots, X_n]^{S_n}$ as a $k$-algebra and are algebraically independent.
 
 Using these observations we can now prove the theorem by simpley copying the proof of the previous theorem and replacing $s_i$ with $\tr_i$ and $e_i$ with $p_i$.
\end{proof}





\section{Zariski dense subsets}


\begin{defi}
 Let $W$ be a finite-dimensional $k$-vector space. A subset $X \subseteq W$ is called Zariski dense if for any $f \in \mc{P}(W)$ we have
 \[
  f_{|X} = 0 \Rightarrow f = 0.
 \]
 If $X \subseteq Y \subseteq W$ then $X$ is Zariski dense in $Y$ if for all $f \in \mc{P}(W)$
 \[
  f_{|X} = 0 \Rightarrow f_{|Y} = 0.
 \]
\end{defi}


\begin{expl}
 Let $k$ be a field and $X \subseteq k$ an infinite subset. Then $X$ is Zariski dense in $k$: Let $f \in \mc{P}(k)$ with $f_{|X} = 0$. Since $f$ is a polynomial function there exists $p \in k[X]$ with $f(\lambda) = p(\lambda)$ for all $\lambda \in k$. Since $f_{|X} = 0$ and $X$ is infinite the polynomial $p$ has infinitely many zeroes. Therefore $p = 0$ and $f = 0$.
\end{expl}


\begin{lem}
 Let $k$ be a infinite field and $L/k$ a field extension. Then the subset $k^n \subseteq L^n$ is Zariski dense in the $L$-vector space $L^n$ for all $n \geq 1$.
\end{lem}
\begin{proof}
 We show the statement by induction over $n$.
 
 ($n = 1$) For $f \in \mc{P}(L)$ with $f_{|k} = 0$ the corresponding polynomial $p \in L[X]$ (i.e. $f(\lambda) = p(\lambda)$ for all $\lambda \in L$) has infintely many zeroes. So $p = 0$ and therefore $f = 0$.
 
 ($n \geq 2$) Suppose the statement holds for $n-1$. Let $f \in \mc{P}\left(L^n\right)$ with $f_{|k^n} = 0$. Because $f$ is a polynomial function there exists $p \in L[X_1, \ldots, X_n]$ with $f((\lambda_1, \ldots, \lambda_n)) = p(\lambda_1, \ldots, \lambda_n)$ for all $(\lambda_1, \ldots, \lambda_n) \in L^n$. We can write $p$ as
 \[
  p(X_1, \ldots, X_n) = \sum_{i \in \N} p_i(X_1, \ldots, X_{n-1}) X_n^i
 \]
 with $p_i \in L[X_1, \ldots, X_{n-1}]$ for all $i \in \N$ and $p_i \neq 0$ for only finitely many $i \in \N$. Let $(\lambda_1, \ldots, \lambda_{n-1}) \in k^{n-1}$ be fixed but arbitrary and $\bar{p} \in L[X_n]$ be defined as
 \[
  \bar{p}(X_n) := p(\lambda_1, \ldots, \lambda_{n-1}, X_n) = \sum_{i \in \N} p_i(\lambda_1, \ldots, \lambda_{n-1}) X_n^i.
 \]
 Since $\bar{p}(\lambda_n) = 0$ for all $\lambda_n \in k$ we find that $\bar{p} = 0$ (since $\bar{p}$ has infinitely many zeroes). Because $L$ is infinite this means that $p_i(\lambda_1, \ldots, \lambda_{n-1}) = 0$ for all $i \in \N$. Since $(\lambda_1, \ldots, \lambda_{n-1}) \in k^{n-1}$ is arbitrary we find that $p_i(\lambda_1, \ldots, \lambda_{n-1}) = 0$ for all $\lambda_1, \ldots, \lambda_{n-1} \in k$ for every $i \in \N$. By using the induction hypothesis we find that $p_i(\lambda_1, \ldots, \lambda_{n-1}) = 0$ for all $i \in \N$ and $\lambda_1, \ldots, \lambda_{n-1} \in L$. Because $L$ is infinite it follows that $p_i = 0$ for all $i \in \N$. Therefore $p = 0$ and $f = 0$.
\end{proof}


\begin{lem}
 Let $W$ be a finite-dimensional $k$-vector space for an infinite field $k$. Let $h \in \mc{P}(W)$ with $h \neq 0$ and
 \[
  W_h := \{w \in W \mid h(w) \neq 0\}.
 \]
 Then $W_h$ is Zariski dense (in $W$).
\end{lem}
\begin{proof}
 Let $f \in \mc{P}(W)$ with $f_{|W_h} = 0$. Then
 \[
  (fh)(w) = f(w)h(w) = 0
 \]
 for all $w \in W$, so $fh = 0$. Since $\mc{P}(W) \cong k[X_1, \ldots, X_{\dim W}]$ we know that $\mc{P}(W)$ is an integral domain. Since $h \neq 0$ and $fh = 0$ it follows that $f = 0$.
\end{proof}

























